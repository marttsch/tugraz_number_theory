\documentclass[NumTh.tex]{subfiles}
\begin{document}

\section*{Organizatorial stuff}

Dates (in TUGrazOnline):
\begin{table}[!h]
  \begin{tabular}{cccl}
    Mon & 14:15--15:45 & C208 & Exercises (starting 19.10. first exercise class) \\
    Tue & 14:15--15:45 & C307 & Lecture (starting 20.10. first (real) lecture) \\
    Wed & 08:15--09:45 & C208 & Lecture
  \end{tabular}
\end{table}

From now until 15.12. lectures by Martin Widmer.
Then C. Frei.

End: oral exams

Exercises: Find details on website of the instructor Dijana Kreso.
\url{math.tugraz.at/~kreso}

\section{Basics}

\begin{align}
  \N &= \{1,2,\dots\} \\
  \N_0 &= \N \cup \{0\}\\
  \Z &= \{ \dots,-2,-1,0,1,2,\dots\}
\end{align}

\subsection{Divisibility}
\begin{defi}
Let $a,b \in \Z$. $a$ divides $b$ (written $a\divides b$) if $\exists q \in \Z : b = qa$.
\newline
Some properties: Let $a,b,c \in \Z$. Then the following statements hold:
\begin{align}
  a\divides b &\Rightarrow ac\divides bc \\
  a\divides b \land b\divides c &\Rightarrow a\divides c \\
  a\divides b \land b\divides a &\iff a = \pm b \\
  a\divides b \land a\divides c &\Rightarrow a\divides (b+c)
\end{align}
\end{defi}

\begin{defi}[Remainder]
Let $a \in \Z$, $b \in \N$. Then there are unique $q,r \in \Z$ such that
\[a = qb + r \text{ and } 0 \leq r < b.\]
\end{defi}

\begin{rem}
\begin{enumerate}
  \item $b\divides a \Leftrightarrow r = 0$
  \item $q = \lfloor \frac{a}{b} \rfloor$ (largest integer $\leq \frac{a}{b}$)
  \item we will sometimes write: $a \mod b \coloneq c$
\end{enumerate}
\end{rem}

\begin{defi}
Let $a_1,a_2,\dots,a_n,d \in \mathbb{Z}$. $d$ is a greatest common divisor (gcd) of $a_1,\dots,a_n$ if
$d\divides a_i$ $\forall 1\leq i \leq n$,
and for every $e \in \Z$ with $e\divides a_i$ $\forall 1\leq i \leq n$, $e\divides d$.
\end{defi}

\begin{rem}

\begin{enumerate}
  \item a $\gcd$ of $a_1,\dots,a_n$ is unique up to sign
  \item we write $d = \gcd(a_1,\dots,a_n)$ if $d$ is a $\gcd$ of $a_1,\dots,a_n$
  \item for $a_1,\dots,a_n \in \Z$, a $\gcd$ exists and can be written as a linear combination of $a_1,\dots,a_n$,
  i.e., $\exists x_1,\dots,x_n \in \Z$ such that $$\gcd(a_1,\dots,a_n) = x_1 a_1+ \dots + x_n a_n$$
  \item $\gcd(a_1,\dots,a_n) = \gcd(\gcd(a_1,\dots,a_{n-1}),a_n)$
  \item if $a\divides bc$ and $\gcd(a,b) = 1$ then $a\divides c$.
  \item let $a^\prime \coloneq \frac{a}{\gcd(a,b)}$, $b^\prime = \frac{b}{\gcd(a,b)}$. Then $\gcd(a^\prime,b^\prime) = 1$
\end{enumerate}
\end{rem}
\todo{Hier verwendest du $\coloneq $, sonst aber nur $=$, evtl. einheitlich machen für alle Definitionen?}

\begin{algorithm}
  \caption{Compute the $\gcd$ of two integers: Euclidean algorithm}
  \begin{algorithmic}
    \Require{$a,b \in \Z$. $\abs{a} \geq \abs{b}$}
    \Ensure{$a \coloneq \gcd(a,b)$}
    \State replace $a$ by $\abs{a}$, $b$ by $\abs{b}$
    \While{$b \neq 0$}
      \State write $a = qb +r$, $0 \leq r < b$
      \State $a \coloneq b$
      \State $b \coloneq r$
    \EndWhile
    \State \Return $a$
  \end{algorithmic}
\end{algorithm}

The algorithm is correct, since $\gcd(a,b) = \gcd(b,a \mod b)$. \\
The algorithm terminates because $b$ decreases in each step. \\
The algorithm is fast: ($\mathcal{O}(\log b)$)

\todo[caption={}]{sollte ausgebessert werden, 1. $O(logn)$ steps, 2. stimmt nur wenn $\abs{r} \leq b/2$}

The Euclidean algorithm also allows us to find $x, y$ such that $\gcd(a,b) = ax+by$ by doing all computations backwards.

\begin{ex}
  $\gcd(56,22) = \,?$
  \begin{align*}
    a &= 56, b = 22\\
    56 &= 2 \cdot 22 + 12\\
    a &= 22, b = 12 \neq 0\\
    22 &= 1 \cdot 12 + 10\\
    a &= 12, b = 10 \neq 0\\
    12 &= 1 \cdot 10 + 2\\
    a &= 10, b = 2 \neq 0\\
    10 &= 5 \cdot 2 + 0\\
    a &= 2, b = 0 & \Rightarrow \gcd(56,22) = 2
  \end{align*}
  Doing the computations backwards:
  \begin{align*}
    2 &= 12 -10 = 12 - (22 - 12) = -22 + 2 \cdot 12 = -22 + 2(56-2 \cdot 22) = 2 \cdot 56 - 5 \cdot 22\\
    x &= 2, y = -5
  \end{align*}
\end{ex}

\begin{app*}[linear diophantine equations]
Let $a,b,c \in \Z$, $a,b,c \neq 0$. Find all $(x,y) \in \Z^2$ which satisfy
\begin{align}
  \label{eq:lde}
  ax+by = c.
\end{align}
\end{app*}

\begin{description}
  \item[Existence of solution] let $d = \gcd(a,b)$.
    \[
	    (d\divides a \Rightarrow d\divides xa) \land
	    (d\divides b \Rightarrow d\divides yb)
	\] \[
	    \Rightarrow d\divides xa+yb = c \\
	\] \[
	    \Rightarrow \cref{eq:lde}
	\]
	can have solutions only if $d\divides c$.

  \item[Solution in case $d=1$]
    Let $x_0,y_0 \in \Z$ such that $ax_0+by_0 = 1$ using the Euclidean algorithm.
    Then from $acx_0 + bcy_0 = c$ the solution $(cx_0, cy_0)$ of (\cref{eq:lde})
    follows: for all $n \in \Z: (x,y) \coloneq (cx_0+nb,cy_0+na)$ is a solution.
  
	Indeed,
	\begin{align*}
	  ax+by &= acx_0 + anb + bcy_0 - bna = c &\checkmark
	\end{align*}
    These $(x,y)$ are all solutions: let $(x,y)$ be a solution. Then
	\begin{align*}
	   ax + by &= c \\
	   acx_0 + bcy_0 &= c \\
	   \Rightarrow a(x-cx_0) &= b(cy_0-y)
	\end{align*}
	\[ \gcd(a,b) = 1 \:\Rightarrow\: b \divides x - cx_0 \:\Rightarrow\: x = cx_0 + nb, n \in \Z \]
	\[ \Rightarrow\: a \divides cy_0 - y \:\Rightarrow\: y = cy_0 + ma, m\in \Z \]
    \[ c = ax+by = acx_0 + anb + bcy_0 + bma \]
    \[ = c + (n+m)ab \Rightarrow (n+m)ab = 0 \Rightarrow m = -n \]
  
  \item[Solutions in the general case] Assume $d = \gcd(a,b)$ and $d\divides c$, let
    \[
		a^\prime = \frac{a}{d} \hspace{25pt}
		b^\prime = \frac{b}{d} \hspace{25pt}
		c^\prime \coloneq \frac{c}{d}
	\]
    Then $\gcd(a^\prime, b^\prime) = 1$ and the solution to (\cref{eq:lde}) is exactly the solution of $a^\prime x + b^\prime y = c^\prime$.
\end{description}

\subsection{Primes}

\todo{1. Beistriche für bessere Lesbarkeit 2. faustregel, vor und nach ``i.e.'' gehört eigentlich beistrich}
\begin{defi}
$p \in \N$, $p > 1$ is a \emph{prime number} if the only positive divisors of $p$ are $1$ and $p$, i.e., $a \in \N$, $a\divides p \Rightarrow a \in \{1,p\}$.
$\PP \coloneq \{ primes\} \subset \N, \PP = \{2,3,5,7,11,13,\dots\}$.
$p$ prime and $p\divides ab \Rightarrow p\divides a$ or $p\divides b$
\end{defi}

\begin{theorem}[Fundamental theorem of arithmetic]
Every $n \in \N$ can be written uniquely (up to reordering) as a product of primes.
i.e. there are distinct primes $p_1,\dots,p_l$, and $\alpha_1,\dots,\alpha_l \in \N$ such that
$n = p_1^{\alpha_1} \dots p_l^{\alpha_l}$
\end{theorem}

\begin{proof}[Sketch]\hfill{}
  \begin{description}
    \item[Existence]
      let $p_0 > 1$ be the smallest divisor $> 1$ of $n$. Then $p_0$ is prime.
      $n = p_0 n_0$, induction $\checkmark$

    \item[Uniqueness]
      let $p_1 \dots p_m = q_1 \dots q_l = n$, $p_i, q_j$ primes.
      $p_1 \divides  q_1 \dots q_l \Rightarrow \exists i: p_1\divides q_i$, both prime $\Rightarrow p_1 = q_i$, wlog: $i = 1$.
      $p_1\dots p_m = q_1 \dots q_l$, induction $\checkmark$
  \end{description}
\end{proof}

\begin{theorem}[Euclid]
There are $\infty$-many primes.
\end{theorem}

\begin{proof}
  Given primes $p_1,\dots,p_n \in \PP$. We construct one more prime \[N \coloneq p_1 \cdot \dots \cdot p_n + 1 \text{.}\]
  Assume $P$ is a prime factor of $N$.
  If $P \in \{p_1, \dots ,p_n\}$ then $P\divides N$ and $P\divides p_1\dots p_n \Rightarrow P\divides 1$ $\lightning$
\end{proof}

\begin{rem}[prime factors and $\gcd$s]
Let $a_1,\dots, a_n \in \Z$, write
\[a_i = \prod_{p \in \PP} p^{\alpha_{p,i}}\text{, }\alpha_{p,i} \in \N_0 \text{,}\] almost all $\alpha_{p,i} = 0$,
then \[\gcd(a_1,\dots,a_n) = \prod_{p \in \PP} p^{\min_{1\leq i\leq n} \{\alpha_{p,i}\}}\]
\end{rem}

\subsection{Congruences}

All rings are commutative with $1$.
\todo{zu Begriffserklärung oder zum ersten Vorkommen von Ringen}

\begin{defi}
Let $a,b \in \Z$, $n \in \N$. Then \emph{$a$ is congruent to $b \pmod{n}$}, $a \equiv b \pmod{n}$,
if $n\divides  a-b$.
We write $\overbar{a} = [a]_n \coloneq \{b \in \Z : b \equiv a (\mod n)\}$
\end{defi}

\begin{rem}
\label{four}
\begin{enumerate}
  \item Congruence $\pmod{n}$ is an \emph{equivalence relation}
  \item $\overbar{0},\overbar{1},\dots,\overline{n-1}$ is a partition of $\Z$.
  \item if $a \equiv b \pmod{n}$, $c \equiv d \pmod{n}$, then 
  $-a \equiv -b \pmod{n}$,
  $a \overset{+}{\underset{-}{\cdot}} d \equiv \overset{\pm}{\cdot} \pmod{n}$.
  \todo{fix + - *}
\end{enumerate}
\end{rem}

\todo{a ring }
\begin{defi}
$\Z / n\Z = \Z_n \coloneq \{[a]_n : a\in \Z\} = \{\overbar{0},\overbar{1},\dots,\overline{n-1}\}$ residue class ring modulo $n$
\end{defi}

\begin{rem}
$\Z_n$ is a ring with operation $\overbar{a} \overset{+}{\underset{-}{\cdot}} \overbar{b} \coloneq \overline{a\overset{+}{\underset{-}{\cdot}}b}$ (well defined due to item~3 of~\cref{four})
$\Z_n^\times = \{ \overbar{a} \in \Z_n : \exists \overbar{b} \in \Z_n : \overbar{a}\overbar{b} = \overbar{1} \}$ ... group of units $\mod n$
\end{rem}

\begin{lemma}
Let $a \in \Z$. Then $\overbar{a} \in \Z_n^\times \Leftrightarrow \gcd(a,n) = 1$.
\end{lemma}

\begin{proof}\hfill
\begin{itemize}
  \item[``$\Rightarrow$'']
    $\overbar{a}\overbar{b} = \overbar{1} \Leftrightarrow ab \equiv 1 \pmod{n} \Leftrightarrow n\divides ab-1$ \\
    $\Rightarrow$ no prime factor of $n$ divides $a$ \\
    $\Rightarrow \gcd(a,n) =1$.
  \item[``$\Leftarrow$'']
    $1 = \gcd(a,n) = ax + ny \Rightarrow \overbar{1} = \overbar{a}\overbar{x}$
\end{itemize}
\end{proof}

\begin{rem}
The inverse of $\overbar{a}$ can be computed by the Euclidean algorithm.
\end{rem}

\begin{ex}[Simultaneous congruences]
Find $x \in \Z$ such that 
\begin{align}
x &\equiv 2 \pmod{3}\\
x &\equiv 1 \pmod{5}\\
x &\equiv 0 \pmod{7}
\end{align}
\end{ex}

\begin{theorem}[Chinese remainder theorem (CRT)] Let
\[ n_1,\dots,n_l \in \N \text{ subject to } \gcd (n_i,n_j) = 1 \;\forall i\neq j \]
\[ x_1, \dots, x_l \in \Z \text{.} \]
Then
\[ \exists x \in \Z \text{ such that } x \equiv x_i \pmod{n_i} \; \forall 1\leq i \leq l \]
where $x$ is unique modulo $n_1 \cdot \dots \cdot n_l$.
\end{theorem}

\begin{proof}
How to compute $x$? For $i \in \{1,\dots,l\}$, let
\[ N_i \coloneq \prod_{j \neq i} n_j = n_1 \dots n_{i-1} n_{n+1} \dots n_l \]
and let
\[ N \coloneq \prod_i n_i = n_1 N_1 = n_2 N_2 = \dots = n_l N_l \]
because $\gcd(n_i,N_i) = 1 \Rightarrow N_i$ in invertible $\bmod n_i$. Let
\[ m_i N_i \equiv 1 \pmod{n_i} \]
and let
\[ x \coloneq N_1 m_1 x_1 + \dots + N_l m_l x_l \text{.} \]
We have $N_i m_i x_i \equiv 0 \pmod{n_j, j\neq i}$%\\ x_i \mod m_i$
\end{proof}

\begin{ex}
  \[
    n_1 = 3, \hspace{20pt} n_2 = 5, \hspace{20pt} n_3 = 7
  \] \[
    x_1 = 2, \hspace{20pt} x_2 = 1, \hspace{20pt} x_3 = 0
  \] \[
    N_1 = 35, \hspace{20pt} N_2 = 21, \hspace{20pt} N_3 = ?
  \]
  \[ \overbar{m}_1 = \inv{\overbar{35}} \pmod{3} = \inv{\overbar{2}} \pmod{3} = \overbar{2} \pmod{3} \Rightarrow m_1 = 2 \]
  \[ \overbar{m}_2 = \inv{\overbar{21}} \pmod{5} = \inv{\overbar{1}} \pmod{5} = \overbar{1} \pmod{5} \Rightarrow m_2 = 1 \]
  \begin{align*}
    x &= 35 \cdot 2 \cdot 2 + 21 \cdot 1 \cdot 1 + 0 \\
      &= 140 + 21 \\
      &= 161 \\
      &\equiv 56 \pmod{105}
  \end{align*}
\end{ex}

\begin{ex}[more abstract CRT]
  Let $n_1,\dots,n_l \in \N$, with $\gcd(n_i,n_j)= 1$ $\forall i \neq j$.
  There is a ring isomorphism $f: \Z_{n_1 \dots n_l} \overset{\simeq}{\mapsto} \Z_{n_1} \times \dots \times \Z_{n_l}$ that satisfies
  $f([a]_{n_1 \dots n_l}) = ([a]_{n_1},\dots,[a]_{n_l})$ $\forall a \in \Z$.
  In particular: $\Z_{n_1 \dots n_l}^\times \cong \Z_{n_1}^\times \times \dots \times \Z_{n_l}^\times$ (restrict $f$ to $\Z_{n_1 \dots n_l}^\times$)
\end{ex}


\subsection{Arithmetic functions}

\begin{defi}
  $f: \N \to \C$ is an \emph{arithmetic function}.
  $f$ is \emph{multiplicative} if $\forall m,n $ it holds that $\gcd(m,n) = 1$. We have $f(mn) = f(m) f(n)$.
  $f$ is \emph{completely multiplicative} if $\forall m,n: f(mn) = f(m) f(n)$.
  Let $f: \N \to \C$. Its \emph{summatory function} is $S_{f}(n) \coloneq \sum_{d\divides n} f(d)$.
\end{defi}
\todo{add lemma}
\begin{proof}
  If $\gcd(m,n) = 1$ and $d\divides mn$, then $\exists$ unique $d_1,d_2$ such that $d = d_1 \cdot d_2$ with $d_1 \divides  m$, $d_2\divides n$.
  \[S_f (mn) = \sum_{d\divides mn} f(d) = \sum_{d_1\divides m} \sum_{d_2\divides n} f(d_1 d_2) = \sum_{d_1\divides m} f(d_1) \sum_{d_2\divides n} f(d_2) = S_f(m) S_f(n)\]
\end{proof}

\begin{ex}
  \begin{align*}
    \tau(n) &\coloneq S_1(n) = \sum_{d\divides n} 1 &\text{\dots number of divisors of $n$} \\
    \sigma(n) &\coloneq S_{id}(n) = \sum_{d\divides n} d &\text{\dots divisor sum of $n$}
  \end{align*}
\end{ex}

\begin{defi}
  The function $\phi(n) \coloneq \card{\Z_n^\times}$ is called \emph{Euler's $\phi$-function}.
\end{defi}

\begin{rem}
  \begin{enumerate}
    \item $\phi(n) = \card{\set{ 0 \leq a < n : \gcd(a,n) = 1}}$
    \item $\phi$ is multiplicative (CRT: $\gcd(m,n) = 1$. $\Z_{nm}^\times \cong \Z_n^\times \times \Z_m^\times$)
    \item $\phi(p) = p - 1$ ($\Z_p$ is a field)
  \end{enumerate}
\end{rem}

\begin{lemma}
  $\phi(p^n) = p^n - p^{n-1}$
\end{lemma}

\begin{proof}
  \begin{align*}
    \phi(p^n) &= \card{\set{0 \leq a < p^n}} - \card{\set{0 \leq a < p^n : \gcd(a,p^n) \neq 1}} \\
              &= p^n - \card{\set{0 \leq a < p^n : p|a}} \\
              &= p^n - p^{n-1}
  \end{align*}
\end{proof}

\begin{prop}
  If $n = p_1^{\alpha_1} \dots p_l^{\alpha_l}$ with $p_i \neq p_j$ primes, $\alpha_i \in \N$.
  Then
  \[ \phi(n) = \prod_{i=1}^l p_i^{\alpha_i} (1 - \frac{1}{p_i}) = n \prod_{p\divides n} (1 - \frac{1}{p}) \]
\end{prop}

\begin{theorem}[Euler-Fermat]
  Then $a^{\phi(n)} \equiv 1 \mod n$.
  In particular: $a^{p-1} \equiv 1 \mod p$ $\forall p \nmid a$ (little Fermat).
\end{theorem}

\begin{proof}[Proof 1]
  Lagrange's Theorem, $G = \Z_n^\times, \overbar{a} \in G \Rightarrow \overbar{a}^{|G|} = \overbar{1}, |G| = \phi(n)$.
\end{proof}

\begin{proof}[Proof 2]
  $\prod_{x \in \Z_n^\times} x = \prod_{x \in \Z_n^\times} (\overbar{a}x) = \overbar{a}^{\phi(n)} \prod_{x \in \Z_n^\times} x \Rightarrow a^{\phi(n)} \equiv 1 \mod n$ 
\end{proof}

\begin{defi}
  The Möbius function $\mu: \N \to \{-1,0,+1\}$ is defined as
  \[
	\mu(n) = \begin{cases}
	  (-1)^l & n = p_1 \dots p_l, p_i \neq p_j, i \neq j, p_i \text{ primes} \\
	  0       & \text{otherwise i.e. if } \exists p: p^2 \divides n
	\end{cases}
  \]
\end{defi}

\begin{rem}\hfill
  \begin{enumerate}
    \item $\mu(1) = 1$, $\mu(2) = -1$, $\mu(3) = -1$, $\mu(4) = 0$, $\mu(5) = -1$, $\mu(6) = 1$, $\dots$
    \item $\mu$ is \emph{multiplicative}
  \end{enumerate}
\end{rem}

\begin{lemma}
  \[
    S_\mu(n) = \begin{cases}
      1 & \text{if } n=1 \\
      0 & \text{if } n>1
    \end{cases}
  \]
\end{lemma}

\begin{proof}
  \[ S_\mu(1) = \sum_{d\divides 1} \mu(d) = \mu(1) = 1 \]
  By multiplicativity, it suffices to prove $S_\mu(p^n) = 0$ $\forall p,n$.
  \begin{align*}
    S_\mu(p^n) &= \sum_{d\divides p^n} \mu(d) \\
               &= \sum_{i=0}^n \mu(p^i) \\
               &= \mu(1) + \mu(p) + 0 +\dots+0 \\
               &= 0
  \end{align*}
\end{proof}

\begin{theorem}[Möbius inversion formula]
  Let $f: \N \to \C$. Then
  \[ f(n) = \sum_{d\divides n} \mu(d) S_f(\frac{n}{d}) \text{.}\]
\end{theorem}

\begin{proof}
  \begin{align*}
    \sum_{d\divides n} \mu(d) S_f\left(\frac{n}{d}\right)
      &= \sum_{d\divides n} \mu(d) \sum_{e\divides \frac{n}{d}} f(e) \\
      &= \sum_{e\divides n} f(e) \sum_{\substack{d\divides n \\ s.t. e\divides \frac{n}{d}}} \mu(d) \\
      \text{For the next step we use }
		    &d\divides n \land e\divides \frac{n}{d}
		    \Leftrightarrow e d \divides n
		    \Leftrightarrow e\divides n \land d\divides \frac{n}{e}\\
      &= \sum_{e\divides n} f(e) \sum_{d\divides  \frac{n}{e}} \mu(d)\\
      &= f(n)\\
      \text{since }
        \sum_{d\divides  \frac{n}{e}} \mu(d) =
		&\begin{cases}
		  1 & \frac{n}{e} = 1 \\
		  0 & \text{otherwise}
		\end{cases}
  \end{align*}
\end{proof}


\subsection{Structure of $\Z_n^\times$}

\[ n = p_1^{\alpha_1} \dots p_l^{\alpha_l} \text{ with } p_i \neq p_j, i \neq j, \alpha_i \in \N \text{ where } p_i \text{ are primes} \]
From the CRT it follows that $\Z_n^\times \cong \Z_{p_1^{\alpha_1}}^\times \times \dots \times \Z_{p_l^{\alpha_l}}^\times$.
So we only consider prime powers $p^\alpha$, $p \in \PP$, $\alpha \in \N$

\subsubsection{Case 1: $\alpha = 1$}

\begin{theorem}
  $\Z_p^\times$ is cyclic, i.e. $\Z_p^\times \cong \Z_{(p-1)}$
\end{theorem}

\begin{proof}
  Use structure theorem for finite abelian groups. If $G$ is a finite abelian group then $\exists d_1,\dots d_l \in \N$ such that
  $1 < d_1\divides d_2\divides d_3\divides \dots\divides d_l$, and $G \cong \Z_{d_1}^\times \times \dots \times \Z_{d_l}^\times$
  thus, $\Z_p^\times \cong \Z_{d_1}^\times \times \dots \times \Z_{d_l}^\times$ 
  (every element $x \in \Z_{d_1}^\times \times \dots \times \Z_{d_l}^\times$ satisfies $d_l x = 0$
  $\Rightarrow$ every $x \in \Z_p^\times$ satisfies $x^{d_l} = 1$).
  $x^{d_l} -1$ is a polynomial of degree $d_l$ over the field $\Z_p \Rightarrow x^{d_l} -1$ has $\leq d_l$ roots $\Rightarrow p-1 \leq d_l$,
  but $p-1 = d_1 \dots d_l \Rightarrow l = 1, p-1 = d_l$
\end{proof}

\begin{rem}
  The same proof shows: Let $F$ be a field, $G \leq F^\times$, $|G| < \infty$. Then $G$ is cyclic.
\end{rem}

\subsubsection{Case 2: $\alpha \geq 2$; $p \geq 3$}

Denote $\abs{x}$ as the order of $x$ in $\Z_{p^\alpha}^\times$;
i.e. $\abs{x} = \min\set{l\in \N : x^l\equiv 1 \mod p^\alpha}$

$\card{\Z_{p^\alpha}^\times} = \phi(p^\alpha) = p^{\alpha -1}(p-1)$,
find $x,y \in \Z_{p^\alpha}^\times$ such that $\abs{x} = p^{\alpha-1}$, $\abs{y} = p-1$
then $\abs{xy} = \abs{x}\abs{y} = p^{\alpha-1}(p-1)$, since $\gcd(\abs{x},\abs{y}) = 1$

\begin{lemma}
  \[ (1+p)^{p^{n-1}}
    \begin{cases}
      \equiv 1 \mod p^n \\
      \nequiv 1 \mod p^{n+1}
  \end{cases} \]
\end{lemma}

\begin{proof}
  Proof by induction
  \begin{itemize}
    \item[$n = 1$] $\checkmark$
    \item[$n \to n+1$]
      \begin{align*}
        (1+p)^{p^{n-1}} &= 1 + a p^n, p\nmid a \\
        (1+p)^{p^n} &= (1 + a p^n)^p \\
                    &= 1 + pap^{n} + \sum_{i=2}^{p-1} {p \choose i} (ap^n)^i + (ap^n)^p
      \end{align*}
      \[
          p^{np} \divides \bullet, \hspace{10pt}
          np \geq n+2, \hspace{10pt}
          (\text{or } p \geq 3), \hspace{10pt}
          p^{2n+1} \divides \bullet, \hspace{10pt}
          2n+1 \geq n+2
      \]
      \[
        p\divides{p \choose i} = \frac{p!}{i!(p-i)!}, 1 \leq i < p
        \Rightarrow (1+p)^{p^n} \equiv 1 + ap^{n+1} \pmod{p^{n+2}}, p \nmid a
      \]
  \end{itemize}
\end{proof}

$2 \times$ Lemma: $x = 1+p$ satisfies $\abs{x} = p^{\alpha -1}$, now find $y$.

\begin{enumerate}
  \item $\exists z \in \Z: \abs{\overbar{z}} = p-1 \text{ is } \Z_p^\times$
  \item let $l \coloneq \card{\overbar{z}} \text{ is } \Z_{p^\alpha}^\times$
  \item Then $p^\alpha \divides  z^l -1 \Rightarrow z^l \equiv 1 \mod p$
  \item $\Rightarrow$ $p-1 \divides  l$.
  \item Let $y \coloneq z^{\frac{l}{p-1}}$, then $\abs{\overbar{y}} = p-1$.
\end{enumerate}

We have proven: Theorem: $\Z_{p^\alpha}^\times$ is cyclic, i.e. $\Z_{p^\alpha}^\times \cong \Z_{p^{\alpha-1}(p-1)}$, if $p \geq 3, \alpha \geq 1$.

$p = 2$: $\Z_{2^\alpha}^\times \cong \set{0, \alpha = 1 \hspace{10pt} \Z_2, \alpha = 2 \hspace{10pt} \Z_2 \times \Z_{p^{\alpha -2}}, \alpha \geq 3}$

\begin{cor}
  Let $m \in \N$. Then $\Z_m^\times$ is cyclic iff $m$ has one of the following forms:
  \begin{itemize}
    \item $m = 2$
    \item $m = 4$
    \item $m = p^\alpha, p \geq 3, \alpha \in \N$
    \item $m = 2 p^\alpha, p \geq 3, \alpha \in \N$
  \end{itemize}
\end{cor}

In these cases a generator of $\Z_m^\times$ is called \emph{a primitive root modulo $m$}.

\end{document}
