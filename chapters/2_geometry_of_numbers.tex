\documentclass[NumTh.tex]{subfiles}
\begin{document}

\section{Geometry of Numbers}

\subsection{Basic notions}
Let $R$ be an integral domain (with $1$), and $n \in \N$.
We write $\mat_n(R) = \{ n \times n \text{ matrices with entries in } R\}$
and $\gl_n(R) = \{ A \in \mat_n(R): A \text{ is invertibel and } A^{-1} \in \mat_n(R) \}$.
Then $(\gl_n(R), \cdot)$ is a group. We say $u \in R$ is a unit (in $R$) if $\exists u^\prime \in R$ such that $u^\prime \cdot u = 1$.\\
If $A \in \gl_n(R)$ then $(\det A)^{-1} = \det A^{-1} \in R$. So $\det A$ is a unit in $R$.
On the other hand if $A \in \mat_n(R)$ and $\det A$ is a unit in $R$ then $A^{-1} = (\det a)^{-1} \adj(A) \in \mat_n(R)$ as the adjungate matrix $\adj(A)$ of $A$ clearly is in $\mat_n(R)$. 
Thus we have
\[ \gl_n(R) = \{ A \in \mat_n(R) : \det A \text{ is a unit in } R \} \text{.} \]
In particular, $\gl_n(\Z) = \{ A \in \mat_n(\Z): \det A = \pm 1 \}$.\\
\\
Let $n \in \N$. A \underline{lattice} $\Lambda$ in $\R^n$ is a set of the form
\[ \Lambda = A \Z^n = \{ Ax : x \in \Z^n \} \]
where $A \in \gl_n(\R)$.
The column vectors of $A$ are called a \underline{basis} of $\Lambda$.

\begin{lemma}[2.1.1 \label{l_2_1_1}]
  Let $A,B \in \gl_n(\R)$. Then 
  \[ A \Z^n = B \Z^n \iff \exists T \in \gl_n(\Z) \text{ such that } B = AT \]
\end{lemma}

\begin{proof}
  "$\Leftarrow$" If $B = AT$ with $T \in \gl_n(\Z)$ then $T \Z^n = \Z^n$.
  Hence, $B \Z^n = A \Z^n$.\\
  "$\Rightarrow$" If $A\Z^n = B \Z^n$ then each column vector of $B$ lies in $A \Z^n$,
  thus $\exists T \in \mat_n(\Z)$ such that $B = AT$.
  Similarly $\exists T^\prime \in \mat_n(\Z)$ such that $A = B T^\prime$.
  Hence, $ A = A T T^\prime$.
  Thus $T^\prime = T^{-1}$. So $T \in \gl_n(\Z)$.
\end{proof}

By Lemma \ref{l_2_1_1} we see that if $\Lambda = A \Z^n$ then $\abs{\det A}$ is uniquely determined by $\Lambda$.
We call it the \underline{determinant} of $\Lambda$
\[ \det \Lambda = \abs{\det A} \]
Let $b_1,\dots,b_n$ be a basis of $\Lambda$ and $v \in \Lambda$. We set
\[ F_v = [0,1)\cdot b_1 + \dots + [0,1) \cdot b_n + v \]
and call it a fundamental cell of $\Lambda$.
\todo{add pic}

Note that
\begin{itemize}
  \item $\det \Lambda = \vol F$
  \item $\R^n = \overset{\cdot}{\cup}_{v \in \Lambda} F_v$ is a partition of $\R^n$ (cf sheet 5).
\end{itemize}

Recall that $C \subset \R^n$ is called \underline{convex} if:
\[ x,y \in C \implies tx + (1-t)y \in C \forall t \in [0,1] \]
We say $C$ is symmetric if:
\[ x \in C \implies -x \in C\]
Recall that every convex set is measurable.

Let $C$ be a convex, compact and symmetric set in $\R^n$ with the origin in the interior of $C$.
Let $\Lambda$ be a lattice in $\R^n$.
Then we define the successive minima $\lambda_1,\dots,\lambda_n$ of $\Lambda$ with respect to $C$ by
\[ \lambda_i = \inf \{ \lambda : \lambda C \cap \Lambda \text{ contains } i \text{ linearly independent vectors} \} \text{.} \]

Note that $0 < \lambda_1 \leq \lambda_2 \leq \dots \leq \lambda_n < \infty$.

\begin{ex}
  \begin{itemize}
    \item $\Lambda = \Z^n$, $C = [-1,1]^n$.
    Then $\lambda_1 = \dots = \lambda_n = 1$.
    \item $\Lambda = \left(
    \begin{matrix}
      1 & 0 \\
      0 & 2
    \end{matrix}
    \right) \Z^2$, $C = [-1,1]^2$.
    Then $\lambda_1 = 1$, $\lambda_2 = 2$.
    \item $\Lambda = \left(
    \begin{matrix}
      1 & 0 \\
      0 & 2
    \end{matrix}
    \right) \Z^2$, $C = [-1,1] \times [-2,2]$.
    Then $\lambda_1 = \lambda_2 = 1$.
  \end{itemize}
  \todo{add pic}
\end{ex}

\begin{theorem}\label{th_2_1_2}
  Let $\Lambda$ be in $\R^n$. Then $\Lambda$ is a lattice in $\R^n$ if and only if:
  \begin{enumerate}
    \item[i)\label{th_2_1_2_1}] $(\Lambda,+)$ is a group
    \item[ii)\label{th_2_1_2_2}] $\Lambda$ contains $n$ linearly independent vectors
    \item[iii)\label{th_2_1_2_3}] $\Lambda$ is discrete, $\# S \cap \Lambda < \infty \forall \text{ compact } S \subset \R^n$.
  \end{enumerate}
\end{theorem}

\begin{proof}
  First suppose $\Lambda$ is a lattice.
  Then i) and ii) are clear.
  And iii) is clear, at least if $\Lambda = \Z^n$.
  But if $\Lambda = A \Z^n$ then $\# \Lambda \cap S = \# \Z^n \cap A^{-1} S$ as $x \mapsto A^{-1} x$ is continuous we have $S$ compact $\implies A^{1-} S$ compact.
  So this proves the first direction.\\
  Now let's suppose $\Lambda \subset \R^n$ such that i), ii), iii) hold.
  We use induction on $n$. \\
  Let $n=1$. Then $\Lambda$ contains a non-zero vector $b$ that is closest to the origin (using ii) and iii)).
  By i) we easily see that $\Lambda = b \Z$. So $\Lambda $ is a lattice in $\R^1$.\\
  Now suppose the claim holds in $\R^m$ if $m < n$.
  Let $u_1,\dots,u_n$ be $n$ linearly independent vectors in $\Lambda$.
  Consider the subspace $U = \left\langle u_1,\dots,u_{n-1} \right\rangle_\R$; thus $\dim U = n-1$.
  Let $\tilde{e_1},\dots,\tilde{e_{n-1}}$ be an orthonormal basis of $U$.
  Let
  \[ O \in \oth_n(\R) = \{ A \in \gl_n(\R): A^T A = I_n \} \text{ (the orthogonal group)} \]
  with
  \[ O(\tilde{e_i}) = e_i \; (1 \leq i \leq n-1) \]
  where $e_1,\dots,e_n$ is the canonical basis of $\R^n$.
  Hence, $O(U) = \R^{n-1} \times \{ 0 \}$, and
  \[ O( \Lambda \cap U) \subset \R^{n-1} \times \{ 0 \} \]
  is a discrete additive group that contains the $n-1$ linearly independent vectors $O(u_1),\dots,O(u_{n-1})$.
  Let $\Pi: \R^n \to \R^{n-1}$, $\Pi(x) = (x_1,\dots,x_{n-1})$.
  Then $\Pi \circ O(\Lambda \cap U)$ is also a discrete additive group that contains $n-1$ linearly independent vectors
  and it is also in $\R^{n-1}$.
  Hence, by induction hypothesis $\Pi \circ O(\Lambda \cap U)$ is a lattice in $\R^{n-1}$.
  So $\exists A_{n-1} \subset \gl_{n-1}(\R)$ such that $\Pi \circ O (\Lambda \cap U) = A_{n-1} \Z^{n-1}$.
  So 
  \[ O (\Lambda \cap U) = \left(
  \underbrace{\begin{matrix}
    A_{n-1} & 0 \\
    0 & 0
  \end{matrix}}{\tilde{A}}
  \right) \Z^n \]
  Now let $\mu = \inf \{ \abs{ w_n} : w = (w_1,\dots,w_n) \in O(\Lambda \setminus U) \}$.
  Suppose $v_1,v_2,v_3,\dots \in O(\Lambda \setminus U)$ with
  \[ \abs{ v_{in}} \to \mu \; (\text{as } i \to \infty \]
  where $v_{in}$ is the last coordinate.
  
  Adding elements from $O(U)$ does not change the last coordinate.
  Hence, we can assume that the vectors 
  \[ (v_{i1},\dots,v_{in-1}) \in [0,1)a_1 + \dots + [0,1) a_{n-1} \]
  where $a_i =$ column vector of $A_{n-1}$.
  In particular, the first $n-1$ coordinates of $v_i$ are  bounded in absolute value.
  But the absolute value of the last coordinate also tends to $\mu$, so is also bounded.
  As $\Lambda$ is discrete by iii) also $O(\Lambda)$ is discrete.
  Thus the sequence $v_i$ contains only finitely many vectors, in particular
  \[ \exists v \in O(\Lambda \setminus U) \text{ such that } v_n \overset{after v \to -v}{=} \mu \text{ and } \mu > 0 \text{.}\]
  Let $u \in O(\Lambda)$.
  Then also $u^\prime = u - \left[ \frac{u_n}{\mu} \right] \cdot v$ is in $O(\Lambda)$ ($O(\Lambda)$ is a group as $\Lambda$ is).
  So $0 \leq u_n^\prime < \mu$, so by minimality of $\mu$, $u_n^\prime = 0$.
  Hence, $u^\prime \in O(\Lambda) \cap \R^{n-1} \times \{ 0 \} = O(\Lambda) \cap O(U) = O(\Lambda \cap U)$.
  Now 
  \[ u = u^\prime + \left[ \frac{u_n}{\mu} \right] \cdot v \in O(\Lambda \cap U) + \Z \cdot v\]
  and thus
  \[ u \in \tilde{A} \Z^n + v \cdot \Z = \underbrace{\left[ (\tilde{a_1}) \dots (\tilde{a_{n-1}}) (v)  \right]} \Z^n = A \Z^n \]
  where $\tilde{a_i} = i$-th column vector of $\tilde{A}$.
  Thus $O(\Lambda) \subset A \Z^n$.
  Clearly (as $O(\Lambda)$ is a group) also $A \Z^n \subset O(\Lambda)$.
  Now the rows of $A$ are linearly independent.
  Thus $A \in \gl_n(\R)$.
  So $O(\Lambda)$ is a lattice and thus $\Lambda$ is a lattice.
\end{proof}

\begin{cor}[2.1.3 \label{cor_2_1_3}]
  Let $n \in \N$, $m_1,\dots,m_n \in \N$ and $a_{ij} \in \Z$ $(1\leq i,j \leq n)$.
  Then 
  \[ \Lambda = \{ x \in \Z^n : \sum_{j=1}^n a_{ij} x_j \equiv 0 \bmod m_i (1 \leq i \leq n) \} \]
  is a lattice in $\R^n$.
\end{cor}

\begin{proof}
  As $\Lambda \subset \Z^n$, it is discrete. Clearly $0 \in \Lambda$, and if $x_1,x_2 \in \Lambda$ then $x_1 + x_2 \in \Lambda$.
  So $\Lambda$ is a discrete additive subgroup of $\Z^n$.
  Moreover, the $n$ linearly independent vectors ($m,0,\dots,0),\dots,(0,\dots,0,m)$ where $m = m_1 \cdot \dots \cdot m_n$
  are all in $\Lambda$.
  Hence, by Theorem \ref{th_2_1_2} we conclude that $\Lambda$ is a lattice in $\R^n$.
\end{proof}


\subsection{The Theorems of Blichfeldt and Minkowski}

Minkowski's First and Second Theorem are possibly some of the most useful theorems in number theory.
We will deduce Minkowski's First Theorem via Blichfeldt's Theorem which is of interest for its own sake.

\begin{theorem}[Blichfeldt \label{th_2_2_1_blichfeldt}]
  Let $\Lambda$ be a lattice in $\R^n$, and let $S \subset \R^n$ be measurable such that $\vol S > \det \Lambda$ ($\vol S = \infty$ is allowed).
  Then 
  \[ \exists x_1,x_2 \in S, x_1 \neq x_2 \text{ and } x_1 - x_2 \in \Lambda \text{.}\]
\end{theorem}

\begin{proof}
  Let $b_1,\dots,b_n$ be a basis of $\Lambda$ and let $F = [0,1) b_1 + \dots + [0,1) b_n$ be a fundamental cell.
  Thus $\vol F = \det \Lambda$, and if $x \in \R^n$ then $\exists$ \underline{unique} $v \in \Lambda$ and $\theta \in F$ such that
  \[ x = v + \theta \text{.}\]
  Now for each $v \in \Lambda$ consider
  \[ \mathcal{R}(v) = \{ \theta \in F : v + \theta \in S \} \]
  \todo{add picture}
  Hence, 
  \[ \sum_{v \in \Lambda} \vol(\mathcal{R}(v)) = \vol S \]
  Now if $\vol S > \det \Lambda$ then (2.1) implies $\sum_{v \in \Lambda} \vol (\mathcal{R}(v)) > \det \Lambda = \vol F  $.
  But $\cup_{v \in \Lambda} \mathcal{R}(v) \subset F$; so the union \underline{cannot} be disjoint.
  Thus $\exists v_1,v_2 \in \Lambda, v_1 \neq v_2$ such that $\theta_0 \in \mathcal{R}(v_1) \cap \mathcal{R}(v_2)$.
  Hence, the points $x_1 = v_1 + \theta_0, x_2 = v_2 + \theta_0$ are both in $S$ and $x_1 - x_2 = v_1 - v_2 \in \Lambda \setminus 0$.
\end{proof}

\begin{theorem}[Minkowski's First Theorem \label{th_2_2_2_minkowski1}]
  Let $\Lambda$ be a lattice in $\R^n$, and let $S \subset \R^n$ be convex and symmetric.
  Suppose that either
  \begin{itemize}
    \item $\vol S > 2^n \det \Lambda$ ($\vol S = \infty$ allowed) \\
    or
    \item $\vol S \geq 2^n \det \Lambda$ and $S$ is compact
  \end{itemize}
  Then $S$ contains a \underline{non-zero} lattice point.
\end{theorem}

\begin{rem}
  $2^n$ is sharp, take $\Lambda = \Z^n$ and $S = (-1,1)^n$,
  then $\vol S = 2^n$, $\det \Lambda = 1$, $S$ is symmetric and convex, but $S \cap \Lambda = \{ 0\}$.
\end{rem}

\begin{proof}
  First suppose $\vol S > 2^n \cdot \det \Lambda$.
  Now $\vol \left( \frac{1}{2}S \right) = 2^{-n} \vol S > \det \Lambda$.
  By Theorem \ref{th_2_2_1_blichfeldt} (with the set $\frac{1}{2} S$) we see that $\exists x_1,x_2 \in S$,
  $x_1 \neq x_2$ such that $\frac{1}{2} x_1 - \frac{1}{2} x_2 \in \Lambda \setminus 0$.
  But $S$ is symmetric, thus $-x_2 \in S$.
  As $S$ is convex we conclude that $\frac{1}{2} x_1 + \frac{1}{2} (-x_2) \in S$.
  This proves the first part.\\
  Now suppose $S$ is compact and $\vol S = 2^n \det \Lambda$.
  If $v \in \Lambda \setminus S$ then $\exists \varepsilon_v > 0$ such that $B_{\varepsilon_v}(v) \cap S = \emptyset$ ($S^c$ is open!)
  where $B_r(y) = \{ x \in \R^n : \abs{x - y} < r \}$.
  As $S$ is compact $\exists R > 0$ such that $\lambda S \subset B_R (0)$ for all $\lambda$ with $0 < \lambda <2$.
  So $(\Lambda \setminus S) \cap B_R(0)$ is finite by Theorem \ref{th_2_1_2} and hence $\exists \varepsilon > 0$
  such that 
  \[ B_\varepsilon(v) \cap S = \emptyset  \; \forall v \in (\Lambda \setminus S) \cap B_R(0) \text{.} \]
  \todo{add picture}
  Hence, $\exists \lambda > 1$ such that
  \[ \lambda S \cap \Lambda = S \cap \Lambda \text{.} \]
  By the first part we know that $\lambda S$ contains a non-zero lattice point, and this completes the proof.
\end{proof}

\begin{cor}\label{cor_2_2_3}
  Let $\Lambda$ be a lattice in $\R^n$ and let $a_{ij} \in \R \; (1 \leq i,j \leq n)$.
  Suppose $c_1,\dots, c_n > 0$ and $c_1 \dots c_n \geq \abs{\det A} \det \Lambda$.
  Then $\exists u \in \Lambda \setminus 0$ such that 
  (2.2) $\begin{cases}  
    \abs{ \sum_{j=1}^n a_{1j} u_j} \leq c_1\\
    \abs{ \sum_{j=1}^n a_{ij} u_j} < c_i \: (2 \leq i \leq n) \text{.}
  \end{cases}$
\end{cor}

\begin{proof}
  First suppose $\det A \neq 0$.
  Then $\mathcal{L} = A \cdot \Lambda$ is  a lattice in $\R^n$ with $\det \mathcal{L} = \abs{\det A} \det \Lambda$.
  Then (2.2) means we are looking for a non-zero lattice point $x \in \mathcal{L}$ such that
  \[ \abs{x_1} \leq c_1 \]
  \[ \abs{x_i} < c_i \: (2 \leq i \leq n) \]
  These inequalities define a symmetric, convex set of points $x \in \R^n$
  with volume $2^n c_1 \cdots c_n$. So if $c_1 \cdots c_n > \abs{\det(A)} \cdot \det(\Lambda)$ then we can apply Theorem \ref{th_2_2_2_minkowski1} and the claim follows at once.\\
  Next let $0 < \varepsilon < 1$.
  Then the set
  \begin{align*}
  S_\varepsilon:
  \begin{cases} \abs{x_1} \leq c_1 + \varepsilon < c_1 + 1 & \\
  \abs{x_i} < c_i & \text{for } (2 \leq i \leq n)
  \end{cases}
  \end{align*}
  still has a non-zero lattice point in $\mathcal{L}$.
  But these sets $S_\varepsilon$ all lie in $S_1$ which lies in a compact set and hence has only finitely many lattice points.
  Hence, there must be a non-zero lattice point of $\mathcal{L}$ in $S_0$. 
  This proves the Corollary if $\det(A) \neq 0$.\\
  Now if $\det(A) = 0$ then (2.2) defines a set of points $u \in \R^n$ of infinite volume and so Theorem \ref{th_2_2_2_minkowski1} applies again and yields the claim.
\end{proof}

\begin{cor}[Lagrange's four-square Theorem\label{cor_2_2_4_lagrange}]
  Every positive integer is the sum of four squares.
\end{cor}

\begin{proof}
  First we observe that 
  \begin{align*}
    (x_1^2 + x_2^2 &+ x_3^2 + x_4^2) \cdot (y_1^2 + y_2^2 + y_3^2 + y_4^2) = \\
    &= (x_1 y_1 + x_2 y_2 + x_3 y_3 + x_4 y_4)^2 + (-x_1 y_2 + x_2 y_1 -x_3 y_4 + x_4y_3)^2 \\
    &+ (-x_1 y_3 + x_2 y_4 + x_3 y_1 - x_4 y_2)^2 + (-x_1 y_4 - x_2 y_3 + x_3y_2 + x_4 y_1)^2 \text{.}
  \end{align*}
  Now $1 = 1^2 + 0^2 + 0^2 + 0^2$. So it suffices to prove the claim for primes $p$.
  And we can assume $p \neq 2$ since $2 = 1^2 + 1^2 +0^2 + 0^2$. \\
  Now $a^2$ and $-(b^2 + 1)$ run through exactly $\frac{p+1}{2}$ distinct residue classes modulo $p$ as $a \bmod b$ run through an entire system of residue classes. ($0^2, 1^2, \dots, \left( \frac{p-1}{2} \right)^2$ are all distinct in $\mathcal{F}_p$.)\\
  Hence, they have a common residue class; thus 
  \[ \exists a,b \in \Z \text{ such that } a^2 + b^2 +1 \equiv 0 \bmod p \text{.} \]
  With this choice of $a$ and $b$ consider
  \[ \Lambda = A \cdot \Z^4 \text{ where } A= \left( 
  \begin{matrix}
    1 & 0 & 0 & 0\\
    0 & 1 & 0 & 0\\
    a & b & p &  0\\
    b & -a & 0 & p
  \end{matrix} \right) \text{.}\]
  So $\Lambda$ is a lattice in $\R^4$ with $\det(\Lambda) = p^2$.\\
  Next consider the convex, symmetric set 
  \[ S = \{ (x_1,x_2,x_3,x_4) \in \R^4 : x_1^2 + x_2^2 + x_3^2 + x_4^2 < 2p \} \]
  Then $\vol(S) = \frac{\pi^2}{2} (2p)^4 = (2 \pi)^2 p^4 > 16 \cdot p^2 = 2^4 \det(\Lambda)$.
  By Theorem \ref{th_2_2_2_minkowski1} there exists an $x \in \Lambda \cap S$ with $x \neq 0$.\\
  Now 
  \[ x = \left(
  \begin{matrix}
    x_1 \\
    x_2 \\
    x_3 \\
    x_4
  \end{matrix} \right)
  = A \cdot z = \left(
  \begin{matrix}
    z_1 \\
    z_2 \\
    a z_1 + b z_2 + p z_3 \\
    b z_1 - a z_2 + p z_4
  \end{matrix} \right) \text{ for some } z \in \Z^4 \setminus \{0\} \]
  Hence,
  \begin{align*}
    x_1^2 + x_2^2 + x_3^2 + x_4^2 &\equiv z_1^2 + z_2^2 + (a z_1 + b z_2)^2 + (b z_1 - a z_2)^2 \\
    &\equiv \underbrace{(1 + a^2 + b^2)}_{\equiv 0 \bmod p} (z_1^2 + z_2^2) \\
    &\equiv 0 \bmod p
  \end{align*}
  Since $x \in S$, and $x \neq 0$ we conclude that $x_1^2 + x_2^2 + x_3^2 + x_4^2 = p$.
  This proves the corollary.
\end{proof}

\subsection{Basis reduction}

Let $M$ be a lattice in $\R^n$ and let $\Lambda$ be a \underline{sublattice} of $M$, i.e., $\Lambda \subset M$ and $\Lambda$ is a lattice in $\R^n$.
Hence, there exists a matrix $C \in \mat_n(\Z)$ with $\det C \neq 0$ such that $\Lambda = C \cdot M$.\\
We define the \underline{index} of $\Lambda$ in $M$ by 
\[ I = \abs{\det(C)} = \frac{\det(\Lambda)}{\det(M)}\text{.} \]
Note that $I \cdot C^{-1} = \adj(C) \in \mat_n(\Z)$,
and thus $I \cdot M = I \cdot C^{-1} \cdot \Lambda \subset \Lambda$, so
\begin{align}
  I \cdot M \cdot C \subset \Lambda \subset M \label{3_1}
\end{align}

\begin{theorem}\label{th_2_3_1}
  Let $\Lambda$ be a sublattice of the lattice $M \subset \R^n$, 
  and let $b_1,\dots,b_n$ be a basis of $M$.
  Then there exists a basis $a_1,\dots,a_n$ of $\Lambda$ with
  \begin{align} \label{3_2}
    \begin{split}
    a_1 &= v_{11} b_1\\
    a_2 &= v_{21} b_1 + v_{22} b_2\\
    &\vdots\\
    a_n &= v_n1 b_1 + \dots + v_{nn} b_n
    \end{split}
  \end{align}
  with $v_{ij} \in \Z$ and $v_{ii} \neq 0$ for $(1 \leq j \leq i \leq n)$.\\
  Conversely, if $a_1,\dots, a_n$ is a basis of $\Lambda$ then there exists a basis $b_1,\dots, b_n$ of $M$ such that (\ref{3_2}) holds.
\end{theorem}

\begin{proof}
  By (\ref{3_1}) we know that there exist $v_{ij} \in \Z$ with $v_{ii} \neq 0$ and $\abs{v_{ii}}$ minimal such that
  \[ a_i = v_{i1} b_1 + \dots + v_{ii} b_i \in \Lambda \]
  We will show that $a_1,\dots,a_n$ is a basis for $\Lambda$.\\
  Let $c \in \Lambda$ and suppose $c$ is \underline{not} a $\Z$-linear combination of the $a_i$'s.
  As $c \in M$ there exist $t_i \in \Z$ such that
  \[ c = t_1 b_1 + \dots + t_k b_k \; (1 \leq k \leq n, \, t_k \neq 0)\]
  If there exist several such $c$'s then we choose one where $k$ is minimal.
  Next we note that $v_{kk} \neq 0$.
  Hence, there exists a $s \in \Z$ such that
  \begin{align}
    \abs{t_k - s v_{kk}} < \abs{v_{kk}} \text{.} \label{3_3}
  \end{align}
  Thus
  \[ c - sa_k = (t_1 - sv_{k1}) b_1 + \dots + (t_k - sv_{kk}) b_k \]
  lies in $\Lambda$ (as $c$ and $a_k$ do) and is \underline{not} a $\Z$-linear combination of the $a_i$'s as $c$ is not.
  Thus, by minimality of $k$ we must have $t_k - sv_{kk} \neq 0$.
  But then (\ref{3_3}) contradicts the minimality of $\abs{v_{kk}}$.
  Hence, $c$ must be a $\Z$-linear combination of the $a_i$'s and thus $a_1,\dots, a_n$ is a basis.
  This proves the first part.\\
  For the second part suppose $a_1,\dots,a_n$ is a basis of $\Lambda$.
  By the first part and (\ref{3_1}) there exists a basis $I \cdot b_1,\dots, Ib_n$ of $I \cdot M \subset \Lambda$ with
  \begin{align*}
    I b_1 &= w_{11} a_1\\
    I b_2 &= w_{21} a_1 + w_{22} a_2\\
    &\vdots\\
    I b_n &= w_{n1} a_1 + \dots + w_{nn} a_n
  \end{align*}
  with $w_{ij} \in \Z$ and $w_{ii} \neq 0$.\\
  Successively solving the above system for $a_i$ we get a system as in (\ref{3_2}) but a priori with $v_{ij} \in \Q$.
  But $b_1,\dots,b_n$ is a basis of $M$ and the $a_i \in M$.
  As the representation
  \[ a = t_1 b_1 + \dots + t_n b_n \; (t_i \in \R) \]
  is unique we conclude that $v_{ij} \in \Z$, and this proves the second part.
\end{proof}

\begin{lemma}[Hadamard's inequality\label{l_2_3_2_hadamard}]
  Let $a_1,\dots,a_n \in \R^n$. Then
  \[ \abs{\det ( a_1,\dots, a_n)} \leq \norm{a_1} \cdots \norm{a_n} \text{.} \]
\end{lemma}

\begin{proof}
  This is geometrically obvious as the volume of a parallelepiped is not larger than product of the lengths of the spanning vectors.
  However, here is a formal proof. \\
  If $a_1,\dots,a_n$ are linearly dependent then the inequality is trivial.
  Now assume $a_1,\dots,a_n$ are linearly independent.
  Put
  \[ c_i = a_i - \sum_{j< i} \left\langle a_i, c_j \right\rangle \norm{c_j}^{-2} \cdot c_j\]
  Then
  \begin{align}
    \left\langle c_i, c_j \right\rangle = 0 \; (i \neq j)  \label{3_4}
  \end{align}
  and 
  \begin{align}
    a_i = t_{i1} c_1 + \dots + t_{i (i-1)} c_{i-1} + c_i \label{3_5}
  \end{align}
  By (\ref{3_4}) and (\ref{3_5}) we get 
  \[ \norm{a_i}^2 = \left\langle a_i, a_i \right\rangle = (\sum_{j=1}^{i-1} t_{ij}^2 \norm{c_j}^2) + \norm{c_i}^2 \geq \norm{c_i}^2 \]
  and $\det (a_1,\dots,a_n) = \det(c_1,\dots,c_n)$ (by linearity of determinant in columns).\\
  Moreover,
  \begin{align*}
    (\det (c_1,\dots,c_n))^2 = \det \left(    
    \begin{bmatrix}
      c_1\\
      \vdots\\
      c_n
    \end{bmatrix}
    \begin{bmatrix}
      c_1 & \cdots & c_n
    \end{bmatrix}
    \right)
    = \prod_{i=1}^n \norm{c_i}^2 \leq \prod_{i=1}^n \norm{a_i}^2 \text{.}
  \end{align*}
\end{proof}

\begin{defi*}
  A \underline{distance function} $f$ on $\R^n$ is a function $f: \R^n \to \R$ such that
  \begin{itemize}
    \item $f(x) \geq 0 \; \forall x \in \R^n$
    \item $f(tx) = \abs{t} f(x) \; \forall x \in \R^n \; \forall t \in \R$
    \item $f$ is continuous.
  \end{itemize}
\end{defi*}

\begin{defi*}
  We say $C$ is a star body in $\R^n$ if
  \begin{itemize}
    \item $C \subset \R^n$ compact
    \item $0 \in \inner(C)$, i.e., origin lies in the interior of $C$
    \item $x \in C \implies t \cdot x \in C \; (0 \leq t \leq 1)$
  \end{itemize}
\end{defi*}

%TODO add an example - picture

\begin{rem}
  \begin{itemize}
    \item For a star body $C$ we have $t \cdot C \subset C$ for $0 \leq t \leq 1$.
    \item Every compact, convex $C \subset \R^n$ with the origin in its interior is a star body in $\R^n$.
    \item To every symmetric star body $C$ in $\R^n$ we can associate a distance function $f_C$ defined by
    \[ f_C(x) = \inf \{ \lambda: x \in \lambda \cdot C \} \text{.} \]
    Note that $f_C(x) = 0 \implies x =0$. \\
    Why? If $x \neq 0$, then there exists a $\lambda > 0$ such that $\lambda x \nin C$.
    Hence, $f_C(x) \geq \frac{1}{\lambda}$.
    \item If $C$ is symmetric and convex then $f_C$ is actually a norm on $\R^n$ (cf exercise sheet 5).
    In particular, $f_C$ satisfies the triangle-inequality.
  \end{itemize}
\end{rem}

\begin{lemma}\label{l_2_3_3}
  Let $C$ be a convex, symmetric star body in $\R^n$, and let $\Lambda$ be a lattice in $\R^n$ with successive minima $\lambda_1,\dots,\lambda_n$ with respect to $C$.
  Then there exist linearly independent $a_1,\dots,a_n \in \Lambda$ with $f_C(a_i) =  \lambda_i$.\\
  Moreover, if $a \in \Lambda$ and $f_C(a) < \lambda_j$ then $a_1,\dots,a_{j-1},a$ are linearly dependent.
\end{lemma}

\begin{proof}
  The set $(\lambda_n +1) \cdot C$ is compact and by definition of $\lambda_n$ contains $n$ linearly independent lattice points.
  By the definition of the $\Lambda_i$'s it suffices to consider these points.
  But by Theorem \ref{th_2_1_2} there are only finitely many of these, and so the claim easily follows.
\end{proof}

\begin{cor}\label{cor_2_3_4}
  Let $C$ be a convex, symmetric star body in $\R^n$, and let $\Lambda$ be a lattice in $\R^n$ with successive minima $\lambda_1,\dots,\lambda_n$ with respect to $C$.
  Then there exists a basis $b_1,\dots,b_n$ of $\Lambda$ such that for $j = 1,\dots,n$: 
  \[x \in \Lambda \text{ and } f_C(x) < \lambda_j \implies x = u_1 b_1 + \dots + u_{j-1} b_{j-1} \] 
  for some $u_1,\dots,u_{j-1} \in \Z$.
\end{cor}

\begin{proof}
  Let $a_1,\dots,a_n \in \Lambda$ be as in Lemma \ref{l_2_3_3}.
  Let $\Lambda^\prime = (a_1,\dots,a_n) \Z^n$ be the sublattice of $\Lambda$ with basis $a_1,\dots,a_n$.
  By Theorem \ref{th_2_3_1} there exists a basis $b_1,\dots, b_n$ of $\Lambda$ with (\ref{3_2});
  so $a_j$ is dependent only on $b_1,\dots,b_j$.
  By Lemma \ref{l_2_3_3} if $f_C(x) < \lambda_j$, then 
  \begin{align*}
    x &= s_1 a_1 + \dots + s_{j-1} a_{j-1} \\
    &= u_1 b_1 + \dots + u_{j-1} b_{j-1}
  \end{align*}
  with $u_i \in \Q$.\\
  As $x \in \Lambda$ and $b_1,\dots, b_{j-1}$ are linearly independent we conclude that $u_1,\dots,u_{j-1} \in \Z$.
\end{proof}

\begin{ex}[Exercise]
  Let $C = B_1(0)$ and $\Lambda = \begin{pmatrix}
  2 & 0 & 0 & 0 & 1\\
  0 & 2 & 0 & 0 & 1\\
  0 & 0 & 2 & 0 & 1\\
  0 & 0 & 0 & 2 & 1\\
  0 & 0 & 0 & 0 & 1
  \end{pmatrix} \Z^5$.\\
  There exists \underline{no} basis $b_1,\dots, b_5$ such that
  $\norm{b_i} = \lambda_i \; (1\leq i\leq 5)$ with $\lambda_i = \lambda_i (\Lambda,C)$.
\end{ex}

\begin{lemma}\label{l_2_3_5}
  Let $C$ be a convex, symmetric star body in $\R^n$, let $\Lambda$ be a lattice in $\R^n$, and let $a_1,\dots,a_n$ be linearly independent vectors in $\Lambda$.
  Then there exists a basis $b_1,\dots,b_n$ of $\Lambda$ such that
  \[ f_C(b_j) \leq \max \left\lbrace f_C(a_j), \frac{1}{2} (f_C(a_1)+\dots+f_C(a_j)) \right\rbrace \]
\end{lemma}

\begin{proof}
  Consider the sublattice $\Lambda^\prime = (a_1 \dots a_n) \Z^n \subset \Lambda$.
  By Theorem \ref{th_2_3_1} there exists a basis $c_1,\dots,c_n$ of $\Lambda$ such that
  \begin{align} \label{3_6}
    \begin{split}
      a_1 &= v_{11} c_1\\
      a_2 &= v_{21} c_1 + v_{22} c_2\\
      &\vdots\\
      a_n &= v_{n1} c_1 + \dots + v_{nn} c_n
    \end{split}
  \end{align}
  with $v_{ij} \in \Z$ and $v_{ii} \neq 0$.\\
  Consider 
  \begin{align}
    b_j = c_j + t_{j(j-1)}a_{j-1} + \dots + t_{j1} a_1 \label{3_7}
  \end{align}
  where $t_{ji} \in \R$.\\
  If $b_1,\dots,b_n$ are in $\Lambda$ then by (\ref{3_6}) they form a basis of $\Lambda$.\\
  How do we choose $t_{ji}$?
  If $v_{jj} = \pm 1$ then we put $b_j = \pm a_j$, which clearly is in the required form (\ref{3_7}) and obviously 
  \[ f_C(b_j) = f_C(a_j) \text{.} \]
  Now suppose $\abs{v_{jj}} \geq 2$.
  Now solving (\ref{3_6}) for $c_j$ yields
  \[ c_j = v_{jj}^{-1} a_j + k_{j j-1} a_{j-1} + \dots + k_{j1} a_1 \]
  with $k_{ji} \in \Q$.\\
  We choose $t_{ji} \in \Z$ such that
  \[ \abs{k_{ji} + t_{ji}} \leq \frac{1}{2} \text{.} \]
  Then $b_j \in \Lambda$ and 
  \[b_j = l_{jj} a_j + l_{j j-1} a_{j-1} + \dots + l_{j1} a_1 \]
  with 
  \begin{align*}
    \abs{l_{jj}} &= \abs{v_{jj}^{-1}} \leq \frac{1}{2} \text{ and}\\
    \abs{l_{ji}} &= \abs{t_{ji} + k_{ji}} \leq \frac{1}{2} \text{ for } i < j
  \end{align*}
  Using that $C$ is a \underline{convex}, \underline{symmetric} star body we have the triangle inequality. 
  Hence,
  \begin{align*}
    f_C(b_j) &\leq f_C(l_{jj}a_j) + \dots + f_C(l_{j1}a_1) \\
    &= \abs{l_{jj}} f_C(a_j) + \dots + \abs{l_{j1}} f_C(a_1) \\
    &\leq \frac{1}{2} (f_C(a_j) + \dots + f_C(a_1)) \text{.} 
  \end{align*}
\end{proof}

\begin{cor}\label{2_3_6}
  Let $C$ be a convex, symmetric star body, and let $\Lambda$ be a lattice in $\R^n$ with successive minima $\lambda_1,\dots,\lambda_n$ with respect to $C$.
  Then there exists a basis $b_1,\dots,b_n$ of $\Lambda$ with
  \[ f_C(b_j) \leq \max \left\lbrace \lambda_j, \frac{1}{2}(\lambda_1 + \dots + \lambda_j) \right\rbrace \]
\end{cor}

\begin{proof}
  Immediate form Lemmas \ref{l_2_3_3} and \ref{l_2_3_5}.
\end{proof}


\subsection{Minkowski's Second Theorem}

Minkowski's Second Theorem is a refinement of his First Theorem and a central result in Geometry of Numbers.
Let's start by rephrasing Minkowski's First Theorem.%\roman{1}

First note if $C \subset \R^n$ is convex, symmetric and of positive volume then there exist $\varepsilon > 0$, $x \in C$ such that
\[ B_\varepsilon(x) \subset C \text{.} \]
But then there exists $\varepsilon^\prime > 0$ such that $B_{\varepsilon^\prime}(0) \subset C$.
So the origin lies in the interior of $C$.
So we can consider the successive minima $\lambda_1,\dots,\lambda_n$ of $\Lambda$ with respect to $C$, where $\Lambda$ is a lattice in $\R^n$.

Note that by definition of $\lambda_1$:
\[ \forall \varepsilon > 0: \; (\lambda_1 - \varepsilon) C \text{ contains \underline{no} non-zero lattice point.} \]

Minkowski's First Theorem yields:
\begin{align}
  \lambda_1^n \cdot \vol(C) = \vol(\lambda_1 C) \leq 2^n \cdot \det \Lambda \label{4_1}
\end{align}
On the other hand (\ref{4_1}) and $\vol C > 2^n \det \Lambda$ implies $\lambda_1 < 1$ and hence $C$ contains a non-zero lattice point.
The following theorem is much more precise than (\ref{4_1})!

\begin{theorem}[Minkowski's Second Theorem\label{th_2_4_1_minkovski2}]
  Let $C$ be a convex, symmetric star body in $\R^n$, and let $\Lambda$ be a lattice in $\R^n$ with successive minima $\lambda_1,\dots,\lambda_n$ with respect to $C$. Then
  \[ \frac{2^n}{n!} \det \Lambda \leq \lambda_1 \dots \lambda_n \cdot \vol C \leq 2^n \det \Lambda \]
\end{theorem}

\begin{rem}
  \begin{itemize}
    \item Both bounds are sharp. For the upper bound take $\Lambda = \Z^n$ and $C = [-1,1]^n$; 
    so $\lambda_1 = \dots = \lambda_n = 1 = \det \Lambda$, and $\vol C = 2^n$.
    For the lower bound take $\Lambda = \Z^n$ and $C$ defined by $\abs{x_1} + \dots + \abs{x_n} \leq 1$.
    Then $\vol C = \frac{2^n}{n!}$ and $\lambda_1 =  \dots = \lambda_n = 1 = \det \Lambda$.
    \item The upper bound is much harder to prove.
    We will prove Theorem \ref{th_2_4_1_minkovski2} only for the ball $C = B_1(0)$.
  \end{itemize}
\end{rem}

\begin{proof}
  (Special case $C = B_1(0)$)\\
  Put 
  \[ \delta(C) = \sup_M \frac{\lambda_1^n(M,C)}{\det M}\] 
  where the supremum runs over all lattices $M$ in $\R^n$.
  By Minkowski's First Theorem \ref{th_2_2_2_minkowski1} and (\ref{4_1}), we have 
  \[ \delta(C) \leq \frac{2^n}{\vol C} \text{.} \]
  We will show that if $C = B_1(0)$ then
  \begin{align}
    \det(\Lambda) \leq \lambda_1 \dots \lambda_n \leq \delta(C) \det(\Lambda) \text{.} \label{4_2}
  \end{align}
  In particular, as $\vol(C) \geq \frac{2^n}{n!}$, we get 
  \[ \frac{2^n}{n!} \lambda_1 \dots \lambda_n \leq \vol(C) \lambda_1 \dots \lambda_n \leq 2^n \det(\Lambda) \text{.} \]
  For the lower bound in (\ref{4_2}) take linearly independent $a_1,\dots,a_n \in \Lambda$ with $\norm{a_i} = \lambda_i$ using Lemma \ref{l_2_3_3}. 
  For the sublattice $\Lambda^\prime = (a_1 \dots a_n) \Z^n \subset \Lambda$ we have 
  \[ \det(\Lambda^\prime) = I \det(\Lambda) \text{,} \]
  where $I$ is the index of $\Lambda^\prime$ in $\Lambda$.
  By Hadamard's inequality we have
  \[ \norm{a_1} \cdot \dots \cdot \norm{a_n} \geq \det(\Lambda^\prime) \geq \det(\Lambda) \text{.} \]
  Thus
  \[ \det(\Lambda) \leq \lambda_1 \dots \lambda_n \text{.} \]
  Now we prove the upper bound in (\ref{4_2}).
  Let $b_1,\dots,b_n$ be a basis of $\Lambda$ as in Corollary \ref{cor_2_3_4}.
  As in the prove of Lemma \ref{l_2_3_2_hadamard} we can find mutually orthogonal vectors $c_1,\dots,c_n$ such that
  \[ b_j = t_{j1} c_1 + \dots + t_{jj} c_j \; (t_{ji} \in \R) \]
  By scaling we can assume $\norm{c_j}^2 = 1$ $(1 \leq j \leq n)$.
  Now
  \[ \sum_{j=1}^n u_j b_j = \sum_{i=1}^n \left( \sum_{j \geq i} u_j t_{ji} \right) c_i \]
  thus
  \begin{align}
    \norm{\sum_{j=1}^n u_j b_j}^2 = \sum_{i=1}^n \left( \sum_{j \geq i} u_j t_{ji} \right)^2 \label{4_3}
  \end{align}
  Next we show that
  \begin{align}
    \sum_{i=1}^n \lambda_i^{-2} \left( \sum_{j \geq i} u_j t_{ji} \right)^2 \geq 1 \label{4_4}
  \end{align}
  where $u = (u_1,\dots,u_n) \in \Z^n \setminus 0$.\\
  Let $u \in \Z^n \setminus 0$ with
  \begin{align}
    u_J \neq 0 \text{ and } u_j = 0 \text{ for } j > J \text{.} \label{4_5}
  \end{align}
  Then $u_1 b_1 + \dots + u_n b_n$, $b_1,\dots, b_{J-1}$ are linearly independent and by Corollary \ref{cor_2_3_4} we have 
  \begin{align}
    \norm{\sum_{j=1}^n u_j b_j}^2 \geq \lambda_J^2 \label{4_6}
  \end{align}
  Moreover, (\ref{4_5}) implies that summands with $j > J$ in (\ref{4_3}) and (\ref{4_4}) are zero.
  Thus, the left hand side in (\ref{4_4}) is equal to 
  \begin{align}
    \sum_{i \leq J} \lambda_i^{-2} \left( \sum_{i \geq j} u_j t_{ji} \right)^2 \geq \sum_{i \leq J} \lambda_J^{-2} \left( \sum_{i \geq j} u_j t_{ji} \right)^2 = \lambda_J^{-2} \norm{\sum_{j=1}^n u_j b_j}^2 \underbrace{\geq}_{\text{by (\ref{4_6}) and (\ref{4_3})}} 1
  \end{align}
  So if $\Lambda^\prime$ is the lattice with basis 
  \begin{align}
    b_j^\prime = t_{j1} \lambda_1^{-1} c_1 + \dots + t_{jj} \lambda_j^{-1} c_j \; (1 \leq j \leq n)
  \end{align}
  Then
  \[ \norm{\sum_{j=1}^n u_j b_j^\prime} \geq 1\]
  for every point $\sum_{j=1}^n u_j b_j^\prime \in \Lambda^\prime \setminus 0$.
  Hence, 
  \begin{align}
    \lambda_1(\Lambda^\prime,C) \geq 1 \label{4_7}
  \end{align}
  But 
  \begin{align}
    \det(\Lambda^\prime) = \lambda_1^{-1} \cdots \lambda_n^{-1} \det(\Lambda) \label{4_8}
  \end{align}
  $\lambda_i = \lambda_i(\Lambda,C)$\\
  Moreover, by definition
  \begin{align}
    \frac{\lambda_1^n(\Lambda^\prime,C)}{\det \Lambda^\prime} \leq \sup_M \frac{\lambda_1^n(M,C)}{\det M} = \delta(C) \label{4_9}
  \end{align}
  Combining (\ref{4_7}), (\ref{4_8}) and (\ref{4_9}) we conclude
  \begin{align}
    \lambda_1 \cdots \lambda_n = \det(\Lambda) \frac{1}{\det(\Lambda^\prime)} \leq \det(\Lambda) \frac{\lambda_1^n(\Lambda^\prime,C)}{\det(\Lambda^\prime)} \leq \det(\Lambda) \cdot \delta(C)
  \end{align}
\end{proof}

\subsection{Counting lattice points}

How many integer pairs $(x,y)$ solve the Diophantine inequality
\[ x^2 + y^2 \leq T \text{?} \]
What about
\[ 3 x^2 + 5 y^2 + 7 z^2 \leq T \text{,} \]
or more generally
\[F(\underline{x}) \leq T \text{,} \]
where $F$ is a positive definite quadratic form in $n$ variables?
Even more generally, let $S \subset \R^n$ and $\Lambda$ a lattice in $\R^n$, we would like to get a (non-trivial) estimate for $(\Lambda \cap S)$.\\

Suppose $S$ is measurable and "nicely shaped", and let 
\[ F_v = [0,1) b_1 + \dots + [0,1) b_n + v \; (v \in \Lambda) \] 
be a fundamental cell (with respect to basis $b_1,\dots,b_n$).\\

The idea is as follows:
$\abs{\Lambda \cap S} \approx $ number of $F_v$'s that lie in $S \approx \frac{\vol(S)}{\vol(F_v)} = \frac{\vol(S)}{\det(\Lambda)}$

\todo{"Nice set"} %TODO picture
\todo{"Bad set"} %TODO picture

To characterize nice sets we use the following definition.

\begin{defi*}
  Let $n \geq 2$, $M$ be in $\N$ and $L \geq 0$ real.
  We say the set $Z$ lies in \underline{$\lip(n,M,L)$} if
  \begin{itemize}
    \item $Z \subset \R^n$
    \item there exist $M$ maps $\phi_i: [0,1]^{n-1} \to \R^n$ %TODO mising text
  \end{itemize}
  satisfying a Lipschitz condition with constant $L$, i.e.,
  \[ \norm{\phi_i(x) - \phi_i(y)} \leq L \norm{x-y} \; \forall x,y \in [0,1]^{n-1} \]
  and such that the union of their images covers $Z$, i.e.,
  \[ Z \subset \cup_{i=1}^M \phi_i([0,1]^{n-1}) \text{.} \]
\end{defi*}

\begin{ex}
  The sphere $S^\prime \subset \R^2$ lies in $\lip(2,1,2\pi)$.
  \[ \phi(x) = (\cos(2\pi x), \sin(2 \pi x)) \text{ for } 0 \leq x \leq 1 \]
\end{ex}

We can now state the main result of Section 5.
Recall that the boundary $\partial S$ of $S \subset \R^n$ is defined by the topological closure $\bar{S}$ minus the interior $\inner(S)$ of $S$.
\[ \partial S = \bar{S} \setminus \inner(S)\]
We follow an approach of Masser and Vaaler.

\begin{theorem}\label{2_5_1}
  Let $S \subset \R^n$ be bounded and suppose that $\partial S \in \lip(n,M,L)$.
  Let $\Lambda$ be a lattice in $\R^n$ and $\lambda_1$ its first successive minimum with respect to the unit ball.
  Then, $S$ is measurable and
  \[ \abs{\abs{\Lambda \cap S} - \frac{\vol(S)}{\det \Lambda}} \leq c \cdot M \cdot \left( \frac{L}{\lambda_1} + 1 \right)^{n-1} \]
  where $c$ is a constant depending only on $n$.
\end{theorem}

For the proof we need the following lemma.

\begin{lemma}\label{2_5_2}
  Let $S \subset \R^n$ be bounded and measurable and let $\Lambda$ be a lattice in $\R^n$.
  Let $b_1,\dots,b_n$ be a basis of $\Lambda$, $F_v = [0,1)b_1 + \dots + [0,1)b_n + v$ the corresponding fundamental cells and write
  \[ \mathcal{T} = \abs{\{v \in \Lambda: F_v \cap \partial S \neq \emptyset\} } \text{,} \]
  the number of cells that intersect the boundary $\partial S$ of $S$.
  Then 
  \[ \abs{\abs{\Lambda \cap S} - \frac{\vol S}{\det \Lambda}} \leq \mathcal{T} \]
\end{lemma}

\begin{proof}
  The cells $F_v$ ($v \in \Lambda$) define a partition of $\R^n$.
  Every $F_v$ contains exactly one lattice point, namely $v$.
  Let $m = \abs{\{v \in \Lambda : F_v \subset S \}}$.
  Then $m \leq \abs{S \cap \Lambda}$.\\
  Now suppose $v \in S$.
  Then either $F_v \subset S$ or $F_v$ contains a point from $S$ and from its complement $S^C$.
  The connecting line segment of these points lies in $F_v$ as $F_v$ is convex and it must contain a point of the boundary $\partial S$.
  Hence, 
  \[ \abs{\Lambda \cap S} \leq m + \mathcal{T} \text{.} \]
  Now $\vol F_v = \det \Lambda$, and the union of all cells $F_v$ that lie in $S$ has volume $m \cdot \det \Lambda$.
  The volume of the union of cells $F_v$ that have non-empty intersection with $S$ is at most $(m + \mathcal{T}) \det \Lambda$.
  So we have proved the inequalities
  \begin{itemize}
    \item $m \leq \abs{\Lambda \cap S} \leq (m + \mathcal{T})$
    \item $m \cdot \det \Lambda \leq \vol S \leq (m + \mathcal{T}) \det \Lambda$
  \end{itemize}
  Thus \[ \abs{\abs{\Lambda \cap S} - \frac{\vol S}{\det \Lambda}} \leq \mathcal{T}\]
\end{proof}

We can now prove Theorem \ref{2_5_1}.

\begin{proof}[Proof of Theorem \ref{2_5_1}]
  We use $c_1,c_2,c_3,c_4,c_5,c_6,c_7$ to denote constants that depend only on $n$.\\
  First let's assume $\Lambda = \Z^n$, so $\lambda_1 = 1$.
  We take the standard basis $b_i = c_i \; (1\leq i\leq n)$ and apply Lemma \ref{2_5_2}; so $\mathcal{T} = \mathcal{T}(e_1,\dots,e_n)$.
  We split $[0,1]^{n-1}$ into $L_1^{n-1}$ subcubes of side length $\frac{1}{L_1}$ where $L_1 = [L] + 1$. %TODO replace [] by \floor  
  The images of these subcubes under the parametrizing maps $\phi$ have diameter at most $L \cdot \frac{c_1}{L_1} \leq c_1$.
  Thus, no more than $c_2 = (c_1 + 2)^n$ cells $F_v$ can meet such a single image (= the image of a single subcube).
  Hence, 
  \begin{align}
    \mathcal{T} \leq M \cdot c_2 \cdot L_1^{n-1} \leq M \cdot c_2 \cdot (L + 1)^{n-1} \label{5_1}
  \end{align}
  As $\lambda_1(\Z^n) = 1$ this proves the claim for $\Lambda = \Z^n$ thanks to Lemma \ref{2_5_1}.
  
  Now let $\Lambda$ be an arbitrary lattice.
  By Corollary \ref{2_3_6} there exists a basis $b_1,\dots,b_n$ of $\Lambda$ such that
  \[ \abs{b_i} \leq c_3 \lambda_i \; (1 \leq i \leq n) \]
  where $\lambda_1,\dots,\lambda_n$ are the successive minima with respect to the unit ball.
  Let 
  \[ A^{-1} = \left( b_1 \dots b_n \right) \in \gl_n(\R) \text{,} \]
  so that 
  \[ A(\Lambda) = \Z^n \text{.} \]
  Now 
  \[ \abs{S \cap \Lambda} = \abs{A(S \cap \Lambda)} = \abs{A(S) \cap A(\Lambda)} = \abs{A(S) \cap \Z^n} \text{.} \]
  So we can apply the case $\Lambda = \Z^n$ to the set $A(S)$.
  The boundary $\partial A(S)$ can  be parametrized by the $M$ maps $\psi(x) = A(\phi(x))$ which satisfy
  \[ \norm{\psi(x) - \psi(y)} \leq \norm{A} \cdot L \norm{x-y} \text{,}\]
  where $\norm{A}$ denotes the Euclidean operator norm of $A$.\\
  By Cramer's rule the entries of row $i$ of $A$ are of the form $\frac{\mu}{\det \Lambda}$, 
  where $\mu$ is a minor of the matrix with columns $b_1,\dots,b_n$ omitting $b_i$.
  Using Hadamard's inequality we conclude 
  \[ \abs{\mu} \leq \frac{\norm{b_1} \cdots \norm{b_n}}{\norm{b_i}} \leq c_4 \cdot \frac{\lambda_1 \cdots \lambda_n}{\lambda_i} \text{.}\]
  By Minkowski's Second Theorem we have
  \[ \lambda_1,\dots,\lambda_n \leq c_5 \cdot \det \Lambda \text{.} \]
  Hence, each entry of $A$ has absolute value at most $\frac{c_6}{\lambda_1}$.
  It follows 
  \[ \norm{A} \leq \frac{c_7}{\lambda_1} \text{.} \]
  Replacing $L$ in (\ref{5_1}) by $\frac{c_7 L}{\lambda_1}$ proves the theorem.
\end{proof}

\begin{rem}
  We have not shown that $S$ is measurable.
  One could do that by showing that $\partial S$ has measure zero, and noting that every closed set is measurable.\\
  Why is $\vol(\partial S) = 0$?
  Take $\Lambda = k^{-1} \Z^n$ where $k \in \N$, and $\mathcal{T}$ be associated to the basis $k^{-1} e_1, \dots, k^{-1}e_n$.
  The proof of Theorem \ref{2_5_1} yields
  \[ \mathcal{T} \leq c_n M \left( \frac{L}{k^{-1}} + 1 \right)^{n-1} \leq c_n M (L+1)^{n-1} k^{n-1} \text{.} \]
  Since 
  \[\vol(\partial S) \leq \mathcal{T} \vol F_v \leq c_n M (L+1)^{n-1} k^{n-1} k^{-n} \to 0 \text{ as } k \to \infty \text{.} \]
  Hence $\vol(\partial S) = 0$.
\end{rem}


In some applications a more precise error term is needed that involves also the higher successive minima.
With a bit more effort the following result could be proved.

\begin{theorem}\label{2_5_3}
  Same hyptothesis as in Theorem \ref{2_5_1} and $\lambda_1,\dots,\lambda_n$ successive minima with respect to the unit ball.
  Then 
  \[ \abs{\abs{\Lambda \cap S} - \frac{\vol S}{\det \Lambda}} \leq c M \underbrace{\max_{0 \leq i < n} \frac{L}{\lambda_1 \cdot \lambda_i}}_{\coloneq 1 \text{ for } i = 0} \]
  \todo{Professor: compare with Adrian}
  and $c = n^{3n^2}$.
\end{theorem}

\end{document}
