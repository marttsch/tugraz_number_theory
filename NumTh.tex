\documentclass[a4paper]{article}

\usepackage[utf8]{inputenc}
\usepackage[english]{babel}
\usepackage{amsmath}
\usepackage{amsfonts}
\usepackage{amssymb}
\usepackage{hyperref}
\usepackage{amsthm}
\usepackage{stmaryrd}
\usepackage{algpseudocode}
\usepackage{algorithm}
\usepackage{MnSymbol}
\usepackage{xcolor}
\usepackage[nameinlink]{cleveref}
\usepackage{mathtools}
\usepackage{subfiles}
\usepackage{accents}
\usepackage{color}
\usepackage{titlesec}

\usepackage[colorinlistoftodos,prependcaption,textsize=footnotesize]{todonotes}

\newcommand{\sectionbreak}{\clearpage}

\hypersetup{
  pdfborder={0 0 0},
  colorlinks=true,
  linkcolor=darkgray,
  anchorcolor=darkgray
}

\newcommand{\lecture}{\vspace{5mm}\textcolor{blue}}

\setcounter{section}{-2}

\newcounter{cnt} \numberwithin{cnt}{subsection}

\newtheorem{defi}[cnt]{Definition}
\newtheorem*{defi*}{Definition}
\newtheorem*{rem}{Remark}
\newtheorem*{rem*}{Remark}
\newtheorem{lemma}[cnt]{Lemma}
\newtheorem*{lemma*}{Lemma}
\newtheorem*{ex}{Example}
\newtheorem*{ex*}{Example}
\newtheorem{theorem}[cnt]{Theorem}
\newtheorem*{theorem*}{Theorem}
\newtheorem{prop}[cnt]{Proposition}
\newtheorem*{prop*}{Proposition}
\newtheorem{cor}[cnt]{Corollary}
\newtheorem*{cor*}{Corollary}
\newtheorem*{app}{Application}
\newtheorem*{app*}{Application}
\newtheorem{conj}[cnt]{Conjecture}
\newtheorem{op}[cnt]{Open Problem}


\crefname{rem}{Remark}{remarks}

\DeclareMathOperator{\mat}{Mat}
\DeclareMathOperator{\gl}{GL}
\DeclareMathOperator{\adj}{adj}
\DeclareMathOperator{\vol}{Vol}
\DeclareMathOperator{\oth}{O}
\DeclareMathOperator{\lip}{Lip}
\DeclareMathOperator{\inner}{Int}
\DeclareMathOperator{\ord}{ord}

\newcommand{\N}{\mathbb{N}}
\newcommand{\Z}{\mathbb{Z}}
\newcommand{\PP}{\mathbb{P}}
\newcommand{\C}{\mathbb{C}}
\newcommand{\Q}{\mathbb{Q}}
\newcommand{\R}{\mathbb{R}}

\newcommand{\p}{\mathbb{P}} %projection

\newcommand{\overbar}[1]{\mkern 1.5mu\overline{\mkern-1.5mu#1\mkern-1.5mu}\mkern 1.5mu}
\newcommand{\inv}[1]{{#1}^{-1}}
\newcommand{\card}[1]{\left\vert{#1}\right\vert}
\newcommand{\set}[1]{\left\{#1\right\}}

\newcommand{\Vek}[3]{\left(\begin{array}{r}#1\\#2 
\ifthenelse{\equal{#3}{}}{}{\\#3}\end{array}\right)}

\newcommand{\ubar}[1]{\underaccent{\bar}{#1}}

% use \abs{} for absolute values
\DeclarePairedDelimiter\abs{\lvert}{\rvert}%
\DeclarePairedDelimiter\norm{\lVert}{\rVert}%
\makeatletter
\let\oldabs\abs
\def\abs{\@ifstar{\oldabs}{\oldabs*}}
\let\oldnorm\norm
\def\norm{\@ifstar{\oldnorm}{\oldnorm*}}
\makeatother

\renewcommand{\algorithmicrequire}{\textbf{Given:}}
\renewcommand{\algorithmicensure}{\textbf{Find:}}

\author{Martina Tscheckl}
\title{Number Theory}

\begin{document}
\maketitle
Please send feedback to \url{martina@tscheckl.eu}.
\tableofcontents

\section{Organizatorial information}

Dates (in TUGrazOnline):
\begin{table}[!h]
  \begin{tabular}{cccl}
    Mon & 14:15--15:45 & C208 & Exercises (starting 19.10. first exercise class) \\
    Tue & 14:15--15:45 & C307 & Lecture (starting 20.10. first (real) lecture) \\
    Wed & 08:15--09:45 & C208 & Lecture
  \end{tabular}
\end{table}

From now until 15.12. lectures by Martin Widmer.
Then C. Frei.

End: oral exams

Exercises: Find details on website of the instructor Dijana Kreso.
\url{math.tugraz.at/~kreso}

\subsection{Oral exam information}
\begin{itemize}
  \item all parts of the lecture will be covered
  \item roughly 20-30 min 
  \item know at least the ideas behind the long proofs
  \item to get a date write an e-mail
  \item before the end of March take the exam by Frei, then Tichy
\end{itemize}

\subfile{chapters/0_basics.tex}
\subfile{chapters/1_diophantine_approximation.tex}
\subfile{chapters/2_geometry_of_numbers.tex}
\subfile{chapters/3_algebraic_number_theory.tex}
\subfile{chapters/4_analytic_number_theory.tex}

\section{Literature}

\subsection{Chapter1: Diophantine Approximation}
\begin{enumerate}
  \item Approximation to algebraic numbers; Wolfgang M. Schmidt, 1972 L'Ehseignement Mathématique
  \item Lectures Notes in Mathematics 785; W.M.Schmidt, Springer
  \item LNM 1467, W.M.S., Springer
  \item For section 2 (continued fractions) he will strictly follow the lecture notes of MT421 of Professor James McKee
\end{enumerate}

\subsection{Chapter2: Geometry of Numbers}
\begin{enumerate}
  \item J.W.S. Cassels "An Introduction to the Geometry of Numbers"
  \item W.M. Schmidt Lecture Notes M. 785 and 1467
\end{enumerate}

\subsection{Chapter3: Algebraic Number Theory}
\begin{enumerate}
  \item Daniel Marcus "Number fields" Springer
  \item Course Notes "Algebraic Number Theory" Math 2803(?) by Matt Baker (available online on his webpage)
  \item Serge Lang: Algebraic Number Theory Addison \& Wesley
\end{enumerate}



%\subfile{lectures/NT_2015-10-05.tex}
%\subfile{lectures/NT_2015-10-06.tex}

%\subfile{lectures/NT_2015-10-20.tex}
%\subfile{lectures/NT_2015-10-21.tex}

%\subfile{lectures/NT_2015-10-27.tex}
%\subfile{lectures/NT_2015-10-28.tex}

%\subfile{lectures/NT_2015-11-03.tex}
%\subfile{lectures/NT_2015-11-04.tex}

%\subfile{lectures/NT_2015-11-10.tex}
%\subfile{lectures/NT_2015-11-11.tex}

%\subfile{lectures/NT_2015-11-17.tex}
%\subfile{lectures/NT_2015-11-18.tex}

%\subfile{lectures/NT_2015-11-24.tex}
%\subfile{lectures/NT_2015-11-25.tex}

%\subfile{lectures/NT_2015-12-01.tex}
%\subfile{lectures/NT_2015-12-02.tex}

%\subfile{lectures/NT_2015-12-09.tex}

%\subfile{lectures/NT_2015-12-15.tex}
%\subfile{lectures/NT_2015-12-16.tex}

%\subfile{lectures/NT_2016-01-12.tex}
%\subfile{lectures/NT_2016-01-13.tex}

%\subfile{lectures/NT_2016-01-19.tex}
%\subfile{lectures/NT_2016-01-20.tex}

%\subfile{lectures/NT_2016-01-26.tex}

\end{document}
