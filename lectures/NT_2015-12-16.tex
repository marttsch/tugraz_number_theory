\documentclass[NumTh.tex]{subfiles}
\begin{document}

%renewcommand \hom form hom to Hom

Additional references: J. Neukirch, "Algebraic number Theory", Springer \\
\\
We will not prove Theorem \ref{th_3_3_6}.\\
\\

\begin{ex}
  Consider $K = \Q[\sqrt{-5}]$ then $O_K = \Z[\sqrt{-5}]$ (see sheet 6).\\
  Consider the following ideals
  \begin{align*}
    \mathfrak{p}_1 &= (2, 1 + \sqrt{-5}) \text{,} \\
    \mathfrak{p}_2 &= (2, 1 - \sqrt{-5})
  \end{align*}
  generated as $O_K$-modules by $2, 1 \pm \sqrt{-5}$. Then 
  \begin{align*}
    \mathfrak{p}_1 \mathfrak{p}_2 &= (4, 2 (1 - \sqrt{-5}), 2 (1+ \sqrt{-5}), 6) \\
    &= (2) \underbrace{(2, 1 + \sqrt{-5}, 1 - \sqrt{-5}, 3)}_{ = (1)} = (2)
  \end{align*}
  With 
  \begin{align*}
  \mathfrak{p}_3 &= (3, 1 + \sqrt{-5}) \text{,} \\
  \mathfrak{p}_4 &= (3, 1 - \sqrt{-5})
  \end{align*}
  we find 
  \[ p_3 p_4 = (3) \text{.} \]
  Then the non-unique factorization into irreducable elements
  \[ 6 = 2 \cdot 3 = (1 + \sqrt{-5}) (1 - \sqrt{-5}) \]
  in $O_K$ becomes the unique prime factorization into ideals in $O_K$
  \begin{align*}
    (6) = (2) \cdot (3) &= (\mathfrak{p}_1 \mathfrak{p}_2) (\mathfrak{p}_3 \mathfrak{p}_4) \\
    &= (\mathfrak{p}_1 \mathfrak{p}_3) (\mathfrak{p}_2 \mathfrak{p}_4) \\
    &= (1 + \sqrt{-5}) (1 - \sqrt{-5}) \text{.}
  \end{align*}
\end{ex}

\begin{cor}\label{cor_3_3_7}
  If $O_K$ is a principal ideal domain (PID) then $O_K$ is a unique factorization domain (UFD).
\end{cor}

\begin{proof}
  Exercise (on your own)
\end{proof}

\begin{ex}
  With $K = \Q(\sqrt{-1})$ or $\Q(\sqrt{-2})$ then $O_K = \Z[\sqrt{-1}]$ or $\Z[\sqrt{-2}]$ respectively (see sheet 6).
  We know that the above rings are Euclidean and hence PIDs, thus UFD.
\end{ex}


\subsection{The ideal class group}

Throughout this subsection $K$ denotes a number field.\\
\\
A \emph{fractional ideal} $I$ is an additive subgroup of $K$ such that there exists $a \in O_K$, $a \neq 0$ with
\[ aI = \{ a \cdot r : r \in I \} \]
is an ideal in $O_K$.\\
Note that the product of two fractional ideals $I$, $J$
\[ I \cdot J = \{ x \cdot y : x \in I, y \in J \} \]
is again a fractional ideal.
For an ideal $J \neq 0$ in $O_K$ we denote
\[ J^{-1} = \{ x \in K : x \cdot J \subset O_K \} \text{.} \]
As $a \cdot J^{-1} \subset O_K$ for any $a \in J$ we easily see that $J^{-1}$ is a fractional ideal of $O_K$.

\begin{lemma}["to divide is to contain"\label{l_3_4_1}]
  Let $\mathfrak{a}$, $\mathfrak{b}$ be ideals in $O_K$.
  Then 
  \[ \mathfrak{a} \divides \mathfrak{b} \iff \mathfrak{b} \subset \mathfrak{a} \text{.} \]
\end{lemma}

\begin{proof}
  If $\mathfrak{b} \subset \mathfrak{a}$ then $\mathfrak{c} \coloneq \mathfrak{b} \cdot \mathfrak{a}^{-1} \subset \mathfrak{a} \mathfrak{a}^{-1} = O_K$.
  Thus $\mathfrak{c}$ is an ideal in $O_K$ and $\mathfrak{b} = \mathfrak{c} \mathfrak{a}$.
  Conversely if $\mathfrak{b} = \mathfrak{a} \cdot \mathfrak{c}$ with $\mathfrak{c}$ in $O_K$ then $\mathfrak{b} = \mathfrak{a} \cdot \mathfrak{c} \subset \mathfrak{a}$.
\end{proof}

\begin{lemma}\label{l_3_4_2}
  The set $I_K$ of non-zero fractional ideal of $O_K$ forms a group under multiplication.
\end{lemma}

\begin{proof}
  It suffices to check that we have inverses.\\
  Let $J \in I_K$.
  Then there exists an $a \in O_K$, $a \neq 0$ such that 
  \[ I \coloneq a J \subset O_K \text{.} \]
  Then also
  \[ a \cdot I^{-1} \in I_K \text{.} \]
  Moreover,
  \[ J \cdot a I^{-1} = I \cdot I^{-1} = O_K \text{.} \]
\end{proof}

A fractional ideal $I$ is called \emph{principal} if there exists an $x \in K$ such that 
\[ I = (x) = \{ x \cdot r: r \in O_K \} \text{.} \]
Write $P_K$ for the subset of $I_K$ of non-zero principal fractional ideals.
$P_K$ is a subgroup of $I_K$.

The \emph{ideal class group} $CL_K$ is defined as the quotient group
\[ CL_K = I_K/P_K \text{.} \]
We have the following exact sequence
\[ 1 \to O_K^\ast \to K^\ast \to I_K \to CL_K \to 1 \]
(all maps are homomorphisms).\\
The map $I_K \to CL_K$ is clearly surjective.
The expansion when passing from numbers (in $K^\ast$) to ideals (in $I_K$) is measured by the class group ($CL_K$)
and $O_K^\ast$ measures the contraction in the same process.

\begin{theorem}\label{th_3_4_3}
  $CL_K$ is finite.
\end{theorem}

Let $\hom_\Q(K)$ be the set of $\Q$-homomorphisms from $K$ into $\C$.
If $K = \Q[\alpha]$ and $\sigma \in \hom_\Q(K)$ then
\[ 0 = \sigma(f_\alpha(\alpha)) = f_\alpha(\sigma ( \alpha)) \]
so $\sigma(\alpha)$ is a root of $f_\alpha$.
Denote these roots by $\alpha_1,\dots,\alpha_d$ so $d = [K:\Q]$.\\
Indeed each $\sigma(\alpha) = \alpha_i$ extends to a $\Q$-homomorphism of $K$.
After relabelling let 
\[ \alpha_1,\dots,\alpha_r \] 
be the real and
\[ \alpha_{r+1},\alpha_{r+1+s},\dots,\alpha_{r+s},\alpha_{r + 2s} \] 
be the $s$ pairs of complex conjugate roots of $f_\alpha$.\\
Then
\[ \sigma_1,\dots,\sigma_r \]
are the real embeddings and
\[ \sigma_{r+1},\sigma_{r+1+s},\dots,\sigma_{r+s},\sigma_{r+2s} \]
are the $s$ pairs of complex conjugate embeddings.\\
We consider the \emph{Minkowski-embedding}:
\begin{align*}
  \sigma: K &\to \R^r \times \C^s \\
  \alpha &\mapsto (\sigma_1(\alpha),\dots,\sigma_r(\alpha),\sigma_{r+1}(\alpha),\dots,\sigma_{r+s}(\alpha))
\end{align*}
\\

Let $\mathfrak{a} \neq (0)$ be an ideal in $O_K$ and let $N(\mathfrak{a}) = [O_K:\mathfrak{a}]$ be the group index.
We call $N(\mathfrak{a})$ the \emph{norm} of $\mathfrak{a}$.
\\

We make use of the following lemma which we won't prove.

\begin{lemma}[3.4.4\label{l_3_3_4}]
  \begin{itemize}
    \item $N(\mathfrak{a})$ is finite for all $\mathfrak{a} \neq (0)$ ideals in $O_K$
    \item $N(\mathfrak{a} \cdot \mathfrak{b}) = N(\mathfrak{a}) \cdot N(\mathfrak{b})$ for $\mathfrak{a}, \mathfrak{b} \neq (0)$ ideals in $O_K$
    \item If $\alpha \in O_K$, $\alpha \neq 0$ $N((\alpha)) = \prod_{\sigma \in \hom_\Q(K)} \abs{\sigma(\alpha)}$
  \end{itemize}
  Moreover, if $(0) \neq \mathfrak{a}$ is an ideal in $O_K$ then $\sigma \mathfrak{a}$ is a lattice in 
  \[ \R^r \times \C^s \simeq \R^{r + 2s} = R^d \] 
  $(d = [K:\Q])$ with 
  \todo{$\frac{1}{2}$ or $-\frac{1}{2}$?} \[ \det(\sigma \mathfrak{a}) = 2^{-s} N(\mathfrak{a}) \cdot \abs{\Delta_K}^{-\frac{1}{2}} \] 
  where $\Delta_K \in \Z \setminus 0$ is a certain invariant of $K$ called the discriminant of $K$.\\
  Any $I \in I_K$ has the form $I = \mathfrak{a} \mathfrak{b}^{-1}$ with $\mathfrak{a}$ and $\mathfrak{b}$ ideals in $O_K$.
  By the multiplicity of the norm we can extend $N(\cdot)$ to $I_K$; 
  \[ N(I) = N(\mathfrak{a})/N(\mathfrak{b}) \text{.} \] 
\end{lemma}

\begin{lemma}\label{l_3_4_5}
  Let $\mathfrak{a} \neq (0)$ be an ideal in $O_K$. 
  Then there exists $0 \neq \alpha \in \mathfrak{a}$ such that
  \[ N((\alpha)) \leq \left( \frac{2}{\pi} \right)^s \cdot \sqrt{\abs{\Delta_K}} \cdot N(\mathfrak{a}) \text{.} \]
\end{lemma}

\begin{proof}
  Choose $c_i > 0$ ($1 \leq i \leq r+s$) with 
  \[ \prod_{i=1}^{r+s} c_i^{d_i} > \left( \frac{2}{\pi} \right)^s N(\mathfrak{a}) \sqrt{\abs{\Delta_K}} \text{,} \]
  where $d_i = 
  \begin{cases}
    1 &: 1 \leq i \leq r \\
    2 &: r+1 \leq i \leq r+s
  \end{cases} \text{.}$\\
  Consider
  \[ S = \{ x \in \R^r \times \C^s: \abs{x_i} < c_i (1 \leq i \leq r+s) \} \text{.} \]
  Now $S$ is convex, symmetric in $\R^r \times \C^s \simeq \R^d$ with 
  \begin{align*}
    \vol S &= (2 \cdot c_1) \dots (2 c_r) (\pi c_{r+1}^2) \dots (\pi c_{r+s}^2) \\
    &> 2^d \cdot \det \sigma(\mathfrak{a}) \text{.}
  \end{align*}
  By Minkowski's First Theorem there exists an $\alpha \in \mathfrak{a}$, $\alpha \neq 0$ such that $\sigma \alpha \in S$.
  Thus $\abs{\sigma_i \alpha} < c_i \; (1 \leq i \leq r+s)$,
  and hence 
  \begin{align*}
    N((\alpha)) &= \prod_{i=1}^{r+s} \abs{\sigma_i(\alpha)}^{d_i} \\
    &< \prod_{i=1}^{r+s} c_i^{d_i}.
  \end{align*}
  As $\prod_{i=1}^{r+s} c_i^{d_i}$ can be chosen arbitrarily close to $\left( \frac{2}{\pi} \right)^s N(\mathfrak{a}) \sqrt{\abs{\Delta_K}}$ the claim follows.
\end{proof}

\begin{lemma}\label{l_3_4_6}
  There are finitely many ideals in $O_K$ with bounded norm, i.e., 
  \[ \abs{\{ \mathfrak{a} \subset O_K : a \neq 0, N(\mathfrak{a}) < M \}} < \infty \; \forall M > 0 \text{.} \]
\end{lemma}

\begin{proof}
  Let $\mathfrak{p}$ be a prime ideal in $O_K$.
  Then 
  \[ \mathfrak{p} \cap \Z = p \Z \]
  with a prime number $p \in \Z$.\\
  By Lemma \ref{l_3_4_1} $\mathfrak{p} \divides (p)$, hence
  \[ N(\mathfrak{p}) \divides N((p)) = p^d \; (d = [K:\Q]) \text{.} \]
  As there are only finitely many prime ideals $\mathfrak{p}$ that divide $(p)$ we conclude that there are only finitely many prime ideals of bounded norm.
  This implies that there are only finitely many ideals in $O_K$ of bounded norm.
\end{proof}

\begin{proof}[Proof of Theorem \ref{th_3_4_3}]
  Let $c \in CL_K$ and let $I \subset O_K$ be an ideal in $c^{-1}$. 
  We write $[I] = c^{-1}$.
  By Lemma \ref{l_3_4_5} we can choose $\alpha \in I$, $\alpha \neq 0$ such that
  \[ N((\alpha)) \leq \left( \frac{2}{\pi} \right)^s \cdot \abs{\Delta_K}^{\frac{1}{2}} N(I) \text{.} \]
  By Lemma \ref{l_3_4_1} we have
  \[ (\alpha) \subset I \implies I \divides (\alpha) \] 
  so
  \[ (\alpha) = I \cdot J \] 
  with $J \subset O_K$.
  Now $( \alpha) \in P_K$.
  So 
  \[  [J] = [I]^{-1} = c \text{.} \]
  Now 
  \[ N(J) = \frac{N((\alpha))}{N(I)} \leq \left( \frac{2}{\pi} \right)^s \abs{\Delta_K}^{\frac{1}{2}} \text{.} \]
  Hence, any ideal class $c$ has an integral representative $J$ of bounded norm.
  But by Lemma \ref{l_3_4_6} there are only finitely many of these.
\end{proof}


\subsection{Dirichlet's Unit Theorem}

Using geometry of numbers for a "multiplicative version" of the Minkowski-embedding one can prove the following fundamental result.

\begin{theorem}[Dirichlet's Unit Theorem\label{th_3_5_1}]
  Let $K$ be a number field with $r$ real and $s$ pairs of complex conjugate embeddings.
  The group $O_K^\ast$ is the direct product of a finite cyclic group and of an abelian free group of rank $r+s-1$.\\
  So there exist 
  \[ \varepsilon_1,\dots,\varepsilon_{r+s-1} \text{ in } O_K^\ast \]
  such that 
  \[ \forall \varepsilon \in O_K^\ast \text{exists a unique root of unity } \xi \text{ and a vector } (a_1,\dots,a_{r+s-1}) \in \Z^{r+s-1} \]
  such that
  \[ \varepsilon = \xi \varepsilon_1^{a_1} \cdots \varepsilon_{r+s-1}^{a_{r+s-1}} \text{.} \]
\end{theorem}

\end{document}
