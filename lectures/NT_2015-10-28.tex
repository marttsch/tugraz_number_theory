\documentclass[NumTh.tex]{subfiles}
\begin{document}

\begin{proof}
  For $n = 0, 1$ we note that 
  \begin{align}
  q_0 &= q = u_1\\
  q_1 &= a = \frac{b}{c} = \frac{u_2}{c}\\
  p_0 &= b = \theta + \bar{\theta} = u_2\\
  p_1 &= a b + 1 = \frac{b^2+c}{c} = \frac{(\theta + \bar{\theta})^2 - \theta \bar{\theta}}{c} = \frac{u_3}{c}
 \end{align}
 Put $\omega_{n+2} = c^{-\lfloor\frac{n+1}{2} \rfloor} u_{n+2}$.\\
 So we need to show that $p_n = \omega_{n+2}$.\\
 Using that $\theta^{n+2} = b \theta^{n+1} + c \theta^n$ and $\bar{\theta}^{n+2} = b \bar{\theta}^{n+1} + c \bar{\theta}^n$
 and hence $u_{n+2} = \frac{\theta^{n+2} - \bar{\theta}^{n+2}}{\theta - \bar{\theta}} = b u_{n+1} + c u_n$.\\
 Moreover, $u_{2m+2} = c^m \omega{2m+2}$, $u_{2m+1} = c^m \omega_{2m+1}$.
 Inserting this into the above, distinguishing $n$ even or odd yields:
 \begin{align}
   \omega_{2m+2} = b \omega_{2m+1} + \omega_{2m}\\
   \omega_{2m+1} = a \omega_{2m} + \omega_{2m - 1}
 \end{align}
 Hence, $p_n$ and $\omega_{n+2}$ satisfy the same recurrence relation. and here the same two starting values, so $p_n = \omega_{n+2}$.\\
 Similar for $q_n$.
\end{proof}

\emph{Counting Diophantine Approximation 2:}\\
We can use Theorem 1.2.5 to show that if $\theta = [\bar{b,a}]$ with $b = a c, a>1$ then 
\[ N_\theta(\frac{1}{2x^2},Q) = \frac{\log Q}{\log(\frac{Q}{\sqrt{c}})} + \mathcal{O}(1)\]
Indeed, we have already seen, that 
\[ N_\theta(\frac{1}{2x^2},Q) = \#\{n: q_n \leq Q \} \]
By Theorem 1.2.5 we know
\[ q_n \leq Q \iff c^{-\lfloor\frac{n+1}{2} \rfloor} \frac{\theta^n - \bar{\theta}^n}{\theta - \bar{\theta}} = \left(\frac{\theta}{\sqrt{c}}\right)^n \left( 1 - \left(\frac{\bar{\theta}}{\theta}\right)^n\right) \epsilon \leq Q\]
where $ \epsilon = \begin{cases} \frac{1}{\theta - \bar{\theta}} & 2 \divides n\\ \frac{1}{\sqrt{c}(\theta -\bar{\theta}} & 2 \nmid n \end{cases}$
\[ \iff n \log\left(\frac{\theta}{\sqrt{c}}\right) + \log\left( 1 - \left(\frac{\bar{\theta}}{\theta}\right)^n\right) + \log \epsilon \leq \log Q \]
Using Taylor series expansion we see that 
\[ \abs{\log\left( 1 - \left(\frac{\bar{\theta}}{\theta} \right)^n\right)} \leq \abs{\frac{\bar{\theta}}{\theta - \bar{\theta}}}  \]
This proves the claim.

\subsection{Liouville's Theorem}
Let $\alpha \in \C$. If $\exists D(x) \in \Z[x]$, $D \neq 0$ and $D(\alpha) = 0$ then we say $\alpha$ is \emph{algebraic}.
In this case $\exists D(x) = a_0 x^d + \dots + a_d \in \Z[x]$ with
\begin{itemize}
  \item $D(\alpha) = 0$
  \item $a_0 > 0$
  \item $\gcd(a_0,\dots,a_d) = 1$
  \item $\deg D(x)$ minimal
\end{itemize}
Imposing all these condition renders $D$ unique; We write $D_\alpha(x)$ and call this the \emph{minimal polynomial} of $\alpha$.
If $\alpha$ is algebraic then we say $\deg D_\alpha$ is the \emph{degree of $\alpha$}.

\begin{ex}
  \begin{itemize}
    \item $\alpha = 0, D_\alpha(x) = x$
    \item $\alpha = \sqrt{2} + 1, D_\alpha(x) = (x -1)^2 - 2 = x^2 -2x -1$
    \item $\alpha = \frac {1}{\sqrt{2}}, D_\alpha(x) = 2x^2 -1$
  \end{itemize}
\end{ex}

\begin{theorem}[1.3.1 Liouville's Theorem]
  Suppose $\alpha$ is a real, algebraic number of degree $d$.
  Then $\exists c(\alpha) > 0$ such that
  \[ \abs{\alpha - \frac{p}{q}} > \frac{c(\alpha)}{q^d} \]
  for every $(p,q) \in \Z \times \N$ with $\alpha \neq \frac{p}{q}$.
\end{theorem}

\begin{proof}
  Suppose $\abs{\alpha - \frac{p}{q}} > 1$ then the claim holds for every $c(\alpha) > 1$.
  Now suppose $\abs{\alpha - \frac{p}{q}} \leq 1$. Taylor series expansion at $D_\alpha$ about $\alpha$ gives:
  \[ D_\alpha(x) = \sum_{i=1}^d (x - \alpha)^i \frac{1}{i!} D_\alpha^{(i)}(\alpha) \]
  Hence, 
  \[ \abs{D_\alpha\left(\frac{p}{q}\right)} = \abs{\sum_{i=1}^d \left(\frac{p}{q} - \alpha \right)^i \frac{1}{i!} D_\alpha^{(i)}(\alpha)} \overset{\leq}{(D)label} \abs{\frac{p}{q} - \alpha} \frac{1}{c(\alpha)} \]
  where
  \[ c(\alpha) = \left( 1 + \sum_{i=1}^d \frac{1}{i!} \abs{D_\alpha^{(i)}(\alpha)}\right)^{-1} \]
  Now if $D_\alpha$ has a rational root then it must have degree one, so have only \emph{one} root.
  Thus $D_\alpha \left(\frac{p}{q}\right) \neq 0$ unless $\alpha = \frac{p}{q}$.
  Hence, if $\alpha \neq \frac{p}{q}$ we get
  \[ \abs{D_\alpha \left(\frac{p}{q}\right)} = | \frac{\text{non-zero integer}}{q^d} \geq \frac{1}{q^d}. \] % TODO: there is no end for |
  Combing this with (D)label yields 
  \[ \abs{\alpha - \frac{p}{q}} > \frac{c(\alpha)}{q^d}.\]
\end{proof}

We say a real number $\alpha$ is a \emph{Liouville number} if for every $n \in \N$
\[ 0 < \abs{\alpha - \frac{p}{q}} < \frac{1}{q^n} \]
has a solution. $p,q \in \Z$ with $q > 1$.

\begin{ex}
  $\alpha = \sum_{k = 1}^\infty 10^{-k^k}$ is a Liouville number.
  Let $n \in\N$ and put $p = \sum_{k=1}^n 10^{n^n - k^k}$ and $q = 10^{n^n}$.
  Then $ 0 < \abs{\alpha - \frac{p}{q}} = \sum_{k>n} 10^{-k} \leq 2\cdot 10^{-(n+1)^{(n+1)}} < 10^{-n^{(n+1)}} = q^{-n}$
\end{ex}

\begin{cor}[1.3.2]
  Every Liouville number is transcendental (i.e., not algebraic).
\end{cor}

\begin{proof}
  Immediate from Theorem 1.3.1 (Liouville's Theorem).
\end{proof}

%\begin{rem}
  Algebraic numbers are enumerable and thus have Lebesgue measure zero.
  It's not difficult to show that the set of Liouville numbers, while \emph{not} enumerable, also has measure zero.
  In fact \grqq most\grqq ~ real numbers are "not very far" from badly approximable as the following theorem shows.
%\end{rem}

\begin{theorem}[Khintchine]
  Suppose $\psi: \N \to (0,\infty)$ is monotone decreasing (not necessarily strictly).
  The set 
  \[A_\psi = \{\alpha \in \R: \abs{\alpha - \frac{p}{q}} < \frac{\psi(q)}{q} \text{ has $\infty$-many solutions } (p,q) \in \Z \times \N \} \]
  has a Lebesgue measure zero of $\sum_{q=1}^\infty \psi(q)$ converges and has full Lebesgue measure (i.e. the complement has measure zero) 
  if $\sum_{q=1}^\infty \psi(q)$ diverges.
\end{theorem}

We will not prove this Theorem. (For a proof see e.g. Glyn Harman "Metric number theory".)

\begin{ex}
  \begin{itemize}
    \item Take $\psi(q) = \frac{1}{q}$. We already know that $A_\psi = \R \setminus \Q$.
    And indeed $\sum \psi(q)$ diverges...
    \item $\psi(q) = \frac{1}{q \log(q-1)}$. Then $\sum \psi(q)$ diverges and thus $A_\psi$ has full measure.
    \item $\psi(q) = \frac{1}{q (\log(q+1))^{1+\epsilon}} (\epsilon > 0)$ then $\sum \psi(q)$ converges,
    so $A_\psi$ has measure zero.
  \end{itemize}
\end{ex}

\end{document}
