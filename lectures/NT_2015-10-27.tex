\documentclass[NumTh.tex]{subfiles}
\begin{document}

\begin{rem}
  \begin{itemize}
    \item (d) implies that if $p,q \in \Z$, $0 < q \leq p_n$ then
    \begin{align*}
      |\theta - \frac{p}{q}| &\geq |\theta - \frac{p_n}{q_n}| \cdot \frac{p_n}{q}\\
      &\geq | \theta - \frac{p_n}{q_n}|
    \end{align*}
    with "$=$" only if $\frac{p}{q} = \frac{p_n}{q_n}$.
    \item We say $\alpha \in \R \setminus \Q$ is \emph{badly approximable} if
    \[ \exists c > 0 \text{ such that } |\alpha - \frac{p}{q}| > \frac{c}{q^2} \forall (p,q) \in \Z \times \N\]
    \item By (c) and (d) we see that $\theta = [a_0,a_1,a_2,\dots]$ is badly approximable if and only if
    the partial quotients $a_i$ are uniformly bounded, i.e., $\exists M > 0$ such that $a_i < M \forall i$.
    \item (c) suggests that the \grqq worst-approximable\grqq ~number is $\theta = [1,1,1,\dots]$. 
    That's indeed the case c.f Exercise sheet 2 \# 5,6 (using that $\theta = 1 + \frac{1}{1+\frac{1}{1+\dots}} = 1 + \frac{1}{\theta}$.
    So $\theta^2 - \theta -1 = 0$.\\
    So $\theta = \frac{1 \pm \sqrt{5}}{2}$ but $a_0 = 1$ so $\theta = \frac{1 + \sqrt{5}}{2}$.
  \end{itemize}
\end{rem}

%-----------
\emph{Counting Diophantine approximations 1:}\\
Let $\alpha \in \R \setminus \Q$ and let $\phi: [1,\infty) \to (0,\infty)$ be decreasing.
Consider the number of "$\phi$-good" approximations:
\[ N_\alpha(\phi,Q) = \#\{\frac{p}{q} \in \Q ; |\alpha - \frac{p}{q} | < \phi(q), 1 \leq q \leq Q\} \]
We put $S_\alpha(\phi,Q) = \{(x,y)\in \R^2 : | \alpha - \frac{x}{y}| < \phi(y), 1 \leq y \leq Q\}$.
Then
\[ N_\alpha(\phi,Q) = \#\{(p,q)\in \Z \times \N : \gcd(p,q)\} \cap S_\alpha(\phi,Q) \]
Note that by Corollary 1.1.2 we have $N_\alpha(\phi,Q) \to \infty$ as $Q \to \infty$ provided $\phi(y) \geq \frac{1}{y^2}$,
and by Exercise sheet 2, even when $\phi(y) \geq \frac{1}{\sqrt{5}y^2}$.
If $\phi$ decays slowly enough then one can easily show that
\[ N_\alpha(\phi,Q) = 2 \cdot \underbrace{\int_1^Q y \phi(y) dy}{Vol S_\alpha (\phi,Q)}(\text{It } \underbrace{o(1)}{\text{tends to } 0 \text{ as } Q \to \infty}) \text{ as Q } \to \infty \]
More specifically, using tools we develop in Chapter 3, one can easily show that
\[ \#\Z^2 \cap S_\alpha(\phi,Q) = 2 \cdot \int_1^Q y \phi(y) dy + \mathcal{O}(Q), \]
using Möbius-inversion, one can show that 
\[ N_\alpha(\phi,Q) = \frac{2}{S(2)} \cdot \int_1^Q y \phi(y) dy + \mathcal{O}(Q \log{Q}).\]
So we get an asymptotic formula
\[ N_\alpha (\phi,Q) \sim \frac{2}{S(2)} Vol S_\alpha (\phi,Q) \]
provided
\[ \frac{Q \log Q}{\int_1^Q y \phi(y)dy} \to 0 \text{ as } Q \to \infty. \]
So, e.g., if $\phi(y) \geq \frac{(\log y)^2}{y}$.

However, the case when $\phi(y)$ decays much quicker is more interesting.
Serge Lang in 1967 proved that if $\alpha$ is real quadratic then
\[ N_\alpha(\frac{1}{x^2},Q) = c_\alpha \cdot \log(Q) + \mathcal{O}(1)\text{. } (c_\alpha > 0). \]
He mentioned that it would seem quite difficult to prove an asymptotic result for algebraic $\alpha$, let alone transcended.\\
Adams showed
\[ N_e(\frac{1}{x^2},Q) = c_e \cdot \frac{\log Q}{\log \log Q} + \mathcal{O}(1) (c_e > 0) \]
where $ e = 2.7122\dots$\\
Lang and Adams both used continuous fractions expansion. How can one prove asymptotics for $N_\alpha(\phi,Q)$?
Here is an example.

\begin{ex}
  Suppose $\phi(x) = \frac{1}{2x^2}$. Consider the continuous fraction expansion $\alpha = [a_0,a_1,a_2,\dots]$.
  By Theorem 1.2.3 we know $| \alpha - \frac{p}{q}| < \phi(q) \Rightarrow \frac{p}{q}$ is a convergent.
  Moreover, if $\frac{p}{q} = \frac{p_n}{q_n}$ is the $n$-th convergent then $|\alpha - \frac{p}{q}| < \frac{1}{a_{n+1} q^2}$.
  So if all $a_i > 1$ then $|\alpha - \frac{p}{q}| < \phi(q) \forall$ convergent $\frac{p}{q}$.\\
  Hence, $N_\alpha(\phi,Q) = \#\{n: q_n \leq Q \}$.\\
  So, we need to compute the number of convergents $\frac{p_n}{q_n}$ with $q_n \leq Q$. 
  We shall soon see that this is rather simple if $\alpha = [b,a,b,a,b,a,\dots]$ with $a \divides b$.\\
  We will get back to this after Theorem 1.2.5.
\end{ex}
%----------

A continued fraction $[a_0,a_1,a_2,\dots]$ is called \emph{periodic} if
\[ \exists k \in \N  \text{ and } L \in \N_0 \text{ such that } a_{k+l} a_l \forall l \geq L.\]
In this case we write $[a_0,a_1,a_2,\dots] = [a_0,\dots,\bar{a_L,a_{L+1},\dots,a_{L+k-1}}]$.

\begin{theorem}
  $\theta = [a_0,a_1,a_2,\dots]$ is periodic $\iff \theta$ is real quadratic ($\theta$ is real quadratic means $\exists D \in \Z[x]\setminus 0$ with $D(\theta) = 0$, but $\theta \nin \Q$ and $\theta \in \R$)
\end{theorem}

See Ex Sheet 2 \#3 for a special instance.\\
A proof can be found, e.g., in Hardy \& Wright "The Theory of numbers", Oxford University press\\
\\
Let's go back to the problem of computing  $p_n, q_n$ of the $n$-th convergent.
The general recursion formula is unhandy.
But in certain cases there is a simple explicit formula. 
Consider $\theta = [b,a,b,a,\dots] = [\bar{b,a}]$ and suppose $b = a \cdot c$ for some $c \in \N$.
Now $\theta = b + \frac{1}{a + \frac{1}{b + \frac{1}{\dots}}} = b + \frac{1}{a + \frac{1}{\theta}}$.
Thus $\underbrace{a \theta^2 - ab\theta - b}{\theta^2 - b\theta -c} = 0$, so $\theta = \frac{b + \sqrt{b^2+4c}}{2}$
and we put $\bar{\theta} = \frac{b - \sqrt{b^2+4c}}{2}$.

\begin{theorem}
  The $p_n$ and $q_n$ of the $n$-th convergent $\frac{p_n}{q_n}$ of $\theta = [\bar{b,a}] (b = ac)$ are give by
  \[ p_n = c^{-\lfloor\frac{n+1}{2} \rfloor} \cdot U_{n+2}, q_n =c^{-\lfloor\frac{n+q}{2} \rfloor} \cdot u_{n+1}\]
  where
  \[ u_n = \frac{\theta^n - \bar{\theta}^n}{\theta - \bar{\theta}}.\]
  (Recall: $\theta = \frac{b+\sqrt{b^2+4c}}{2}, \bar{\theta} = \frac{b-\sqrt{b^2+4c}}{2}$, so $\theta - b\theta -c =0, \bar{\theta}^2 - b \bar{\theta} -c = 0$)
\end{theorem}


\end{document}
