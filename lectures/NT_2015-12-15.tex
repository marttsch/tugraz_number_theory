\documentclass[NumTh.tex]{subfiles}
\begin{document}

Recall that all rings in this Chapter 3 are integral domains with $1$ (unless specified otherwise).
For rings $A \subset B$ and $x \in B$ we write $A[x]$ for the smallest ring contains $A$ and $x$.

\begin{theorem}\label{th_3_3_2}
  Let $A \subset B$ be rings and $x \in B$.
  The following statements are equivalent:
  \begin{enumerate}
    \item[i)] $x$ is integral over $A$
    \item[ii)] $A[x]$ is finitely generated as an $A$-module.
    \item[iii)] $A[x]$ is contained in a subring of $B$ which is finitely generated as an $A$-module.
  \end{enumerate}
\end{theorem}

\begin{proof}
  $i) \Rightarrow ii)$  If $x$ is integral over $A$ then
  \[ x^n + a_1 x^{n-1} + \dots + a_n = 0 \; (a_i \in A) \]
  Thus $x^n = - (a_1x^{n-1} + \dots + a_n)$ and so $A[x]$ is generated by $1,x,\dots,x^{n-1}$ as an $A$-module. \\
  $ii) \Rightarrow iii)$ Trivial \\
  $iii) \Rightarrow i)$ Suppose $A[x] \subset C$ for a subring $C$ of $B$ that is finitely generated as an $A$-module.
  As $C$ is a ring and $x \in C$ we have
  \[ x \cdot C \subset C \text{,} \] 
  i.e., $y \in C \implies x \cdot y \in C$.
  Let $y_1,\dots,y_n$ be generators for $C$ and express 
  \[ x \cdot y_i = \sum_{j} a_{ij}y_j \]
  with $a_{ij} \in A$ ($1 \leq i \leq n$).
  We get a matrix equation 
  \begin{align*}
    \begin{pmatrix}
      xy_1 \\
      \vdots \\
      xy_n
    \end{pmatrix}
    = T
    \begin{pmatrix}
      y_1 \\
      \vdots \\
      y_n
    \end{pmatrix}
  \end{align*}
  with $T = [a_{ij}]$.
  As $1 \in A \subset C$ the vector $
  \begin{pmatrix}
    y_1 \\
    \vdots \\
    y_n
  \end{pmatrix}
  \neq 0$.
  Now
  \begin{align*}
    (xI - T) 
    \begin{pmatrix}
      y_1 \\
      \vdots \\
      y_n
    \end{pmatrix}
    = 0 \text{.}
  \end{align*}
  Hence $\det(xI -T) = 0$.\\
  Now 
  \[ \det(xI - T) = x^n + Q(x) \]
  where $Q(x) \in A[x]$ and $\deg Q \leq n-1$.\\
  
  This proves that $x$ is integral over $A$.
\end{proof}

\begin{cor}\label{cor_3_3_3}
  Let $A \subset B \subset C$ be rings and suppose $C$ is integral over $B$ and $B$ is integral over $A$.
  Then $C$ is  integral over $A$.
\end{cor}

\begin{proof}
  Let $x \in C$.
  We want to show that $x$ is integral over $A$.
  Now 
  \[ x^n + b_1 x^{n-1} + \dots + b_n = 0 \; (b_i \in B) \]
  Let 
  \[ \tilde{B}_i = A[b_1,\dots,b_i] \; (0 \leq i \leq n) \text{.} \]
  Then $b_i$ is integral over $\tilde{B}_{i-1}$ and so by Theorem \ref{th_3_3_2} $\tilde{B}_i$ is finitely generated over $\tilde{B}_{i-1}$.
  By Lemma \ref{l_3_3_1} we conclude that $\tilde{B}_n$ is finitely generated over $A$.
  As $b_1,\dots,b_n \in \tilde{B}_n$ $x$ is integral over $\tilde{B}_n$.
  Thus by Theorem \ref{th_3_3_2} $\tilde{B}_n[x]$ is finitely generated over $\tilde{B}_n$.
  Again by Lemma \ref{l_3_3_1} we get that $\tilde{B}_n[x]$ is finitely generated over $A$ and thus by Theorem \ref{th_3_3_2} also integral over $A$.
\end{proof}

\begin{cor}\label{cor_3_3_4}
  Let $A \subset B$ be rings. Then 
  \[ A_B = \{ b \in B : b \text{ integral over } A \} \]
  is a ring.
\end{cor}

\begin{proof}
  It suffices to show that 
  \[ x,y\in A_B \implies x \cdot y , x+y \in  A_B \text{.} \]
  So let $x,y \in A_B$.
  By Theorem \ref{th_3_3_2} $A[x]$ is finitely generated as an $A$-module.
  As $y$ is integral over $A$ it is also integral over $A[x]$ and thus 
  \[ (A[x])[y] = A[x,y] \] 
  is finitely generated as an $A[x]$-module.
  By Lemma \ref{l_3_3_1} $A[x,y]$ is finitely generated over $A$.
  Thus by Theorem \ref{th_3_3_2} every element in $A[x,y]$ is integral over $A$; in particular $x \cdot y$ and $x+y$
\end{proof}

\begin{rem}
  \begin{itemize}
    \item Note that $\overbar{\Q} = \Q_\C$, so $\overbar{\Q}$ is a ring by Corollary \ref{cor_3_3_4}.\\
    But $\overbar{\Q}$ is even a field; indeed if $x \in \bar{\Q}$, $x \neq 0$ then 
    \begin{align*}
      x^n + a_1 x^{n-1} + &\dots + a_n = 0 \; (a_i \in \Q) \\
      &\implies (x^{-1})^n + \frac{a_{n-1}}{a_n}(x^{-1})^{n-1} + \dots + \frac{1}{a_n} = 0 
    \end{align*}
    \item The ring $\Z_\C = \{ \alpha \in \C: \alpha \text{ integral over } \Z \}$ is called the ring of algebraic integers.
    \item Let $A$ be a field and $\alpha$ a root of a non-zero polynomial with coefficients in $A$,
    i.e., $\alpha$ is algebraic over $A$.\\
    We write $f_\alpha(x)$ for the monic minimal polynomial of $\alpha$ over $A$,
    i.e., the monic polynomial in $A[x]$ of minimal degree that vanishes at $\alpha$.\\
    If $h(x) \in A[x]$, $h \neq 0$ and $h(\alpha) = 0$ then $f_\alpha \divides h$ in $A[x]$ as follows from the Euclidean division algorithm.
    \end{itemize}
\end{rem}

\begin{lemma}\label{l_3_3_5}
  Let $\alpha \in \bar{\Q}$ be an algebraic number and $f_\alpha(x)$ the monic minimal polynomial over $\Q$.
  Then:
  \begin{align*}
    \alpha \in \Z_\C \iff f_\alpha(x) \in \Z[x] \text{.}
  \end{align*}
\end{lemma}

\begin{proof}
  "$\Leftarrow$" trivial\\
  "$\Rightarrow$" $\exists h \in \Z[x]$ monic with $h(\alpha) = 0$.
  Then $f_\alpha \divides h$ in $\Q[x]$.
  Hence all roots of $f_\alpha$ vanish at $h$.
  Hence, all roots of $f_\alpha$ are algebraic integers.
  But the coefficients of $f_\alpha$ are symmetric functions in the roots ($f_\alpha$ is monic!) thus the coefficients are also algebraic integers,
  and they are also in $\Q$.
  We already know that $\Z_\C \cap \Q = \Z$ thus $f_\alpha \in \Z[x]$.
\end{proof}

A \emph{number field} $K$ is a subfield of $\bar{\Q}$ which as a $\Q$-vector space has finite dimension.
The latter is called the degree of $K$ over $\Q$ and denoted by $[K:\Q]$.
By the "primitive element theorem" there exist $\alpha \in K$ such that 
\[ K = \Q(\alpha) = \{ \frac{P(\alpha)}{Q(\alpha)} : P,Q \in \Q[x], Q(\alpha) \neq 0 \} \text{.} \]
In fact $\Q(\alpha) = \Q[\alpha]$ and $1,\alpha,\dots,\alpha^{\deg(f_\alpha) - 1}$ is a $\Q$-basis for $K$,
thus 
\[ [K:\Q] = \deg(f_\alpha) \text{ (see exercise sheet 6).} \]
The integral closure $\Z_K$ of $\Z$ in $K$ is usually dnoted by $O_K$.\\

The following result is central in algebraic number theory.

\begin{theorem}\label{th_3_3_6}
  If $K$ is a number field then $O_K$ has a unique prime factorization of ideals,
  i.e., if $\mathfrak{a} \neq (1),(0)$ is an ideal in $O_K$ then there exist prime ideals $\mathfrak{p}_1,\dots,\mathfrak{p}_s$ such that
  \[ \mathfrak{a} = \mathfrak{p}_1 \dots \mathfrak{p}_s \]
  and this decomposition is, up to the order of the factors, unique.
\end{theorem}

\end{document}
