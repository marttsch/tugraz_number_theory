\documentclass[NumTh.tex]{subfiles}
\begin{document}

Next, we want to introduce a notion of "arithmetic complexity" on elements in $\Q^{n+1}$, the so-called projective height:
\[ H_{\p^n}: \Q^{n+1} \to \left[1,\infty\right) \]
defined by
\[ H_{\p^n}(\underline{x}) = \prod_{v \in M_\Q} \abs{\underline{x}}_v \]
where $\abs{\underline{x}}_v = \max \{\abs{x_0}_v,\dots,\abs{x_n}_v \}$.

\begin{ex}
  If $\underline{x} = (x_0,\dots,x_n)$ where $x_0,\dots,x_n \in \Z$ and $\gcd(x_0,\dots,x_n) = 1$.
  Then $\abs{x}_p = 1$ for all primes p. Hence,
  \[ H_{\p^n} (\underline{x}) = \max \{\abs{x_0}_\infty,\dots,\abs{x_n}_\infty \} \text{.}\]
  Note that $H_{\p^n} ( \lambda \cdot \underline{x}) = H_{\p^n}(\underline{x}) \forall \lambda \in \Q \setminus 0$.
\end{ex}

\begin{theorem}[1.5.3 p-adic Subspace Theorem, Schlickewei and Schnidt]
  Let $\delta > 0$ and let $S \subset M_\Q$ be finite and with $\infty \in S$.
  For $v \in S$ let $L_{v_1},\dots,L_{v_n}$ be $n$ linearly independent linear forms in $n$ variables with coefficients in $\Q$.
  Then the set of solutions $\underline{x} \in \Q^{n+1} \setminus 0$ of
  \[ \prod_{v \in S} \prod_{i=1}^n \frac{\abs{L_{v_i}(\underline{x})_v}}{\abs{\underline{x}}_v} < H_{\p^{n-1}}(\underline{x})^{-n-\delta} \]
  lie in finitely many proper subspace of $\Q^n$.
\end{theorem}

An interesting consequence is a finiteness result for $S$-unit equations.\\
$S$-integers and $S$-units:\\
Let $v$ be a non-Archimedean absolute value or a field $K$. Then
\[ O_v =  \{ x \in K : \abs{x}_v \leq 1 \} \]
is called the valuation ring of $v$. It is indeed a ring, e.g., $\abs{x}_v,\abs{y}_v \leq 1$ then
\[ \abs{x+y}_v \leq \max \{ \abs{x}_v,\abs{y}_v \} \leq 1 \text{.} \]
In particular, if $K = \Q$ and $v = p$ then $O_v$ is a sub-ring of $\Q$.\\
\\
Now let $S \subset M_\Q$  be finite and $\infty \in S$. We define the set of \emph{$S$-integers} $O_S$ to be
\[ O_S = \cap_{v \nin S} O_v \text{.}\]
As $\infty \in S$, this is an intersection of rings, hence a ring.

\begin{ex}
  If $S = \{\infty\}$, then $O_S = \Z$.
  If $S = \{\infty,p_1,\dots,p_s \}$ then
  \[ O_S = \{ \frac{m}{p_1^{a_1} \dots p_s^{a_s}} : m \in \Z, a_1,\dots,a_s \in \N_0 \} \]
\end{ex}

We say $x \in \Q$ is an $S$-unit if $x \neq 0$ and $x, x^{-1}$ are both in $O_S$.
So if $S = \{ \infty \}$, then $\pm 1$ are the only $S$-units. If $S = \{ \infty,p_1,\dots,p_s \}$ then $x$ is an $S$-unit $\iff x = \pm \prod_{p \in S \setminus \infty} p^a_p$ (and $ap \in \Z$).

\begin{theorem}[1.5.4 $S$-unit equation]
  Let $S \subset M_\Q$ be finite, and $\infty \in S$. Let $\alpha_0,\dots,\alpha_n$ be non-zero and in $\Q$.
  Then 
  \[ \alpha_0 x_0 + \dots + \alpha_n x_n = 0 \]
  has only finitely many solutions $\underline{x} = (x_0,\dots,x_n)$ if:
  \begin{itemize}
    \item $x_0,\dots x_n$ are $S$-units
    \item we identify proportional solutions (i.e., $\underline{x} = \lambda \underline{x}$ for $ \lambda \in \Q \setminus 0$).
    \item no proper sub-sum vanishes, i.e., $\sum_{I} \alpha_i x_i \neq 0$ for all $\emptyset \subsetneqq I \subsetneqq \{0,1,\dots,n\}$.
  \end{itemize}
\end{theorem}

\begin{rem}
  $S = \{\infty, p \}$ $x_0+x_1+x_2+x_3 = 0$ then $x_0 = - x_1 =  1$, $x_2 = - x_3 = p^a$ ($a \in \Z$)
  are solutions in $S$-units. So non-vanishing condition is needed!
\end{rem}

\begin{ex}
  The exponential Diophantine equation 
  \[ 3^x + 5^y - 7^z = 1 \]
  has solutions, e.g., $(x,y,z) = (0,0,0)$ or $(x,y,z) = (1,1,1)$.
  However, with $S = \{ \infty,3,5,7 \}$ each solution $(x,y,z)$ yields a solution $u_0 = 3x$, $u_1 = 5y$, $u_2 = - 7z$, $u_3 = - 1$ of the $S$-unit equation $u_0 + u_1 + u_2 + u_3 = 0$.
  These solutions are all non-proportional.
  Moreover, no sub-sum vanishes unless $x y z = 0$ but then we easily see that $x=y=z=0$.
  Hence, Theorem 1.5.4 yields finiteness.
\end{ex}

\begin{proof}[assuming Theorem 1.5.3]
  Induction on $n$. If $n=1$ then $\alpha_0 x_0 + \alpha_1 x_1 = 0$, so all solutions are proportional to $(1, -\frac{\alpha_0}{\alpha_1}$.
  Now suppose the claim holds for all $S$-unit equations in $\leq n$ variables.
  As $x_i$ are $S$-units we have
  \[ \abs{x_i}_v = 1 \forall v \nin S \text{.} \]
  By the product formula (PF)
  \[ 1 = \prod_{v \in M_\Q} \abs{x_i}_v = \prod_{v \in S} \abs{x_i}_v \text{,}\]
  and thus
  \[ \prod_{v \in S} \prod_{i=0}^n \abs{x_i}_v = 1 \text{.} \]
  Let $\tilde{\underline{x}} = (x_0,\dots,x_{n-1})$.
  For each $v \in S$ pick $i(v)$ with $0 \leq i(v) \leq n-1$.
  So we get $n^{\#S}$ such tuples $(i(v))_{v \in S}$.
  Choose one of those tuples and consider all solutions of $\alpha_0 x_0 + \dots + \alpha_n x_n = 0$ with
  \[ \abs{\tilde{\underline{x}}}_v = \abs{x_{i(v)}}_v \]
  Choose the set of linear forms
  \[ \{ L_{v_j} : 1 \leq j \leq n \} = \{ X_0,X_1,\dots,X_{n-1},\frac{\alpha_0}{\alpha_n}X_0+\dots + \frac{\alpha_{n-1}}{\alpha_n}X_{n+1} \} \setminus \{X_{i(v)}\} \]
  Then
  \[ \prod_{ v \in S} \prod_{j=1}^n \frac{\abs{L_{v_j}(\tilde{\underline{x}})}_v}{\abs{\tilde{\underline{x}}}_v} 
  = \frac{1}{H_{\p{n+1}}(\tilde{\underline{x}})^{n+1}} \]
  By Theorem 1.5.3 the solutions $\tilde{\underline{x}}$ lie in finitely many proper subspaces.
  Take one of these  then all elements in this subspace satisfy an equation
  \[ c_0 x_0 + \dots c_{n-1} x_{n-1} = 0  (c_i \in \Q \text{, not all } = 0 \text{!})\]
  Let $J_0$ be the set of $i$ with $c_i \neq 0$.
  Then 
  \begin{align}
    \sum_{i \in J_0} c_i x_i = 0 \text{ marker(S)}
  \end{align}
  is an $S$-unit equation in $\leq n$ unknowns.
  For every solution of "marker(S)" there is a set $J \subset J_0$, $J \neq \emptyset$, such that
  \[ \sum_{i \in J} c_i x_i = 0 \]
  and \underline{no} sub-sum vanishes.
  By the induction hypotheses, up to proportionality, we get only finitely many solutions.
  Moreover, the number of possible choices $J$ is finite.
  Therefore it suffices to consider solutions $\{x_i\}_{i \in J}$ that are proportional to a fixed $\{u_i\}_{i \in J}$,
  i.e., $x_i = \xi u_i (i \in J)$.
  Returning to our initial equation $\sum_{i=0}^n \alpha_i x_i = 0$ we get
  \[ \xi ( \sum_{i \in J} \alpha_i u_i) + \sum_{i \nin J} \alpha_i x_i) = 0 \]
  If $\sum_{i \in J} \alpha_i x_i \neq 0$ then the above is an $S$-unit equation in $1 + (n+1) - \#J \leq n$ unknowns,
  namely $\xi, x_i \: (i \nin J)$. By the induction hypothesis we get only finitely many non-proportional solutions
  $\{ x_i\}_{i=0}^n$ for which no sub-sum vanishes. Finally, if $\sum_{i \in J} \alpha_i x_i = 0$ then $\sum_{i \nin J} \alpha_i x_i = 0$ and we ignore these solutions by assumption of the Theorem.
\end{proof}


\end{document}
