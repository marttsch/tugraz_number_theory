\documentclass[NumTh.tex]{subfiles}
\begin{document}

\begin{theorem}
  $e$ is transcendental. (Hermite 1873)
\end{theorem}

Exploit properties of $e^x$ ($(e^x)^\prime = e^x$)
\begin{align*}
  e^x \cdot \int_0^x e^{-t} f(t) dt &= \{ \text{using integration by points} \} \\
  &= e^x f(0) - f(x) + e^x \cdot \int_0^x e^{-t} f^\prime(f) dt \text{.}
\end{align*}
Plug into this formula $f^\prime, f^{\prime \prime},\dots$ and take the sum of all equations, Cancellations occur!
Choose $f$ polynomial if $n > \deg(f)$, $f^{(n)}(x) = 0$.
Define
\[ F(x) = \sum_{j=0}^\infty f^{(j)}(x) \text{ is a finite sum} \]
Then the sum of all equations gives
\begin{align} \label{doublestar}
  e^x \cdot \int_0^x e^{-t} f(t) dt = e^x F(0) - F(x) \text{.}
\end{align}

\begin{proof}
  Assume $e$ is algebraic.
  There exist integers $a_0, a_1,\dots,a_n \in \Z$ such that
  \[ a_0 + a_1 e + \dots + a_n e^n = 0 \text{.} \]
  Apply (\ref{doublestar}) when $x$ ranges from $0$ to $n$, and multiply by $a_k$.
  \begin{align*}
    a_k e^k \cdot \int_0^k e^{-t} f(t) dt &= a_k e^k F(0) - a_k F(k) \text{, } k=0,1,\dots,n \\
    \sum_{k=0}^n a_k e^k \int_0^k e^{-t} f(t) dt &= \underbrace{\left( \sum_{k=0}^n a_k e^k \right)}_{0} F(0) - \sum_{k=0}^n a_k F(k) \\
    - \sum_{k=0}^n a_k e^k \int_0^k e^{-t} f(t) dt &= \sum_{k=0}^n a_k F(k) \text{.}
  \end{align*}
  Idea: We choose $f$ so that the RHS is a nonzero integer, and thus $m$ absolute value $\geq 1$,
  but also that the LHS can be made arbitrarily small.
  \\
  Hermite's choice of $f$:
  \begin{align}
    f(t) = \frac{1}{(p-1)!} t^{p-1} g(t)^p \text{,}
  \end{align}
  $p$ prime number we specify $p$ later!
  \[ g(t) = (t-1) \cdots (t-n) \text{,} \]
  $n = \deg$ of the minimal polynomial of $e$.
  We choose $p > n$, $p > \abs{a_0}$, but if needed also that is arbitrarily large.
  This will be explained later.
  \begin{align*}
    \abs{LHS} &= \abs{\sum_{k=0}^n a_k e^k \int_0^k e^{-t} f(t) dt} \\
    &\overset{\text{triangle inequ.}}{\leq} \sum_{k=0}^n \abs{a_k} e^k \abs{\int_0^k e^{-t}f(t)dt} \\
    &\leq \sum_{k=0}^n \abs{a_k} e^k \int_0^k e^{-t} \abs{f(t)}dt \\
    \text{comment} \\
    & e^{-t} \leq 1 \text{ for } 0 \leq t \leq 1 \\
    \text{comment end} \\
    &= \sum_{k=0}^n \abs{a_k} e^k \int_0^k \abs{f(t)} dt\\
    \text{comment} \\
    &\text{ Note that the maximum value of functions } t,t-1,\dots t_n \text{ on } [0,n] \text{ is } n \\
    & \abs{f(x)} \text{ on } [0,n] \text{ is bounded above by } \frac{1}{(p-1)!} n^{p-1} n^{np} \\
    \text{comment end} \\
    &\leq \sum_{k=0}^n \abs{a_k} e^k \frac{1}{(p-1)!} n^{np+p-1} \int_0^k 1 dt \\
    &= \sum_{k=0}^n \abs{a_k} e^k \frac{n^{np+p-1}}{(p-1)!} \cdot \underbrace{k}_{\leq n} \\
    &\leq \sum_{k=0}^n \abs{a_k} e^k \frac{n^{np+p}}{(p-1)!} \\
    &\leq \sum_{k=0}^n \abs{a_k} e^n \frac{(n^{n+1})^{p-1}}{(p-1)!} n^{n-1} \\
    &\text{comment} \\
    & \lim_m \frac{c^m}{m} = 0; \text{ In our case } \lim_{p\to \infty} \frac{(n^{n+1})^{p-1}}{(p-1)!} = 0 \\
    & \text{So, the last expression for $p$ sufficently large will be } < 1 \text{.}
    &\text{comment end} \\
  \end{align*}
  Therefore, for sufficiently large $p$, the LHS is $m$ absolute value $<1$.\\
  \\
  We now turn attention to the RHS.
  \begin{align*}
    RHS &= \sum_{k=0}^n a_k F(k) \text{.}
  \end{align*}
  Recall $F(k) = \sum_{j=0}^{\deg f} f^{(i)}(k)$.
  Show that RHS is a nonzero integer.
  Idea: we will show that $f^{(0)}(k) \in \Z$ for all $j$, and that only $F(0)$ is not divisible by $p$.
  
  \begin{align*}
    f(t) &= \frac{1}{(p-1)!} t^{p-1} \cdot g(t) \\
    g(t) &= (t-1) \cdots (t - m)
  \end{align*}
  We analyse $F(0)$, that is the case $k = 0$.
  \[ F(0) = \sum_{j=0}^\infty f^{(j)}(0) \text{.}\]
  \[ \deg(f) = (p-1) + np = n (p+1) - 1 \]
  Since $0$ is a root of $f$ of multiplicity $p-1$, it follows that
  \[ f^{(j)}(0) = 0\]
  for all $j < p-1$.\\
  $f^{(p-1)}(0) = ?$, $f^{(j)}(0) = ?$ when $j \geq p$.\\
  We look at the Taylor expansion of $f$ at $0$.
  \begin{align*}
    f(x) &= \frac{f^{(p-1)}(0)}{(p-1)!} \cdot t^{p-1} + \dots + \frac{f^{(j)}(0)}{j!} \cdot t^j
  \end{align*}
  The coefficients of $f$ next to $t^{p-1}$ is $\frac{f^{(p-q)}(0)}{(p-1)!}$,
  but by definition of $f$, this coefficient is 
  \[ (\frac{(-1)(-2)\cdots(-n))^p}{(p-1)!} \implies f^{(p-1)}(0) = ((-1)^n \cdot n!)^p \]
  For $p > n$, this is not divisible by $p$.\\
  We now show that $f^{(j)}(0)$ is an integer divisible by $p$ when $j \geq p$.
  Look at the Taylor expansion of $f$ and definition of $f$.\\
  Next to the $t^j$:
  \begin{align*}
    \frac{f^{(j)}(0)}{j!} = \frac{1}{(p-1)!} x
  \end{align*}
  the coefficients next to $t^j$ in the polynomial are $t^{p-1}g(t)^p$:
  So, $f^{(j)}(0) = \frac{j!}{(p-1)!} x$ is an integer.
  This implies $f^{(0)} \in \Z$, divisible by $p$.
  Thus,
  \begin{align*}
    f^{(j)}(0) &= 0 \text{ when } j < p-1 \\
    f^{(p-1)} &\in \Z \text{ not divisible by } p \\
    f^{(j)}(0) \in \Z \text{ divisible by } p \text{.}
  \end{align*}
  It follows that $F(0) \in \Z$, and not arbitrary.
  Now , $p$ was chosen so that $p > \abs{q_0}$
  \begin{align*}
    &\implies \abs{q_0 F(0)} \geq 1 \text{ integer }
    &\implies a_0 F(0) \text{ not divisible by } p \text{, integer.}
  \end{align*}
  \[ RHS = \sum_{k=0}^n a_k F(k) \]
  $F(k) = ?$ when $k = 1,2,\dots,n$.
  Follow the same method (look at the Taylor expansion of $f$ at $k$).
  \[ f(t) = \frac{1}{(p-1)!} t^{p-1} \cdot g(t)^p \]
  \[ g(t) = (t-1) \\cdots (t-n) \]
  In the case, $k \in \{1,2,\dots,n \}$ is a root of $f$ of multiplicity by $p$.
  Because of this $f^{(p-1)}(k)$ is going to be $0$, and all $f^{(j)}(K)$, for $0 \leq j \leq \deg(f)$,
  will be divisible by $p$.
  Therefore,
  \[ \sum_{k=0}^n a_k F(k) = a_0 F(0) + \] %TODO text missing
  \todo{text missing}
  Thus,
  \[ \sum_{k=0}^n a_k F(k) \neq 0 \] %TODO text missing
  \todo{text missing}
\end{proof}

\begin{theorem}
  $\pi$ is transcendental. (Lindemann 1882)
\end{theorem}

\begin{proof}
  Assume $\pi$ is algebraic.
  Now $i$ is algebraic (a root of $x^2+1 = 0$).
  It follows that $i \cdot \pi$ is algebraic.
  Recall that
  \[ e^{i \pi} +1 = 0 \text{.} \]
  Let $\theta_1 \in \Q[x]$ such that $\theta_1(i \pi) = 0$,
  let $n = \deg(\theta_1)$.
  So, $\theta_1$ has $n$ roots.
  \[ \alpha_1 = i \pi, \alpha_2,\dots,\alpha_n \]
  be other root of $\theta_1$.
  \[ e^{\alpha_1} + 1 = 0 \text{.} \]
  Thus,
  \begin{align}\label{onestar}
    (e^{\alpha_1} + 1)(e^{\alpha_2} + 1) \cdots (e^{\alpha_n} + 1) = 0
  \end{align}
  Expand the sum (\ref{onestar}), and note the summands are of type $e^\xi$
  \[ \xi = \xi_1 \alpha_1 + \xi_2 \alpha_2 + \dots + \xi_n \alpha_n \]
  for $\xi_i \in \{0,1\}$.
  Exponent $\xi = 0$ occurs at least once (corresponds to the summands $1 \: 1 \: \dots \: 1$).
  Let $k$ be the number of the summands with the exponent $0$.
  There are $2^n-k$ nonzero sums of type (\ref{doublestar}).
  Let $\beta_1,\dots,\beta_r$ be nonzero sums of type (\ref{doublestar}).
  It follows that 
  \[ k \cdot e^0 + e^{\beta_1} + \dots + e^{\beta_r} = 0 \]
  and each $\beta_j \neq 0$.
  Let $\theta$ be a polynomial whose roots are $\beta_i$'s,
  \[ \theta(x) = c x^r + c_1 x^{r-1} + \dots + \underbrace{c_r}_{\neq 0} \]
  We will again exploit the fact that 
  \[ - \int_0^x e^{-y} f(y) dy = e^{-x} F(x) - F(0) \]
  with
  \[ F = \sum_{j=0}^x f^{(j)}(x) \]
  \\
  \\
  \[ f(x) = \frac{c^s x^{p-1}}{(p-1)!} \theta(x)^p \]
  $s = rp -1$.
  
  $\theta = \sigma$!!!
\end{proof}

\end{document}
