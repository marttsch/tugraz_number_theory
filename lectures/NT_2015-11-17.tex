\documentclass[NumTh.tex]{subfiles}
\begin{document}

\section{Geometry of Numbers}

References:
\begin{itemize}
  \item J.W.S. Cassels "An Introduction to the Geometry of Numbers"
  \item W.M. Schmidt Lecture Notes M. 785 and 1467
\end{itemize}

\subsection{Basic notions}
Let $R$ be an integral domain (with $1$), and $n \in \N$.
We write $\mat_n(R) = \{ n \times n \text{ matrices with entries in } R\}$
and $\gl_n(R) = \{ A \in \mat_n(R): A \text{ is invertibel and } A^{-1} \in \mat_n(R) \}$.
Then $(\gl_n(R), \cdot)$ is a group. We say $u \in R$ is a unit (in $R$) if $\exists u^\prime \in R$ such that $u^\prime \cdot u = 1$.\\
If $A \in \gl_n(R)$ then $(\det A)^{-1} = \det A^{-1} \in R$. So $\det A$ is a unit in $R$.
On the other hand if $A \in \mat_n(R)$ and $\det A$ is a unit in $R$ then $A^{-1} = (\det a)^{-1} \adj(A) \in \mat_n(R)$ as the adjungate matrix $\adj(A)$ of $A$ clearly is in $\mat_n(R)$. 
Thus we have
\[ \gl_n(R) = \{ A \in \mat_n(R) : \det A \text{ is a unit in } R \} \text{.} \]
In particular, $\gl_n(\Z) = \{ A \in \mat_n(\Z): \det A = \pm 1 \}$.\\
\\
Let $n \in \N$. A \underline{lattice} $\Lambda$ in $\R^n$ is a set of the form
\[ \Lambda = A \Z^n = \{ Ax : x \in \Z^n \} \]
where $A \in \gl_n(\R)$.
The column vectors of $A$ are called a \underline{basis} of $\Lambda$.

\begin{lemma}[2.1.1 \label{l_2_1_1}]
  Let $A,B \in \gl_n(\R)$. Then 
  \[ A \Z^n = B \Z^n \iff \exists T \in \gl_n(\Z) \text{ such that } B = AT \]
\end{lemma}

\begin{proof}
  "$\Leftarrow$" If $B = AT$ with $T \in \gl_n(\Z)$ then $T \Z^n = \Z^n$.
  Hence, $B \Z^n = A \Z^n$.\\
  "$\Rightarrow$" If $A\Z^n = B \Z^n$ then each column vector of $B$ lies in $A \Z^n$,
  thus $\exists T \in \mat_n(\Z)$ such that $B = AT$.
  Similarly $\exists T^\prime \in \mat_n(\Z)$ such that $A = B T^\prime$.
  Hence, $ A = A T T^\prime$.
  Thus $T^\prime = T^{-1}$. So $T \in \gl_n(\Z)$.
\end{proof}

By Lemma \ref{l_2_1_1} we see that if $\Lambda = A \Z^n$ then $\abs{\det A}$ is uniquely determined by $\Lambda$.
We call it the \underline{determinant} of $\Lambda$
\[ \det \Lambda = \abs{\det A} \]
Let $b_1,\dots,b_n$ be a basis of $\Lambda$ and $v \in \Lambda$. We set
\[ F_v = [0,1)\cdot b_1 + \dots + [0,1) \cdot b_n + v \]
and call it a fundamental cell of $\Lambda$.

Note that
\begin{itemize}
  \item $\det \Lambda = \vol F$
  \item $\R^n = \cap_{v \in \Lambda}^{disjoint with \bullet}$ is a partition of $\R^n$ (cf sheet 5).
\end{itemize}

Recall that $C \subset \R^n$ is called \underline{convex} if:
\[ x,y \in C \implies tx + (1-t)y \in C \forall t \in [0,1] \]
We say $C$ is symmetric if:
\[ x \in C \implies -x \in C\]
Recall that every convex set is measurable.

Let $C$ be a convex, compact and symmetric set in $\R^n$ with the origin in the interior of $C$.
Let $\Lambda$ be a lattice in $\R^n$.
Then we define the successive minima $\lambda_1,\dots,\lambda_n$ of $\Lambda$ with respect to $C$ by
\[ \lambda_i = \inf \{ \lambda : \lambda C \cap \Lambda \text{ contains } i \text{ linearly independent vectors} \} \text{.} \]

Note that $0 < \lambda_1 \leq \lambda_2 \leq \dots \leq \lambda_n < \infty$.

\begin{ex}
  \begin{itemize}
    \item $\Lambda = \Z^n$, $C = [-1,1]^n$.
    Then $\lambda_1 = \dots = \lambda_n = 1$.
    \item $\Lambda = \left(
    \begin{matrix}
      1 & 0 \\
      0 & 2
    \end{matrix}
    \right) \Z^2$, $C = [-1,1]^2$.
    Then $\lambda_1 = 1$, $\lambda_2 = 2$.
    \item $\Lambda = \left(
    \begin{matrix}
      1 & 0 \\
      0 & 2
    \end{matrix}
    \right) \Z^2$, $C = [-1,1] \times [-2,2]$.
    Then $\lambda_1 = \lambda_2 = 1$.
  \end{itemize}
\end{ex}

\begin{theorem}\label{th_2_1_2}
  Let $\Lambda$ be in $\R^n$. Then $\Lambda$ is a lattice in $\R^n$ if and only if:
  \begin{enumerate}
    \item[i)\label{th_2_1_2_1}] $(\Lambda,+)$ is a group
    \item[ii)\label{th_2_1_2_2}] $\Lambda$ contains $n$ linearly independent vectors
    \item[iii)\label{th_2_1_2_3}] $\Lambda$ is discrete, $\# S \cap \Lambda < \infty \forall \text{ compact } S \subset \R^n$.
  \end{enumerate}
\end{theorem}

\begin{proof}
  First suppose $\Lambda$ is a lattice.
  Then i) and ii) are clear.
  And iii) is clear, at least if $\Lambda = \Z^n$.
  But if $\Lambda = A \Z^n$ then $\# \Lambda \cap S = \# \Z^n \cap A^{-1} S$ as $x \mapsto A^{-1} x$ is continuous we have $S$ compact $\implies A^{1-} S$ compact.
  So this proves the first direction.\\
  Now let's suppose $\Lambda \subset \R^n$ such that i), ii), iii) hold.
  We use induction on $n$. \\
  Let $n=1$. Then $\Lambda$ contains a non-zero vector $b$ that is closest to the origin (using ii) and iii)).
  By i) we easily see that $\Lambda = b \Z$. So $\Lambda $ is a lattice in $\R^1$.\\
  Now suppose the claim holds in $\R^m$ if $m < n$.
  Let $u_1,\dots,u_n$ be $n$ linearly independent vectors in $\Lambda$.
  Consider the subspace $U = \left\langle u_1,\dots,u_{n-1} \right\rangle_\R$; thus $\dim U = n-1$.
  Let $\tilde{e_1},\dots,\tilde{e_{n-1}}$ be an orthonormal basis of $U$.
  Let
  \[ O \in \oth_n(\R) = \{ A \in \gl_n(\R): A^T A = I_n \} \text{ (the orthogonal group)} \]
  with
  \[ O(\tilde{e_i}) = e_i \; (1 \leq i \leq n-1) \]
  where $e_1,\dots,e_n$ is the canonical basis of $\R^n$.
  Hence, $O(U) = \R^{n-1} \times \{ 0 \}$, and
  \[ O( \Lambda \cap U) \subset \R^{n-1} \times \{ 0 \} \]
  is a discrete additive group that contains the $n-1$ linearly independent vectors $O(u_1),\dots,O(u_{n-1})$.
  Let $\Pi: \R^n \to \R^{n-1}$, $\Pi(x) = (x_1,\dots,x_{n-1})$.
  Then $\Pi \circ O(\Lambda \cap U)$ is also a discrete additive group that contains $n-1$ linearly independent vectors
  and it is also in $\R^{n-1}$.
  Hence, by induction hypothesis $\Pi \circ O(\Lambda \cap U)$ is a lattice in $\R^{n-1}$.
  So $\exists A_{n-1} \subset \gl_{n-1}(\R)$ such that $\Pi \circ O (\Lambda \cap U) = A_{n-1} \Z^{n-1}$.
  So 
  \[ O (\Lambda \cap U) = \left(
  \underbrace{\begin{matrix}
    A_{n-1} & 0 \\
    0 & 0
  \end{matrix}}{\tilde{A}}
  \right) \Z^n \]
  Now let $\mu = \inf \{ \abs{ w_n} : w = (w_1,\dots,w_n) \in O(\Lambda \setminus U) \}$.
  Suppose $v_1,v_2,v_3,\dots \in O(\Lambda \setminus U)$ with
  \[ \abs{ v_{in}} \to \mu \; (\text{as } i \to \infty \]
  where $v_{in}$ is the last coordinate.
  
  Adding elements from $O(U)$ does not change the last coordinate.
  Hence, we can assume that the vectors 
  \[ (v_{i1},\dots,v_{in-1}) \in [0,1)a_1 + \dots + [0,1) a_{n-1} \]
  where $a_i =$ column vector of $A_{n-1}$.
  In particular, the first $n-1$ coordinates of $v_i$ are  bounded in absolute value.
  But the absolute value of the last coordinate also tends to $\mu$, so is also bounded.
  As $\Lambda$ is discrete by iii) also $O(\Lambda)$ is discrete.
  Thus the sequence $v_i$ contains only finitely many vectors, in particular
  \[ \exists v \in O(\Lambda \setminus U) \text{ such that } v_n \overset{after v \to -v}{=} \mu \text{ and } \mu > 0 \text{.}\]
  Let $u \in O(\Lambda)$.
  Then also $u^\prime = u - \left[ \frac{u_n}{\mu} \right] \cdot v$ is in $O(\Lambda)$ ($O(\Lambda)$ is a group as $\Lambda$ is).
  So $0 \leq u_n^\prime < \mu$, so by minimality of $\mu$, $u_n^\prime = 0$.
  Hence, $u^\prime \in O(\Lambda) \cap \R^{n-1} \times \{ 0 \} = O(\Lambda) \cap O(U) = O(\Lambda \cap U)$.
  Now 
  \[ u = u^\prime + \left[ \frac{u_n}{\mu} \right] \cdot v \in O(\Lambda \cap U) + \Z \cdot v\]
  and thus
  \[ u \in \tilde{A} \Z^n + v \cdot \Z = \underbrace{\left[ (\tilde{a_1}) \dots (\tilde{a_{n-1}}) (v)  \right]} \Z^n = A \Z^n \]
  where $\tilde{a_i} = i$-th column vector of $\tilde{A}$.
  Thus $O(\Lambda) \subset A \Z^n$.
  Clearly (as $O(\Lambda)$ is a group) also $A \Z^n \subset O(\Lambda)$.
  Now the rows of $A$ are linearly independent.
  Thus $A \in \gl_n(\R)$.
  So $O(\Lambda)$ is a lattice and thus $\Lambda$ is a lattice.
\end{proof}

\end{document}
