\documentclass[NumTh.tex]{subfiles}
\begin{document}

\begin{lemma}[2.3.5 \label{l_2_3_5}]
  Let $C$ be a convex, symmetric star body in $\R^n$, let $\Lambda$ be a lattice in $\R^n$, and let $a_1,\dots,a_n$ be linearly independent vectors in $\Lambda$.
  Then there exists a basis $b_1,\dots,b_n$ of $\Lambda$ such that
  \[ f_C(b_j) \leq \max \{ f_C(a_j), \frac{1}{2} (f_C(a_1)+\dots+f_C(a_j)) \} \]
\end{lemma}

\begin{proof}
  Consider the sublattice $\Lambda^\prime = (a_1 \dots a_n) \Z^n \subset \Lambda$.
  By Theorem \ref{th_2_3_1} there exists a basis $c_1,\dots,c_n$ of $\Lambda$ such that
  \begin{align} \label{3_6}
    a_1 &= v_{11} c_1\\
    a_2 &= v_{21} c_1 + v_{22} c_2\\
    &\vdots\\
    a_n &= v_{n1} c_1 + \dots + v_{nn} c_n
  \end{align}
  with $v_ij \in \Z$ and $v_{ii} \neq 0$.
  Consider 
  \[b_j = c_j + t_{jj-1}a_{j-1} + \dots + t_{j1} a_1 \label{3_7}\]
  where $t_{ji} \in \R$.
  If $b_1,\dots,b_n$ are in $\Lambda$ then by (\ref{3_6}) they form a basis of $\Lambda$.
  How do we choose $t_ji$?
  If $v_jj = \pm 1$ then we put $b_j = \pm a_j$, which dearly is in the required form (\ref{3_7}) and obviously 
  \[ f_C(b_j) = f_C(a_j) \text{.} \]
  Now suppose $\abs{v_{jj}} \geq 2$.
  Now solving (\ref{3_6}) for $c_j$ yields
  \[ c_j = v_{jj}^{-1} a_j + k_{j j-1} a_{j-1} + \dots + k_{j1} a_1 \]
  with $k_{ji} \in \Q$.
  We choose $t_{ji} \in \Z$ such that
  \[ \abs{k_{ji} + t_{ji}} \leq \frac{1}{2} \]
  Then 
  \[ b_j \in \Lambda \text{ and } b_j = l_{jj} a_j + l_{j j-1} a_{j-1} + \dots + l_{j1} a_1 \]
  with 
  \begin{align*}
    \abs{l_{jj}} &= \abs{v_{jj}} \leq \frac{1}{2} \text{ and}\\
    \abs{l_{ji}} &= \abs{t_{ji} + k_{ji}} \leq \frac{1}{2} \; (i < j)
  \end{align*}
  Using that $C$ is a \underline{convex}, \underline{symmetric} star body we have the triangle-inequality. Hence,
  \[ f_C(b_j) \leq f_C(l_{jj}a_j) + \dots + f_C(l_{j1}a_1) = \abs{l_{jj}} f_C(a_1) + \dots + \abs{l_{j1}} f_C(a_1) \leq \frac{1}{2} (f_C(a_j) + \dots + f_C(a_1)) \text{.} \]
\end{proof}

\begin{cor}\label{2_3_6}
  Let $C$ be a convex, symmetric star body, and let $\Lambda$ be a lattice in $\R^n$ with successive minima $\lambda_1,\dots,\lambda_n$ with respect to $C$.
  Then there exists a basis $b_1,\dots,b_n$ of $\Lambda$ with
  \[ f_C(b_j) \leq \max \{ \lambda_j, \frac{1}{2}(\lambda_1 + \dots + \lambda_n \} \]
\end{cor}

\begin{proof}
  Immediate form Lemmas \ref{l_2_3_3} and \ref{l_2_3_5}.
\end{proof}


\subsection{4. Minkowski's Second Theorem}

Minkowski's Second Theorem is a refinement of his First Theorem and a central result in Geometry of Numbers.
Let's start by rephrasing Minkowski 1.%\roman{1}

First note if $C \subset \R^n$ is convex, symmetric and of positive volume then there exist $\varepsilon > 0$, $x \in C$ such that
\[ B_\varepsilon(x) \subset C \text{.} \]
But then there exists $\varepsilon^\prime > 0$ such that $B_{\varepsilon^\prime}(0) \subset C$.
So the origin lies in the interior of $C$.
Let $\Lambda$ be a lattice in $\R^n$.
So we can consider the successive minima $\lambda_1,\dots,\lambda_n$ of $\Lambda$ with respect to $C$.

Note that by definition of $\lambda_1$:
\[ \forall \varepsilon > 0: \; (\lambda_1 - \varepsilon) C \text{ contains \underline{no} non-zero lattice point.} \]

Minkowski's First Theorem yields:
\begin{align}
  \lambda_1^n \cdot \vol(C) = \vol(\lambda_1 C) \leq 2^n \cdot \det \Lambda \label{4_1}
\end{align}
On the other hand (\ref{4_1}) and $\vol C > 2^n \det \Lambda$ implies $\lambda_1 < 1$ and hence $C$ contains a non-zero lattice point.
The following theorem is much more precise than (\ref{4_1})!

\begin{theorem}[2.4.1 Minkowski's Second Theorem\label{th_2_4_1_minkovski2}]
  Let $C$ be a convex, symmetric star body in $\R^n$, and let $\Lambda$ be a lattice in $\R^n$ with successive minima $\lambda_1,\dots,\lambda_n$ with respect to $C$. Then
  \[ \frac{2^n}{n!} \det \Lambda \leq \lambda_1 \dots \lambda_n \cdot \vol C \leq 2^n \det \Lambda \]
\end{theorem}

\begin{rem}
  \begin{itemize}
    \item Both bounds are sharp. For the upper bound take $\Lambda = \Z^n$ and $C = [-1,1]^n$; 
    so $\lambda_1 = \dots = \lambda_n = 1 = \det \Lambda$, and $\vol C = 2^n$.
    For the lower bound take $\Lambda = \Z^n$ and $C$ defined by $\abs{x_1} + \dots + \abs{x_n} \leq 1$.
    Then $\vol C = \frac{2^n}{n!}$ and $\lambda_1 =  \dots = \lambda_n = 1 = \det \Lambda$.
    \item The upper bound is much harder to prove.
    We will prove Theorem \ref{th_2_4_1_minkovski2} only for the ball $C = B_1(0)$.
  \end{itemize}
\end{rem}

\begin{proof}
  (Special case $C = B_1(0)$)\\
  Put 
  \[ \delta(C) = \sup_M \frac{\lambda_1^n(M,C)}{\det M}\] 
  where the supremum runs over all lattices $M$ in $\R^n$.
  By Minkowski 1 (cf (\ref{4_1})).
  We have 
  \[ \delta(C) \leq \frac{2^n}{\vol C} \text{.} \]
  We will show that if $C = B_1(0)$ then
  \begin{align}
    \det \Lambda \leq \lambda_1 \dots \lambda_n \leq \delta(C) \det \Lambda \text{.} \label{4_2}
  \end{align}
  In particular, as $\vol(C) \geq \frac{2^n}{n!}$, we get 
  \[ \frac{2^n}{n!} \lambda_1 \dots \lambda_n \leq \vol C \lambda_1 \dots \lambda_n \leq 2^n \det \Lambda \text{.} \]
  For the lower bound in (\ref{4_2}) take linearly independent $a_1,\dots,a_n \in \Lambda$ with $\abs{a_i} = \lambda_i$ using Lemma \ref{l_2_3_3}. 
  For the sublattice $\Lambda^\prime = (a_1 \dots a_n) \Z^n \subset \Lambda$ we have $\det \Lambda^\prime = I \det \Lambda$,
  where $I$ is the index of $\Lambda^\prime$ in $\Lambda$.
  By Hadamard's inequality:
  \[ \abs{a_1} \dots \abs{a_n} \geq \det \Lambda^\prime \geq \det \Lambda \text{.} \]
  Thus
  \[ \det \Lambda \leq \lambda_1 \dots \lambda_n \text{.} \]
  Now we prove the upper bound in (\ref{4_2}).
  Let $b_1,\dots,b_n$ be a basis of $\Lambda$ as in Corollary \ref{cor_2_3_4}.
  As in the prove of Lemma \ref{l_2_3_2} we can find mutually orthogonal vectors $c_1,\dots,c_n$ such that
  \[ b_j = t_{j1} c_1 + \dots + t_{jj} c_j \; (t_{ji} \in \R) \]
  By scaling we can assume $\abs{c_j}^2 = 1$ $(1 \leq j \leq n)$.
  Now
  \[ \sum_{j=1}^n u_j b_j = \sum_{i=1}^n \left( \sum_{j \geq i} u_j t_{ji} \right) c_i \]
  thus
  \begin{align}
    \abs{\sum_{j=1}^n u_j b_j}^2 = \sum_{i=1}^n \left( \sum_{j \geq i} u_j t_{ji} \right)^2 \label{4_3}
  \end{align}
  Next we show that
  \begin{align}
    \sum_{i=1}^n \lambda_i^{-2} \left( \sum_{j \geq i} u_j t_{ji} \right)^2 \geq 1 \label{4_4}
  \end{align}
  where $u = (u_1,\dots,u_n) \in \Z^n \setminus 0$.
  Let $u \in \Z^n \setminus 0$, 
  \begin{align}
    u_J \neq 0 \text{ and } u_j = 0 \text{ for } j > J \text{.} \label{4_5}
  \end{align}
  Then $u_1 b_1 + \dots + u_n b_n, b_1,\dots b_{J-1}$ are linearly independent and by Corollary \ref{cor_2_3_4} we have 
  \begin{align}
    \abs{\sum_{j=1}^n u_j b_j}^2 \geq \lambda_J^2 \label{4_6}
  \end{align}
  Moreover, (\ref{4_5}) implies that summands with $j > J$ in (\ref{4_3}) and (\ref{4_4}) are zero.
  Thus, the left hand-side in (\ref{4_4}) is equal to 
  \begin{align}
    \sum_{i \leq J} \lambda_i^{-2} \left( \sum_{i \geq j} u_j t_{ji} \right)^2 \geq \sum_{i \leq J} \lambda_J^{-2} \left( \sum_{i \geq j} u_j t_{ji} \right)^2 = \lambda_J^{-2} \abs{\sum_{j=1}^n u_j b_j}^2 \underbrace{\geq}_{\text{by (\ref{4_6}) and (\ref{4_3})}} 1
  \end{align}
  So if $\Lambda^\prime$ is the lattice with basis 
  \begin{align}
    b_j^\prime = t_{j1} \lambda_1^{-1} c_1 + \dots + t_{jj} \lambda_j^{-1} c_j \; (1 \leq j \leq n)
  \end{align}
  Then
  \[ \abs{\sum_{j=1}^n u_j b_j^\prime} \geq 1\]
  for every point $\sum u_j b_j^\prime \in \Lambda^\prime \setminus 0$.
  Hence, 
  \begin{align}
    \lambda_1(\Lambda^\prime,C) \geq 1 \label{4_7}
  \end{align}
  But 
  \begin{align}
    \det \Lambda^\prime = \lambda_1det \Lambda^\prime = {-1} \dots \lambda_ndet \Lambda^\prime = {-1} \det \Lambda \label{4_8}
  \end{align}
  $\lambda_i = \lambda_i(\Lambda,C)$\\
  Moreover, by definition
  \begin{align}
    \frac{\lambda_1^n(\Lambda^\prime,C)}{\det \Lambda^\prime} \leq \sup_M \frac{\lambda_1^n(M,C)}{\det M} = \delta(C) \label{4_9}
  \end{align}
  Combining (\ref{4_7}), (\ref{4_8}) and (\ref{4_9}) we conclude
  \begin{align}
    \lambda_1 \dots \lambda_n = \det \Lambda \frac{1}{\det \Lambda^\prime} \leq \det \Lambda \frac{\lambda_1^n(\Lambda^\prime,C)}{\det \Lambda^\prime} \leq \det \Lambda \cdot \delta(C)
  \end{align}
\end{proof}

\end{document}
