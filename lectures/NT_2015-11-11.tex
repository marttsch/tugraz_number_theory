\documentclass[NumTh.tex]{subfiles}
\begin{document}

\subsection{6. Further generalizations and open problems}

Let $\alpha, \beta \in \R \setminus \Q$ and consider the linearly independent  linear forms $L_1{\underbar{x}} = x_0 \alpha -x_1$, $L_2(\underbar{x}) = x_0 \beta - x_2$, $L_3(\underbar{x}) = x_0$
If $\alpha, \beta$ are algebraic then Theorem 1.5.1 implies that the solutions $\underbar{x} \in \Z^3 \setminus \underbar{0}$ of
\[ \abs{L_1(\underbar{x}) \cdot L_2(\underbar{x}) \cdot L_3(\underbar{x})} < \norm{\underbar{x}}^{-\delta} \; (\delta > 0) \]
lie in finitely many proper subspaces of $\Q^3$.

However, in the following is a long-standing conjecture.

\begin{conj}[Littlewood-conjecture, around 1920]
  Let $\alpha,\beta \in \R \setminus \Q$ and $\varepsilon > 0$. Then $\exists \underbar{x} \in \Z^3$ such that
  \[ 0 < \abs{L_1(\underbar{x}) L_2(\underbar{x}) L_3(\underbar{x})} < \varepsilon \]
\end{conj}

\begin{rem}
  Note that the conjecture is obviously true if $\alpha = \beta$ (by Corollary 1.1.2) or if $\alpha$ or $\beta$ are \underline{not} badly approximable.
\end{rem}

Let's consider again approximations to \underline{one} real $\alpha$.
So far our approximations were $\frac{p}{q} \in \Q$.
If we replace $\Q$ ba a smaller or larger set then we get now interesting problems.

\begin{op}[1.6.2]
  Let $\alpha \in \R \setminus \Q$ and $\lambda < 2$. Does $\abs{\alpha - \frac{p}{q}} < q^{-\lambda}$ have $\infty$-many solutions $(p,q) \in \Z \times \N$ with:
  \begin{itemize}
    \item $p$ and $q$ are both square-free.\\
    Best-result (Hoath-Brown 1984): Yes, if $\lambda < \frac{5}{3}$.
    \item $q$ is prime?\\
    Best-restult (Matomaki, 2009): Yes, if $\lambda < \frac{4}{3}$.
  \end{itemize}
\end{op}

Let's now consider problems in which $\Q$ is replaced by a certain subset $A$ of
\[ \overbar{\Q} = \{\alpha \in \C : \alpha \text{ algebraic} \} \text{.} \]

If we assume that $A \subset \R$  then we still can use the (usual) absolute value on $\R$ to measure
\[ \abs{\alpha -x} \; (x \in A) \text{.}\]
But usually we have no "natural denominators" for $x \in A$.
But there is a natural way to interpret the original setting that easily generalizes from $A = \Q$ to $A=\overbar{\Q} \cap \R$.
To this end we introduce the so-called \underline{multiplicative absolute Weil height}:
\[ H: \Q \to \left[1,\infty\right) \]
defined by
\[ H(\alpha) = M(D_\alpha(x)) \]
where $D_\alpha(x) = a_0 (x - \alpha_1)\dots (x - \alpha_d) \in \Z[x]$ is the minimal polynomial of $\alpha$ and
\[ M(D_\alpha(x)) = \abs{a_0} \cdots \prod_{i=1}^n \max \{ 1, \abs{\alpha_i} \} \]
$M$ is called the \underline{Mahler-measure}.

\begin{ex}
  \begin{itemize}
    \item $\alpha = \frac{p}{q} \in \Q$ ($q > 0$, $\gcd(p,q)=1$).\\
    $\deg(\alpha) = 1$, $D_\alpha = q x - p$. So $H(\alpha) = M(D_\alpha) = q \max \{ 1, \abs{\frac{p}{q}} \}
    = \max \{ q, \abs{p} \} = H_{\p^1} ((1,\alpha))$.
    \item $\alpha = 2^{\frac{1}{d}}$, $D_\alpha = x^d -2$ ($2$-Eisenstein), $\deg \alpha = d$,
    $H(\alpha) = M(D_\alpha)^{\frac{1}{d}} = \prod_{i=1}^d \max \{1, \abs{\xi_d^{i-1} 2^{\frac{1}{d}}} \} = 2^{\frac{1}{d}}$
  \end{itemize}
\end{ex}

One can easily show (cf. sheet 4) that 
\begin{align}
  \# \{ \alpha \in \overbar{\Q} : \deg d \leq d, H(\alpha) \leq X \} < \infty \; \forall d \in \N \; X \geq 1\text{.} 
  \label{6_1}
\end{align}

Back to Diophantine approximation with $A = \Q$.
As 
\[ a + m - \frac{p}{q} = a - \left( \frac{p - mq}{q} \right) \]
we can assume $\alpha \in (0,1)$
So all good enough approxmations $\frac{p}{q}$ lie also in $(0,1)$. Now if $\frac{p}{q} \in (0,1)$ then 
\[ H \left( \frac{p}{q} \right) = q \text{.} \]
So
\[ \abs{\alpha - \frac{p}{q}} < \phi(q) \iff \abs{\alpha - \frac{p}{q}} < \phi\left( H \left( \frac{p}{q} \right) \right) \text{.} \]
So now the denominator plays \underline{no} role anymore and we can write more easily:
\begin{align}
  \abs{\alpha - x} < \phi(H(x))
  \label{6_2}
\end{align} 
(\ref{6_2}) makes sense as long as $x - \alpha \in \R$, so $x \in \R$, and $x \in \overbar{\Q}$.
So let's assume $A \subset \overbar{\Q} \cap \R$.
However, if $x_1,x_2,x_3,\dots$ is a sequence of pairwise distinct solutions of (\ref{6_2}) then we want to conclude that
$x_i \to \alpha$ (with respect to $\abs{\bullet}$).
Now as $\phi(t) \to 0$ as $t \to \infty$ but we don't know a priori that $H(x_i) \to \infty$.
So cannot conlude from (\ref{6_2}) that $x_i \to \alpha$.
But if $A \subset \Q_{(d)} = \{ \alpha \in \overbar{\Q} : \deg \alpha \leq d \}$
then (\ref{6_1}) rells us that $H(x_i) \to \infty$ and so $x_i \to \alpha$.

More generally this is true if 
\[ \# \{ \alpha \in A : H(\alpha) \leq X \} < \infty \; \forall X \geq 1\text{.}\]
In this case we say $A$ has \underline{Propery $N$}.

\begin{theorem}[1.6.3 Wirsing 1961 \label{th_1_6_3}]
  Let $d \in \N$, $d > 1$ and $\alpha \in \R \setminus \Q_{(d)}$.
  Then $\exists \infty$-many $x \in \Q_{(d)}$ with
  \[ \abs{\alpha - x} < H(x)^{-(\frac{d+3}{2})} \]
\end{theorem}

\begin{conj}[1.6.4 Wirsing's Conjecture, around 1961 \label{conj_1_6_4}]
  Suppose $\alpha \in \R \setminus \Q_{(d)}$, ($d\in \N$) and $\lambda < d+1$ then 
  $\exists \infty$-many $x \in \Q_{(d)}$ with
  \[ \abs{\alpha - x} < H(x)^{-\lambda} \text{.} \]
\end{conj}

\begin{theorem}[1.6.5 Davenport and Schmidt \label{th_1_6_5}]
  Wirsing's conjecture (\ref{conj_1_6_4}) holds for $d \leq 2$.
\end{theorem}

Instead of taking $A = \Q_{(d)} \cap \R$ let's replace $\Q_{(d)}$ with the smalles field that contains $\Q_{(d)}$;
let's call this field $\Q^{(d)}$.
Unfortunately, nobody knows whether $\Q^{(d)}$ has Propery ($N$), except when $d \leq 2$.

\begin{theorem}[1.6.6 Bombien-Zannier, 2001 \label{th_1_6_6}]
  $\Q^{(2)}$ has Property ($N$).
\end{theorem}

\begin{op}[1.6.7]
  Find an analog of Corollary 1.1.2 for $A = \Q{(2)} \cap \R$.
  How quickly can $\phi : \left[1,\infty\right) \to \left(0,\infty\right)$ decay if for every $\alpha \in \R \setminus A$
  \[ \abs{ \alpha - x } < \phi(H(x)) \]
  has $\infty$-many solutions $x \in A$? 
  It is not difficult to show an inequality in the other direction provided $\alpha$ is algebraic, 
  e.g., if $\alpha \in \overbar{\Q}\setminus \Q^{(2)}$ then
  \[ \abs{\alpha -x } > (2 \cdot H(\alpha) H(x))^{-\deg \alpha \cdot 2^{(2 H(x))}} \]
  How much can this be improved?
\end{op}


\end{document}
