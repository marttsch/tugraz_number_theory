\documentclass[NumTh.tex]{subfiles}
\begin{document}

Roth's Theorem has various new applications to, e.g., Diophantine equations and transcendence.
Let's consider just one now trancendence result:\\
Take $ \alpha = \sum_{k=1}^\infty 2^{-3^k}$; put $q_n = 2^{3^n}$ and
$p_n = q_n \sum_{k=1}^n 2^{-3^k}$.
Then $0 < \abs{\alpha - \frac{p_n}{q_n}} = \sum_{k=n+1}^\infty 2^{-3^k} < 2 \cdot 2^{-3^{n+1}} = 2 \cdot 2 \cdot q_n^{-1}$
so by Roth's Theorem $\alpha$ is transcendental.\\
\\
How does one prove results like Roth's Theorem of the kind
\[ \abs{\alpha - \frac{p}{q}} \geq \phi(q) \text{?} \]

The idea is to find good rational approximations.
\[ \abs{\alpha - \frac{p_n}{q_n}} \leq \delta_n \]
with $\delta_n$ "pretty small". Then
\[ \abs{\alpha - \frac{p}{q}} \geq \abs{\frac{p_n}{q_n} - \frac{p}{q}} - \abs{ \alpha - \frac{p_n}{q_n}} \]
If 
\begin{align}
  \frac{p_n}{q_n} \neq \frac{p}{q} %ref *
\end{align}
then 
\[ \abs{\alpha - \frac{p}{q}} \geq \frac{1}{q q_n} - \delta_n. \]
If we are lucky then $\delta_n < \frac{1}{q q_n}$ and we get a positive lower bound. How do we find these $\frac{p_n}{q_n}$?\\
Usually this is a difficult task, but sometimes one can easily see these approximations $\frac{p_n}{q_n}$.
Here is an example.\\
Take again $\alpha = \sum_{k=1}^\infty 2^{-3^k}$. Then we can take again $q_n = 2^{3^n}$, $p_n = q_n \sum_{k=1}^n 2^{-3^k}$;
so $\abs{\alpha -  \frac{p_n}{q_n}} < 2 \cdot q_n^{-3}$. Hence, if
\[ \frac{p_n}{q_n} \neq \frac{p}{q} \]
then
\[ \abs{\alpha - \frac{p}{q}} \geq \frac{1}{q q_n} - \frac{2}{q_n^3}\]
If $q_n^2 > 4 \cdot q$ then
\[ \frac{1}{q q_n} - \frac{2}{q_n^3} \geq \frac{q}{2\cdot q q_n} \]
As $\frac{p_n}{q_n}$ tends strictly monotonously to $\alpha$, we have $\frac{p_n}{q_n} \neq \frac{p}{q}$ or $\frac{p_{n+1}}{q_{n+1}} \neq \frac{p}{q}$
Let $m$ be minimal with $q_m > 4 \cdot q$. Hence
\[ q_m^{\frac{2}{3}} = q_{m-1}^2 \leq 4 \cdot q < q_m^2 \]
If $\frac{p_m}{q_m} \neq \frac{p}{q}$ we take $n = m$ and $n = m+1$ else.
We conclude
\[ \abs{\alpha - \frac{p}{q}} \geq \frac{1}{2 q q_n} \geq \frac{1}{2 q q_{m+1}} \geq \frac{1}{2 q} \frac{1}{q_m^3} \geq \frac{1}{2q} \frac{1}{(4q)^{\frac{9}{2}}} = 2^{-10}q^{-\frac{11}{2}} \]
In this example everything works out nicely, e.g., (ref*) could easily be guaranteed by using $\frac{p_n}{q_n}$ tending strictly monotonously to $\alpha$.
However, in Roth's Theorem (ref*) becomes the major-problem.

\subsection{5 Simultaneous Diophantine approximation and the Subset Theorem}

Suppose $\alpha_1,\dots,\alpha_n$ are real numbers. Theorem 1.1.1 can be generated to yield a solution $(x_1,\dots,x_n,y) \in \Z^n \times \N$ at the system
\[ \abs{\frac{x_i}{y} - \alpha} \leq \frac{1}{y \cdot Q} (1\leq i\leq n), 0 < y < Q . \]
(c.f. Exercise sheet 4).
This in turn yields $\infty$-many solutions $(x_1,\dots,x_n,y) \in \Z^n \times \N$ of the system
\[ \abs{\frac{x_i}{y} - \alpha_i} < \frac{1}{y^{1+\frac{1}{n}}} (1 \leq i \leq n). \]
provided at least one of the $\alpha_i$'s is irrational. So Corollary 1.1.2 extends to simultaneous approximation.
A much deeper fact is that Roth' Theorem also extends to simultanious approximation.

%convention x _ underbar
For $\ubar{x} \in \R^n$ we write $\norm{\ubar{x}} = ( \sum_{i = 1}^n x_i^2)^{\frac{1}{2}}$ for the Euclidean length.

\begin{theorem}[Subspace Theorem, Schmidt]
  Suppose $L_i(\ubar{x}) = \sum_{j=1}^n a_{ij} x_j (1 \leq i \leq n)$ are linearly independent
  linear forms with algebraic coefficients $a_{ij}$.
  Let $\delta > 0$. Then the solutions $\ubar{x} \in \Z^n \setminus \ubar{0}$ of
  \[ \abs{L_1(\ubar{x}) \dots L_n(\ubar{x})} < \norm{\ubar{x}}^{-\delta} \]
  lie in finitely many proper subspaces of $\Q^n$.
\end{theorem}

\begin{rem}
  linearly independent linear forms means the coefficient vectors $(a_{i1},\dots,a_{in})$ are linearly independent over $\C$.
\end{rem}

\begin{cor}[1.5.2]
  Let $\delta >0$, suppose $\alpha_1,\dots, \alpha_n$ are algebraic and $1,\alpha,\dots,\alpha_n$ are linearly independentn over
  $\Q$. Then there are only finitely many $(x_1,\dots,x_n,y) \in \Z^n \times \N$ with
  \begin{align}
    (5.1) \abs{\frac{x_i}{y} - \alpha_i} < \frac{1}{y^{1+\frac{1}{n} + \delta}} (1 \leq i \leq n)
  \end{align}
\end{cor}

\begin{proof}(assuming Theorem 1.5.1)
  Put $\ubar{X} = (X_1,\dots,X_n,Y)$, $L_i(\ubar{X}) = \alpha_i Y - X_i (1 \leq i \leq n)$, $L_n(\ubar{X}) = Y$.
  These $n+1$ linear forms in $n+1$ unknowns are linarly independet.
  With $\ubar{x} = (x_1,\dots,x_n,y)$ the solutions of (5.1) yield
  \[ \abs{L_1(\ubar{x}) \dots L_{n+1}(\ubar{x})} < \frac{1}{y^\delta} < \frac{1}{\norm{\ubar{x}}^{\frac{\delta}{2}}} \]
  if $y$ is large enough. so by Theorem 1.5.1 (in $n+1$ dimensions), we set that the solutions lie in finitely many prober subspeced at $\Q^{n+1}$.
  Pick one of these (of codimension I say).
  It is giben by an equation $c_1x_1+\dots+c_nx_n+c_{n+1}y = 0$ where $c_i \in \Q$ not all zero.
  On this subspace we have
  \[ (c_1\alpha_1+\dots+c_n\alpha_n+c_{n+1})y = c_1(\alpha_1y - x_1) +\dots+ c_n(\alpha_ny - x_n).\]
  Put $\gamma = c_1 \alpha_1+\dots+c_n\alpha_n + c_{n+1}$.
  By $\Q$-linearly independence of $1,\alpha_1,\dots,\alpha_n$ we have $\gamma \neq 0$. Hence,
  \[ \abs{\gamma} \abs{y} \leq \abs{c_1}\abs{\alpha_1y-x_1} +\dots+ \abs{c_n}\abs{\alpha_ny-x_n} \leq (\abs{c_1}+\dots+\abs{c_n})\frac{1}{y^{1+\frac{1}{n}+\delta}} \leq \abs{c_1}+\dots+\abs{c_n} \]
  So $\abs{y}$ is bounded and we are done.
\end{proof}

In applications one sometimes needs a "$p$-adic" version of the subspace Theorem in which one approximates with respect to also the
so called $p$-adic absolute values.


\begin{defi*}[Absolute values]
  An absolute value on a field $K$ is a map $\abs{\bullet}: K \to [0,\infty)]$ such that
  \begin{itemize}
    \item $\abs{x} = 0 \iff x = 0$
    \item $\abs{x \cdot y} = \abs{x}\cdot\abs{y}$
    \item $\abs{x+y} \leq \abs{x} + \abs{y}$
  \end{itemize}
\end{defi*}

\begin{ex*}
  \begin{itemize}
    \item $K$ arbitrary. $\abs{x} = \begin{cases} 0 & x = 0\\ 1 & x \neq 0 \end{cases}$
    the \emph{trivial absolute value}.
    \item $K = \Q$, $\abs{\bullet} =$ \emph{standatd absolute value} on $\Q$.
    To distinguish it from other absolute values let's write it as $\abs{\bullet} = \abs{\bullet}_\infty$.
    \item $K = \Q$ and let $p \in \N$ be a prime number. If $x\in \Q, x \neq 0, \pm 1$, then $\exists$ a unique prime factoriation
    $ x = \pm p_1^{a_1} \dots p_s^{a_s}$ where $p_1,\dots,p_s$ primes and $a_i \in \Z \setminus 0$.
    For any prime $p \in \N$ write $ord_p(x)$ for the exponent of $p$ in the primfractorisation of $x$ (e.g. $ord_{p_i}x = a_i$).
    For $x = \pm 1$ we put $ord_p x = 0 \forall p_i$.
    The \emph{$p$-adic absolute vlaue} $1 \cdot 1_p$ on $\Q$ is defined by 
    \[ \abs{x}_p = \begin{cases} 0 &: x = 0\\ p^{-ord_p(x)} &: x \neq 0 \end{cases} \]
    The multiplicativity is clear.
    Note that $ord_p(x_1+x_2) \geq \min\{ord_p(x_1),ord_p(x_2)\}$.
    Hence, $\abs{x_1 + x_2}_p = p^{-ord_p(x_1+x_2)} \leq p^{-\min\{ord_p(x_1),ord_p(x_2)\}} \underbrace{=}{strong triangle inequality} \max{\abs{x_1}_p, \abs{x_2}_p\}} \leq \abs{x_1}_p + \abs{x_2}_p$
    An absolute value that satisfies the srtong triange inequality is called non-Archimedean.
  \end{itemize}
\end{ex*}

\begin{defi}
  We set $M_\Q = \{ \text{primes in } \N\} \cup \{ \infty\}$. Then for each $v \in M_\Q$ we get an absolute value
  $\abs{\cdot}_v$. Note that if $v \in M_\Q$ and $p$ a prime, $a \in \Z$, then
  \[ \abs{\pm p^a}_v = \begin {cases} p &: v =p \\ p^a &: v = \infty\\ 1 &: v \neq p, v \neq \infty \end{cases} \]
  Hence
  \[ \prod_{v \in M_\Q} \abs{1 \pm p^a}_v = 1 \]
  and so vy multiplicativity we conclude
  \[ \prod_{v \in M_\Q} \abs{x}_v = 1 \]
  for all $x \in \Q$, $x \neq 0$. (PF)
  Thsi is the so-called producct formula (PF) on $\Q$.
\end{defi}


\end{document}
