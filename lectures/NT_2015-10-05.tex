\documentclass[NumTh.tex]{subfiles}
\begin{document}

\section*{Organizatorial stuff}

Dates (in TUGrazOnline):
\begin{table}[!h]
  \begin{tabular}{cccl}
    Mon & 14:15--15:45 & C208 & Exercises (starting 19.10. first exercise class) \\
    Tue & 14:15--15:45 & C307 & Lecture (starting 20.10. first (real) lecture) \\
    Wed & 08:15--09:45 & C208 & Lecture
  \end{tabular}
\end{table}

From now until 15.12. lectures by Martin Widmer.
Then C. Frei.

End: oral exams

Exercises: Find details on website of the instructor Dijana Kreso.
\url{math.tugraz.at/~kreso}

\section{Basics}

\begin{align}
  \N &= \{1,2,\dots\} \\
  \N_0 &= \N \cup \{0\}\\
  \Z &= \{ \dots,-2,-1,0,1,2,\dots\}
\end{align}

\subsection{Divisibility}
\begin{defi}
Let $a,b \in \Z$. $a$ divides $b$ (written $a\divides b$) if $\exists q \in \Z : b = qa$.
\newline
Some properties: Let $a,b,c \in \Z$. Then the following statements hold:
\begin{align}
  a\divides b &\Rightarrow ac\divides bc \\
  a\divides b \land b\divides c &\Rightarrow a\divides c \\
  a\divides b \land b\divides a &\Leftrightarrow a = b \\
  a\divides b \land a\divides c &\Rightarrow a\divides (b+c)
\end{align}
\end{defi}

\begin{defi}[Remainder]
Let $a \in \Z$, $b \in \N$. Then there are unique $q,r \in \Z$ such that
\[a = qb + r \text{ and } 0 \leq r < b.\]
\end{defi}

\begin{rem}
\begin{enumerate}
  \item $b\divides a \Leftrightarrow r = 0$
  \item $q = \lfloor \frac{a}{b} \rfloor$ (largest integer $\leq \frac{q}{b}$)
  \item we will somtimes write: $a \mod b \coloneq c$
\end{enumerate}
\end{rem}

\begin{defi}
Let $a_1,a_2,\dots,a_n,d \in \mathbb{Z}$. $d$ is a greatest common divisor (gcd) of $a_1,\dots,a_n$ if
$d\divides a_i$ $\forall 1\leq i \leq n$,
and for every $e \in \Z$ with $e\divides a_i$ $\forall 1\leq i \leq n$, $e\divides d$.
\end{defi}

\begin{rem}

\begin{enumerate}
  \item a $\gcd$ of $a_1,\dots,a_n$ is unique up to sign
  \item we write $d = \gcd(a_1,\dots,a_n)$ if $d$ is a $\gcd$ of $a_1,\dots,a_n$
  \item for $a_1,\dots,a_n \in \Z$, a $\gcd$ exists and can be written as a linear combination of $a_1,\dots,a_n$,
  i.e., $\exists x_1,\dots,x_n \in \Z$ such that $$\gcd(a_1,\dots,a_n) = x_1 a_1+ \dots + x_n a_n$$
  \item $\gcd(a_1,\dots,a_n) = \gcd(\gcd(a_1,\dots,a_{n-1}),a_n)$
  \item if $a\divides bc$ and $\gcd(a,b) = 1$ then $a\divides c$.
  \item let $a^\prime \coloneq \frac{a}{\gcd(a,b)}$, $b^\prime = \frac{b}{\gcd(a,b)}$. Then $\gcd(a^\prime,b^\prime) = 1$
\end{enumerate}
\end{rem}
\todo{Hier verwendest du $\coloneq $, sonst aber nur $=$, evtl. einheitlich machen für alle Definitionen?}

\begin{algorithm}
  \caption{Compute the $\gcd$ of two integers: Euclidean algorithm}
  \begin{algorithmic}
    \Require{$a,b \in \Z$. $|a| \geq |b|$}
    \Ensure{$a \coloneq \gcd(a,b)$}
    \State replace $a$ by $|a|$, $b$ by $|b|$
    \While{$b \neq 0$}
      \State write $a = qb +r$, $0 \leq r < b$
      \State $a \coloneq b$
      \State $b \coloneq r$
    \EndWhile
    \State \Return $a$
  \end{algorithmic}
\end{algorithm}

The algorithm is correct, since $\gcd(a,b) = \gcd(b,a \mod b)$. \\
The algorithm terminates because $b$ decreases in each step. \\
The algorithm is fast: ($\mathcal{O}(\log b)$)

\todo{sollte ausgebessert werden, 1. $O(logn)$ steps, 2. stimmt nur wenn $|r|\le b/2$}

The Euclidean algorithm also allows us to find $x, y$ such that $\gcd(a,b) = ax+by$ by doing all computations backwards.

\begin{ex}
  $\gcd(56,22) = \,?$
  \begin{align*}
    a &= 56, b = 22\\
    56 &= 2 \cdot 22 + 12\\
    a &= 22, b = 12 \neq 0\\
    22 &= 1 \cdot 12 + 10\\
    a &= 12, b = 10 \neq 0\\
    12 &= 1 \cdot 10 + 2\\
    a &= 10, b = 2 \neq 0\\
    10 &= 5 \cdot 2 + 0\\
    a &= 2, b = 0 & \Rightarrow \gcd(56,22) = 2
  \end{align*}
  Doing the computations backwards:
  \begin{align*}
    2 &= 12 -10 = 12 - (22 - 12) = -22 + 2 \cdot 12 = -22 + 2(56-2 \cdot 22) = 2 \cdot 56 - 5 \cdot 22\\
    x &= 2, y = -5
  \end{align*}
\end{ex}

\begin{app*}[linear diophantine equations]
Let $a,b,c \in \Z$, $a,b,c \neq 0$. Find all $(x,y) \in \Z^2$ which satisfy
\begin{align}
  \label{eq:lde}
  ax+by = c.
\end{align}
\end{app*}

\begin{description}
  \item[Existence of solution] let $d = \gcd(a,b)$.
    \[
	    (d\divides a \Rightarrow d\divides xa) \land
	    (d\divides b \Rightarrow d\divides yb)
	\] \[
	    \Rightarrow d\divides xa+yb = c \\
	\] \[
	    \Rightarrow \cref{eq:lde}
	\]
	can have solutions only if $d\divides c$.

  \item[Solution in case $d=1$]
    Let $x_0,y_0 \in \Z$ such that $ax_0+by_0 = 1$ using the Euclidean algorithm.
    Then from $acx_0 + bcy_0 = c$ the solution $(cx_0, cy_0)$ of (\cref{eq:lde})
    follows: for all $n \in \Z: (x,y) \coloneq (cx_0+nb,cy_0+na)$ is a solution.
  
	Indeed,
	\begin{align*}
	  ax+by &= acx_0 + anb + bcy_0 - bna = c &\checkmark
	\end{align*}
    These $(x,y)$ are all solutions: let $(x,y)$ be a solution. Then
	\begin{align*}
	   ax + by &= c \\
	   acx_0 + bcy_0 &= c \\
	   \Rightarrow a(x-cx_0) &= b(cy_0-y)
	\end{align*}
	\[ \gcd(a,b) = 1 \:\Rightarrow\: b \divides x - cx_0 \:\Rightarrow\: x = cx_0 + nb, n \in \Z \]
	\[ \Rightarrow\: a \divides cy_0 - y \:\Rightarrow\: y = cy_0 + ma, m\in \Z \]
    \[ c = ax+by = acx_0 + anb + bcy_0 + bma \]
    \[ = c + (n+m)ab \Rightarrow (n+m)ab = 0 \Rightarrow m = -n \]
  
  \item[Solutions in the general case] Assume $d = \gcd(a,b)$ and $d\divides c$, let
    \[
		a^\prime = \frac{a}{d} \hspace{25pt}
		b^\prime = \frac{b}{d} \hspace{25pt}
		c^\prime \coloneq \frac{c}{d}
	\]
    Then $\gcd(a^\prime, b^\prime) = 1$ and the solution to (\cref{eq:lde}) is exactly the solution of $a^\prime x + b^\prime y = c^\prime$.
\end{description}

\subsection{Primes}

\todo{1. Beistriche für bessere Lesbarkeit 2. faustregel, vor und nach ``i.e.'' gehört eigentlich beistrich}
\begin{defi}
$p \in \N$, $p > 1$ is a \emph{prime number} if the only positive divisors of $p$ are $1$ and $p$, i.e., $a \in \N$, $a\divides p \Rightarrow a \in \{1,p\}$.
$\PP \coloneq \{ primes\} \subset \N, \PP = \{2,3,5,7,11,13,\dots\}$.
$p$ prime and $p\divides ab \Rightarrow p\divides a$ or $p\divides b$
\end{defi}

\begin{theorem}[Fundamental theorem of arithmetic]
Every $n \in \N$ can be written uniquely (up to reordering) as a product of primes.
i.e. there are distinct primes $p_1,\dots,p_l$, and $\alpha_1,\dots,\alpha_l \in \N$ such that
$n = p_1^{\alpha_1} \dots p_l^{\alpha_l}$
\end{theorem}

\begin{proof}[Sketch]\hfill{}
  \begin{description}
    \item[Existence]
      let $p_0 > 1$ be the smallest divisor $> 1$ of $n$. Then $p_0$ is prime.
      $n = p_0 n_0$, induction $\checkmark$

    \item[Uniqueness]
      let $p_1 \dots p_m = q_1 \dots q_l = n$, $p_i, q_j$ primes.
      $p_1 \divides  q_1 \dots q_l \Rightarrow \exists i: p_1\divides q_i$, both prime $\Rightarrow p_1 = q_i$, wlog: $i = 1$.
      $p_1\dots p_m = q_1 \dots q_l$, induction $\checkmark$
  \end{description}
\end{proof}

\begin{theorem}[Euclid]
There are $\infty$-many primes.
\end{theorem}

\begin{proof}
  Given primes $p_1,\dots,p_n \in \PP$. We construct one more prime \[N \coloneq p_1 \cdot \dots \cdot p_n + 1 \text{.}\]
  Assume $P$ is a prime factor of $N$.
  If $P \in \{p_1, \dots ,p_n\}$ then $P\divides N$ and $P\divides p_1\dots p_n \Rightarrow P\divides 1$ $\lightning$
\end{proof}

\begin{rem}[prime factors and $\gcd$s]
Let $a_1,\dots, a_n \in \Z$, write
\[a_i = \prod_{p \in \PP} p^{\alpha_{p,i}}\text{, }\alpha_{p,i} \in \N_0 \text{,}\] almost all $a_i = 0$,
then \[\gcd(a_1,\dots,a_n) = \prod_{p \in \PP} p^{\min_{1\leq i\leq n} \{\alpha_{p,i}\}}\]
\end{rem}

\subsection{Congruences}

All rings are commutative with $1$.

\begin{defi}
Let $a,b \in \Z$, $n \in \N$. Then \emph{$a$ is congruent to $b \pmod{n}$}, $a \equiv b \pmod{n}$,
if $n\divides  a-b$.
We write $\overbar{a} = [a]_n \coloneq \{b \in \Z : b \equiv a (\mod n)\}$
\end{defi}

\begin{rem}
\label{four}
\begin{enumerate}
  \item Congruence $\bmod{n}$ is an \emph{equivalence relation}
  \item $\overbar{0},\overbar{1},\dots,\overline{n-1}$ is a partition of $\Z$.
  \item if $a \equiv b \pmod{n}$, $c \equiv d \pmod{n}$, then 
  $-a \equiv -b \pmod{n}$,
  $a \overset{+}{\underset{-}{\cdot}} d \pmod{n}$.
\end{enumerate}
\end{rem}

\begin{defi}
$\Z / n\Z = \Z_n \coloneq \{[a]_n : a\in \Z\} = \{\overbar{0},\overbar{1},\dots,\overline{n-1}\}$ residue class ring modulo $n$
\end{defi}

\begin{rem}
$\Z_n$ is a ring with operation $\overbar{a} \overset{+}{\underset{-}{\cdot}} \overbar{b} \coloneq \overline{a\overset{+}{\underset{-}{\cdot}}b}$ (well defined due to item~3 of~\cref{four})
$\Z_n^\times = \{ \overbar{a} \in \Z_n : \exists \overbar{b} \in \Z_n : \overbar{a}\overbar{b} = \overbar{1} \}$ ... group of units $\mod n$
\end{rem}

\begin{lemma}
Let $a \in \Z$. Then $\overbar{a} \in \Z_n^\times \Leftrightarrow \gcd(a,n) = 1$.
\end{lemma}

\begin{proof}\hfill
\begin{itemize}
  \item[``$\Rightarrow$'']
    $\overbar{a}\overbar{b} = \overbar{1} \Leftrightarrow ab \equiv 1 \pmod{n} \Leftrightarrow n\divides ab-1$ \\
    $\Rightarrow$ no prime factor of $n$ divides $a$ \\
    $\Rightarrow \gcd(a,n) =1$.
  \item[``$\Leftarrow$'']
    $1 = \gcd(a,n) = ax + ny \Rightarrow \overbar{1} = \overbar{a}\overbar{x}$
\end{itemize}
\end{proof}

\begin{rem}
The inverse of $\overbar{a}$ can be computed by the Euclidean algorithm.
\end{rem}

\begin{ex}[Simultaneous congruences]
Find $x \in \Z$ such that 
\begin{align}
x &\equiv 2 \pmod{3}\\
x &\equiv 1 \pmod{5}\\
x &\equiv 0 \pmod{7}
\end{align}
\end{ex}

\begin{theorem}[Chinese remainder theorem (CRT)] Let
\[ n_1,\dots,n_l \in \N \text{ subject to } \gcd (n_i,n_j) = 1 \;\forall i\neq j \]
\[ x_1, \dots, x_l \in \Z \text{.} \]
Then
\[ \exists x \in \Z \text{ such that } x \equiv x_i \pmod{n_i} \; \forall 1\leq i \leq l \]
where $x$ is unique modulo $n_1 \cdot \dots \cdot n_l$.
\end{theorem}

\end{document}
