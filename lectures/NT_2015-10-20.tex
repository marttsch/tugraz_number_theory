\documentclass[NumTh.tex]{subfiles}
\begin{document}

\section*{New Lecturer}
Chapter 1:
\begin{enumerate}
  \item Approximation to algebraic numbers; Wolfgang M. Schmidt, 1972 L'Ehseignement Mathématique
  \item Lectures Notes in Mathematics 785; W.M.Schmidt, Springer
  \item LNM 1467, W.M.S., Springer
  \item For section 2 (continued fractions) he will strictly follow the lecture notes of MT421 of Professor James McKee
\end{enumerate}

\section{Diophantine Approximation}

\subsection{Dirichlet's Theorem}

Let $\alpha \in \R$. As $\Q$ is dense in $\R$ any $\alpha \in \R$ can be approximated arbitrarily well, by rational numbers $p/q$ ($p \in \Z, q \in \N = \{1,2,3,\dots\}$).\\
The question is how well can we approximate $\alpha$ in terms of the denominator $q$, e.g., is it true that for every $\alpha \in \R$ there exists infinitely many $p/q \in \Q$ $q\in \N$) such that $| \alpha - \frac{p}{q} | < \frac{1}{q^2}$?\\
\\
The answer is no!\\
Take $\alpha = r/s (s \in \N)$ a rational number. Then 
\[ | \alpha - \frac{p}{q} | = |\frac{r}{s} - \frac{p}{q} | = | \frac{qr -ps}{sq} | \overset{\geq}{\text{provided } \alpha \neq \frac{p}{q}} \frac{1}{sq}  > \frac{1}{q^2} \text{ provided } q > s.\]
This shows that we have only finitely many solutions $p/q \in \Q$ for $| \alpha -\frac{p}{q} | < \frac{1}{q^2}$.

\begin{theorem}[Dirichlet's Theorem]
  Suppose $\alpha,Q \in \R$ and $Q > 1$. Then $\exists p,p \in \Z \text{s.t.} 0< q<Q$ and $| q\alpha - p | \leq \frac{1}{Q}$.
\end{theorem}

\begin{proof}
  for $\xi \in \R$ put $\{ \xi \} = \xi - \lfloor \xi \rfloor$. so $ 0 \leq \{ \xi\} \leq 1$. First suppose $Q \in \Z$.
  Consider the $Q + 1$ numbers $0,1,\{\alpha\},\{2\alpha\},\dots,\{(Q-1)\alpha\}$.\\
  They all lie in $[0,1]$. We split it up in $Q$ subintervals:
  \[ [0,1] =[0, \frac{1}{Q}] \cup \left[ \frac{1}{Q},\frac{2}{Q} \right] \cup \dots \cup \left[ \frac{Q-1}{Q},1 \right] \]
  By the pigeon hole principle two of the previous numbers lie in the same subinterval. Thus $\exists r_1,r_2,s_1,s_2 \in \Z$ with $0 \leq r_1 < r_2 \leq Q-1$ such that $| (r_1\alpha - s_1) - (r_2 \alpha -s_2) | \leq \frac{1}{Q}$.
  Then with $q = r_2 - r_1$ and $p = s_2 -s_1$ we get $|q\alpha - p| \leq \frac{1}{Q}$ and $0 < q < Q$.
  This proves the Theorem when $Q \in \Z$. Now suppose $Q \nin \Z$. 
  We apply the previous with $Q^\prime = \lfloor Q \rfloor + 1 > 1$.
  Hence, $\exists p,q \in \Z$ with $| q\alpha - p | \leq \frac{1}{Q^\prime}$ and $0 < q < Q^\prime$, and so $|q\alpha - p | \leq \frac{1}{Q}$ and $0 < q < Q$.
\end{proof}

\begin{cor}
  Suppose $\alpha \in \R/\Q$. Then there exist infinitely many solutions $p/q \in \Q$ ($q \in \N$) of $| \alpha - \frac{p}{q}| < \frac{1}{q^2}$.
\end{cor}

\begin{proof}
  Take $Q_1 > 1$. By Theorem 1.1.1 we get $(p_1, q_1) \in \Z^2$ with $0<q_1 < Q$, and $|q_1 \alpha - p_1 | \leq \frac{1}{Q_1}$.
  Thus $|\alpha - \frac{p_1}{q_1} | \leq \frac{1}{q_1 Q_1} < \frac{1}{q_1^2}$
  \\
  Next take $Q_2 = |\alpha - \frac{p_1}{q_1}|^{-1} + 1$. Then Theorem 1.1.1 again yields $\frac{p_2}{q_2} \in \Q$ with $|\alpha - \frac{p_2}{q_2} | < \frac{1}{q^2}$ and $|\alpha - \frac{p_2}{q_2} | \leq \frac{1}{q_r Q_2} \leq \frac{1}{Q_2} < |\alpha - \frac{p_1}{q_1}|$. So $\frac{p_2}{q_2}$ is a better approx then $\frac{p_1}{q_1}$.
  Repeating this process indefinitely proves the claim.
\end{proof}

\begin{theorem}[Pell-equation]
  Suppose $m \in \N$ is not a square (i.e., $m \neq n^2 \forall n \in \Z$).\\
  Then 
  \[x^2 - m y^2 = 1\]
  has infinitely many solutions $(x,y)\in \Z^2$.
\end{theorem}

\begin{proof}
  Apply Corollary 1.1.2 with $\alpha = \sqrt{m}$. So $\alpha \in \R/\Q$.
  We get $| \alpha - \frac{p}{q} | < \frac{1}{q^2}$ and $|\alpha + \frac{p}{q} | \overset{\leq}{triangle inequality} 1 + 2 \alpha$.
  Thus
  \[ |p^2 - mq^2| = q^2 | \alpha - \frac{p}{q} | \cdot |\alpha + \frac{p}{q}| < 1 + 2 \sqrt{m}. \]
  Hence, there exists $k \in \Z$ with $|k| < 1 + 2 \sqrt{m}$. such that $ p^2 - m q^2 = k$ for infinitely many $(p,q) \in \Z^2$ and $p/q$ all distinct.\\
  As $m$ is not a square we have $k \neq 0$.\\
  \\
  Let $S$ be the set of solutions $(p,q) \in \Z^2$ of $p^2 - m q^2 = k$.
  The map $S \to (\Z/k\Z) \times (\Z/k\Z)$.
  This map is not injective ($S = \infty$) hence, $\exists (p_1,q_1) \neq (p_2,q_2)$ both in $S$ such that $p_1 \cong p_2, q_1 \cong q_2 \pmod k$. (MOD)\\%todo marker
  Now we compute
  \begin{align}
    k^2 &= (p_1^2 - m q_1^2)(p_2^2 - m q_2^2)\\
    &= (p_1 + \sqrt{m}q_1)(p_2 - \sqrt{m} q_2)\\
    &= (r - \sqrt{m} s)( r + \sqrt{m} s) = r^2 - m s^2\\
    \text{where } r &= p_1 p_2 - m q_1 q_2\\
    s &= p_1 q_2 - q_1 p_2 = \frac{1}{q_1 q_2} (\frac{p_1}{q_1} - \frac{p_2}{q_2}) \neq 0.
  \end{align}
  because of (MOD) $k \divides s$. Hence, $k^2 \divides s^2$. Thus $k^2 \divides r^2$. Hence $k \divides r$.
  Then $ x = \frac{r}{k}$ and $y = \frac{s}{k}$ are both integers and
  \[x^2 - m y^2 = 1. \]
  We have one solution but we need infinitely many! To this end we replace $m$ by $md^2$ ($d \in \N$).
  The above argument yields a solution $(x^\prime,y^\prime) \in \Z^2$ of ${x^\prime}^2 - md^2 {y^\prime}^2 = 1$.
  Thus, $(x,y) = (x^\prime,dy^\prime)$ is a new solution of $x^2 - m y^2 = 1$.\\
  (Critical: $s \neq 0$)
\end{proof}

\subsection{Continued fractions}

Let $\theta \in \R$. Put $a_0 = \lfloor \theta \rfloor$. If $a_0 \neq \theta$ then we find $\theta_1 > 1$ such that
\[ \theta = a_0 + \frac{1}{\theta_1} \]
and we put $a_1 = \lfloor \theta_1 \rfloor$. If $a_1 \neq \theta_1$ then we can find $\theta_2 > 1$ such that
\[ \theta_1 = a_1 + \frac{1}{\theta_2} \]
and we put $a_ = \lfloor \theta_2 \rfloor$. This process can be continued indefinitely, unless $a_n = \theta_n$ for some $n$.
Note that $a_0$ can be zero or negative but  $a_1,a_2, a_3, \dots$ are all positive integers.\\
We call this process the \emph{continued fraction process}. The $a_i$ are called \emph{partial quotients} of $\theta$.

\begin{ex*}
  \[\theta = \frac{19}{11}\]
  Then $a_0 = \lfloor \theta \rfloor = 1$\\
  Now $\theta = \frac{19}{11} = a_0 + \frac{1}{\theta_1} = 1 + \frac{8}{11} = 1 + \frac{1}{\frac{11}{8}}$\\
  So $\theta_1 = \frac{11}{8}$.\\
  Thus $a_1 = \lfloor \theta_1 \rfloor = 1$.\\
  Now 
  \[ \theta_1 = \frac{11}{8} = a_1 + \frac{1}{\theta_2} = 1 + \frac{3}{8} = 1 + \frac{1}{\frac{8}{3}} \]
  Thus $\theta_2 = \frac{2}{3}$ and $a_2 = \lfloor \theta_2 \rfloor = 2$\\
  and so on...\\
  \\
  If the continued fraction process terminates then we have
  \begin{align}
  \theta &= a_0 + \frac{1}{\theta_1}\\
  &= a_0 + \frac{1}{a_2 + \frac{1}{\theta_2}}\\
  &= a_0 + \frac{1}{a_2 + \frac{1}{a_3 + \frac{1}{\theta_3}}}
  & \dots
  &= a_0 + \frac{1}{a_1 + \frac{1}{..}}%TODO add Kettenbruch
  \end{align}
  
  In this case we write $\theta = [a_0,\dots,a_n]$.\\
  We use the same notation when the $a_i$ are any real numbers, not necessarily integers.\\
  In particular
  \[ \theta = [a_0,\dots,a_i,\theta_{i+1}] \]
  where $a \leq i < n$.
\end{ex*}

If the continued fraction process does not terminate then we write $\theta = [a_0,a_1,a_2,\dots]$.\\
Note that in this case, for every $n \geq 0$, we have 
\[ \theta = [a_0,\dots, a_n,\theta_{n+1}] \]
where $a_0,\dots,a_n$ are integers but $\theta_{n+1}$ is not!
For $n \geq 0$ we set 
\[ \frac{p_n}{q_n} = [a_0,\dots,a_n] \] 
where $\gcd(p_n,q_n) = 1$.
We shall say that $\frac{p_n}{q_n}$ is the $n$-th convergent of $\theta$.
We will prove that $\frac{p_n}{q_n} \to \theta$ as $n \to \infty$.
Next we shall see that  $p_n,q_n > 0$ both satisfy the same simple recurrence relation $x_n = a_n x_{n-1} + x_{n-2}$ with different starting values.

\end{document}
