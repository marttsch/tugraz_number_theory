\documentclass[NumTh.tex]{subfiles}
\begin{document}

After proof of Theorem \ref{2_5_1}:

\begin{rem}
  We have not shown that $S$ is measurable.
  One could do that by showing that $\partial S$ has measure zero, and noting that every closed set is measurable.
  Why is $\vol(\partial S) = 0$?
  Take $\Lambda = k{-1} \Z$ where $k \in \N$, and $\mathcal{T}$ be associated to the basis $k{-1} e_1, \dots, k{-1}e_n$.
  The proof of Theorem \ref{2_5_1} yields
  \[ \mathcal{T} \leq c_n M \left( \frac{L}{k{-1}} + 1 \right){n-1} \leq c_n M (L+1){n-1} k{n-1} \text{.} \]
  Since 
  \[\vol(\partial S) \leq \mathcal{T} \vol F_v \leq c_n M (L+1){n-1} k{n-1} k{-n} \to 0 \text{ as } k \to \infty \text{.} \]
\end{rem}

\section{3 Algebraic Number Theory}

\begin{rem}[References]
  \begin{itemize}
    \item Daniel Marcus "Number fields" Springer
    \item Course Notes "Algebraic Number Theory" Math 2803(?) by Matt Baker (available online on his webpage.
    \item Serge Lang: Algebraic Number Theory Addison \& Wesley
  \end{itemize}
\end{rem}

\subsection{1 Introduction}

Algebraic number theory is concerned with finite field extensions of $\Q$ and their "ring of integers",
e.g., $\Q(\sqrt{2}) = \{a + b\sqrt{2}: a,b\in \Q \}$ and $\Z[\sqrt{2}] = \{ a + b \sqrt{2} : a,b \in \Z \}$.
These extensions of $\Q$ and $\Z$ are often needed; even when studying questions that initially involve only integers.

Let's consider some examples.
We start with the very simple Diophantine equation
\[ x^2 -x = y^3 \]
to be solved with $x,y \in \Z$.
We can factor the left hand side, and note that the factors $x, x-1$ are coprime.
The unique prime factorization in $\Z$ tells us that 
\begin{align*}
  x &= \pm {u^\prime}^3 = u^3 \text{ with } u = \pm u^\prime\\
  x -1 &= \pm {v^\prime}^3 = v^3 \text{ with } v = \pm v^\prime
\end{align*}
So $u^3 - v^3 = 1$.
So $(u,v) = (1,0), (0,-1)$ and thus $(x,y) = (1,0), (0,0)$.\\

So here $\Z$ itself was sufficient. Next let's consider
\[ x^2 + 2 = y^3 \]

Now the polynomial $x^2 + 2$  does not factor over $\Z$, but it does over $\Z[\sqrt{-2}] = \{a + b \sqrt{-2}: a,b \in \Z\}$
\[ x^2 + 2 = (x + \sqrt{-2}) (x - \sqrt{-2}) \]
If $x^2+2 = y^3$ then $x + \sqrt{-2}$ and $x - \sqrt{-2}$ are coprime in $\Z[\sqrt{-2}]$.
Why?
Suppose $r \in \Z[\sqrt{-2}]$ and $r \divides x +\sqrt{-1}$ and $r \divides x - \sqrt{-2}$.
Thus $r \divides 2 \sqrt{-2}$.
Let $\bar{r}$ be the complex conjugate of $r$.
Then $\bar{r} \divides \bar{x+ \sqrt{-2}} = x - \sqrt{-2}$.
Thus $r \bar{r} \divides (x + \sqrt{-2})(x-\sqrt{-2}) = x^2 + 2$ and $r \bar{r} \divides (2 \sqrt{-2})(-2 \sqrt{-2}) = 8$.
As $r \bar{r} \in \Z$ we conclude that $s r \bar{r} = 8$ with $s \in \Z[\sqrt{-2}]$ implies $s \in \Z$.
So either $r = \pm 1$ or $2 \divides r \bar{r}$.
If $2 \divides r \bar{r} \divides x^2 + 2 = y^3$ then $2 \divides y^1$ so $8 \divides y^3$ so $8 \divides x^2 + 2$ which is impossible.
So we have $r = \pm 1$ and so $x + \sqrt{-2}$, $x - \sqrt{-2}$ are coprime in $\Z[\sqrt{-2}]$. \\
We conclude also that  $\pm 1$ are the only units in $\Z[\sqrt{-2}]$.
Suppose we have a unique prime factorization in $\Z[\sqrt{2}]$.
Then we could conclude as before that there exist $u,v \in \Z[\sqrt{-2}]$ such that
\begin{align*}
  u^3 &= x + \sqrt{-2}\\
  v^3 &= x - \sqrt{-2}
\end{align*}
With $u = a +b \sqrt{-2}$ ($a,b\in \Z$) we get
\[ (a^3 - 6ab^2) + (3a^2b - 2b^3) \sqrt{-2} = x + \sqrt{-2} \]
Hence,
\begin{align*}
  a(a^2 - 6b^2) &= x\\
  b(3a^2 - 2b^2) &= 1
\end{align*}
So $b = \pm 1$. If $b = -1$ then $3a^2 -2 = -1$ which is impossible.
So $b = 1$ and $a^2 = 1$.
Hence $(x,y) = (5,3), (-5,3)$.

As we shall see later $\Z[\sqrt{-2}]$ really has a unique prime factorization.\\
\\
Now let's consider the Fermat equation
\[ x^n + y^n = z^n \; (n\geq 3) \]
We could try to apply the same strategy to show that at least one of the coordinates equals $0$.
It suffices to consider prime exponents. Let's assume $p > 2$.
We can also assume $\gcd(x,y,z) = 1$.
Now take $\Z[\zeta]$ where $\zeta = e^{-\frac{2\pi \i}{p}}$.
Then
\[ t^2 -1 = (t-1)(t-\zeta)\dots (t-\zeta^{p-1}) \text{.} \]
Replacing $t$ by $- \frac{x}{y}$ we conclude
\[ x^p + y^p = (x+y)(x + \zeta y) \dots (x + \zeta^{p-1}y) \]
We split the solutions in two classes:
\begin{enumerate}
  \item $(x,y,z)$ with $p \nmid xyz$
  \item $(x,y,z)$ with $p$ divides exactly one of the coordinates.
\end{enumerate}
We consider only solutions as in 1).
For $p = 3$ we note that $x^3 + y^3 = z^3$ is impossible as each of these cubes is $\pm 1 \pmod 9$.
So assume $p>3$.
Suppose that there exists a unique prime factorization in $\Z[\zeta]$. Then one can show that $x + \zeta y = \varepsilon \alpha^p$ where $\varepsilon$ is a unit and $\alpha \in \Z[\zeta]$.
Then one can show that if $x+ \zeta y = \varepsilon \alpha^p$ and $p \nmid xy$ then $x \equiv y \pmod p$.
As 
\[ x^p + (-z)^p = (-y)^p\]
we also conclude $x  \equiv -z \pmod p$.
So 
\[ 2x^p \equiv x^p + y^p = z^p \equiv (-x)^p \pmod p \text{.} \]
So $p \divides \zeta x^p$.
As $p > 3$ and $p \nmid x$ we get a contradiction; so no solutions of class 1), provided $\Z[\zeta]$ has a unique prime factorization.

The latter holds for $p < 23$ but it "usually" fails.
To solve Fermat completely Wiles and Wiles-Taylor used the theory of elliptic curves.
On the other hand the Catalan equation
\[ x^n - y^m = 1 \; (n,m > 1,\:x,y >0) \]
was solved completely by Mihailescu using algebraic number theory.

\end{document}
