\documentclass[NumTh.tex]{subfiles}
\begin{document}

\section{Algebraic Number Theory}

\begin{rem}[References]
  \begin{itemize}
    \item Daniel Marcus "Number fields" Springer
    \item Course Notes "Algebraic Number Theory" Math 2803(?) by Matt Baker (available online on his webpage)
    \item Serge Lang: Algebraic Number Theory Addison \& Wesley
  \end{itemize}
\end{rem}

\subsection{Introduction}

Algebraic number theory is concerned with finite field extensions of $\Q$ and their "ring of integers",
e.g., $\Q(\sqrt{2}) = \{a + b\sqrt{2}: a,b\in \Q \}$ and $\Z[\sqrt{2}] = \{ a + b \sqrt{2} : a,b \in \Z \}$.
These extensions of $\Q$ and $\Z$ are often needed; even when studying questions that initially involve only integers.

Let's consider some examples.
We start with the very simple Diophantine equation
\[ x^2 -x = y^3 \]
to be solved with $x,y \in \Z$.
We can factor the left hand side, and note that the factors $x, x-1$ are coprime.
The unique prime factorization in $\Z$ tells us that 
\begin{align*}
  x &= \pm {u^\prime}^3 = u^3 \text{ with } u = \pm u^\prime\\
  x -1 &= \pm {v^\prime}^3 = v^3 \text{ with } v = \pm v^\prime
\end{align*}
So $u^3 - v^3 = 1$.
So $(u,v) = (1,0), (0,-1)$ and thus $(x,y) = (1,0), (0,0)$.\\

So here $\Z$ itself was sufficient. Next let's consider
\[ x^2 + 2 = y^3 \]

Now the polynomial $x^2 + 2$  does not factor over $\Z$, but it does over $\Z[\sqrt{-2}] = \{a + b \sqrt{-2}: a,b \in \Z\}$
\[ x^2 + 2 = (x + \sqrt{-2}) (x - \sqrt{-2}) \]
If $x^2+2 = y^3$ then $x + \sqrt{-2}$ and $x - \sqrt{-2}$ are coprime in $\Z[\sqrt{-2}]$.\\
Why?
Suppose $r \in \Z[\sqrt{-2}]$ and
\begin{align*}
  r &\divides x + \sqrt{-2} \text{ and } \\
  r &\divides x - \sqrt{-2} \text{.}
\end{align*}
Thus $r \divides 2 \sqrt{-2}$.\\
Let $\bar{r}$ be the complex conjugate of $r$.
Then 
\[ \bar{r} \divides \overbar{x+ \sqrt{-2}} = x - \sqrt{-2} \text{.} \]
Thus
\begin{align*}
  r \bar{r} &\divides (x + \sqrt{-2})(x-\sqrt{-2}) = x^2 + 2 \text{ and } \\
  r \bar{r} &\divides (2 \sqrt{-2})(-2 \sqrt{-2}) = 8 \text{.}
\end{align*}
As $r \bar{r} \in \Z$ we conclude that $s r \bar{r} = 8$ with $s \in \Z[\sqrt{-2}]$ implies $s \in \Z$.
So either 
\[ r = \pm 1 \text{ or } 2 \divides r \bar{r} \text{.} \]
If 
\[ 2 \divides r \bar{r} \divides x^2 + 2 = y^3 \] 
then 
\begin{align*}
  2 &\divides y^1 \\
  \implies 8 &\divides y^3 \\ 
  \implies 8 &\divides x^2 + 2
\end{align*}
which is impossible since $x^2 \in \{ \overbar{0},\overbar{1} \} \bmod 4$.\\
So we have $r = \pm 1$ and so 
\[ x + \sqrt{-2} \text{ and } x - \sqrt{-2} \] 
are coprime in $\Z[\sqrt{-2}]$. \\
We conclude also that  $\pm 1$ are the only units in $\Z[\sqrt{-2}]$.
Suppose we have a unique prime factorization in $\Z[\sqrt{-2}]$.
Then we could conclude as before that there exist $u,v \in \Z[\sqrt{-2}]$ such that
\begin{align*}
  u^3 &= x + \sqrt{-2}\\
  v^3 &= x - \sqrt{-2}
\end{align*}
With $u = a +b \sqrt{-2}$ ($a,b\in \Z$) we get
\[ u^3 = (a^3 - 6ab^2) + (3a^2b - 2b^3) \sqrt{-2} = x + \sqrt{-2} \]
Hence,
\begin{align*}
  a(a^2 - 6b^2) &= x\\
  b(3a^2 - 2b^2) &= 1
\end{align*}
So $b = \pm 1$. If $b = -1$ then $3a^2 -2 = -1$ which is impossible.
So $b = 1$ and $a^2 = 1$.
Hence $(x,y) \in \{(5,3), (-5,3)\}$.
\\
As we shall see later $\Z[\sqrt{-2}]$ really has a unique prime factorization.\\
\\
Now let's consider the Fermat equation
\[ x^n + y^n = z^n \; (n\geq 3) \]
We could try to apply the same strategy to show that at least one of the coordinates equals $0$.
It suffices to consider prime exponents. Let's assume $p > 2$.
We can also assume $\gcd(x,y,z) = 1$.
Now take $\Z[\zeta]$ where $\zeta = e^{-\frac{2\pi \i}{p}}$.
Then
\[ t^p - 1 = (t-1)(t-\zeta)\dots (t-\zeta^{p-1}) \text{.} \]
Replacing $t$ by $- \frac{x}{y}$ we conclude
\[ x^p + y^p = (x+y)(x + \zeta y) \cdots (x + \zeta^{p-1}y) \]
We split the solutions in two classes:
\begin{enumerate}
  \item $(x,y,z)$ with $p \nmid xyz$
  \item $(x,y,z)$ with $p$ divides exactly one of the coordinates.
\end{enumerate}
We consider only solutions as in 1).
For $p = 3$ we note that $x^3 + y^3 = z^3$ is impossible as each of these cubes is $\pm 1 \bmod 9$.
So assume $p>3$.
Suppose that there exists a unique prime factorization in $\Z[\zeta]$. 
Then one can show that 
\[ x + \zeta y = \varepsilon \alpha^p \] 
where $\varepsilon$ is a unit and $\alpha \in \Z[\zeta]$.
Then one can show that if
\begin{align*}
  x+ \zeta y &= \varepsilon \alpha^p \text{ and } \\ 
  p &\nmid xy
\end{align*}
then
\begin{align*}
  x \equiv y \bmod p \text{.}
\end{align*}
As 
\[ x^p + (-z)^p = (-y)^p\]
we also conclude $x  \equiv -z \bmod p$.
So 
\[ 2x^p \equiv x^p + y^p = z^p \equiv (-x)^p \pmod p \text{.} \]
So 
\[ p \divides 3 x^p \text{.} \]
As $p > 3$ and $p \nmid x$ we get a contradiction; so no solutions of class 1), provided $\Z[\zeta]$ has a unique prime factorization.

The latter holds for $p < 23$ but it "usually" fails.
To solve Fermat completely Wiles and Wiles-Taylor used the theory of elliptic curves.
On the other hand the Catalan equation
\[ x^n - y^m = 1 \; (n,m > 1,\:x,y >0) \]
was solved completely by Mih\v{a}ilescu using algebraic number theory.

\end{document}
