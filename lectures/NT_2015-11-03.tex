\documentclass[NumTh.tex]{subfiles}
\begin{document}

\subsection{Theorems of Thue- Siegel and Roth}
In Section 1 we have seen that $\infty$-many solutions $\frac{p}{q}$ to $\abs{\sqrt2-\frac{p}{q}}<\frac{1}{q^2}$  leads to $\infty$-many solutions $\left(x,y\right)\in\Z^2$ of $x^2-2y^2=1$. What about $x^3-2y^3=1$? Starting as for $x^2-2y^2$ we get
$$y^3 \abs{\frac{x}{y}-2^{1⁄3}} \underbrace{\abs{\frac{x}{y}-2^{1/3} \omega}}_{\ge\operatorname{Im}{\omega}} \underbrace{\abs{\frac{x}{y}-2^{1/3} \omega^2}}_{{\ge{\operatorname{Im}{\omega}}}}$$
where $\omega=e^{\frac{2\pi i}{3}}$.

So to get boundedness of $x^3-2y^3$ for $\infty$-many $\left(x,y\right)$ we need $\exists c>0$ such that
$$\abs{\frac{x}{y}-2^{1/3}}<\frac{c}{y^3}$$
has $\infty$-many solutions $\left(x,y\right)\in\Z\times\N$. 

Theorem 1.3.3 tells us that we would be extremely lucky if that were the case. And erven if so, we still would lack the group structure for $\Z+\sqrt 2 \Z$ (closed under multiplication but $\Z+2^{1/3} \Z$ is not). On the other hand, suppose we could show that
$$\abs{\frac{x}{y}-2^{1/3}}<1/y^\lambda$$
has only finitely many solutions $\left(x,y\right)\in\Z\times\N$ for some fixed $\lambda<3$. As $x^3-2y^3=1$, and $y\ne0$ yields: 
$$\abs{\frac{x}{y}-2^{1/3}}<\frac{1}{2^{1/3} {\left(\operatorname{Im}\omega\right)}^2 y^3}$$
We would conclude that $x^3-2y^3=1$ has only finitely many solutions $\left(x,y\right)\in\Z^2$. Note that $"deg" 2^{1/3}=3 (D(x)=x^3-2)$ and so Liouville’s Theorem yields only $\lambda=3$ not $\lambda<3$. So the big challenge is to improve Liouville’s Theorem. After Liouville it has taken 65 years until the first breakthrough was obtained by Axel Thue in 1909. 

\begin{theorem}[Thue]\label{1_4_1}
Let $\alpha$ be a real algebraic number of degree $d\ge 2$, and let $\lambda>\frac{d}{2}+1$. Then $\exists c=c(\alpha,\lambda)>0$ such that
$$\abs{\alpha-\frac{p}{q}}>\frac{c}{q^\lambda} ,\quad\forall(p,q)\in\Z\times\N.$$
\end{theorem}

\begin{itemize}
\item Note that for $d=2$ Liouville is stronger
\item Given $\alpha$ and $\lambda$ there is no method to determine a feasible value for $c$. This is in stark contrast to Liouville’s Theorem. 
\end{itemize}

Just as for $x^3-2y^3=1$ one can now very easily show that if $f(X,Y)=a_0 (X-\alpha_1 Y)\cdots(X-\alpha_d Y)\in\Q[X,Y]$ with $a_0\ne0, d\ge 3$, and $\alpha_1,\dots,\alpha_d$ pairwise distinct, and $b\in\Q\setminus\set{0}$, then
$$f(x,y)=b$$
has only finitely many solutions $\left(x,y\right)\in\Z^2$. \\
Wrong if $d=2$:
$$X^2-2Y^2=1$$
or $b=0$:
$$X^3-Y^3=0$$
or $\alpha_1,\dots,\alpha_d$ not pairwise distinct: 
$$(X-Y)^5=1$$
We will show that Theorem 1.4.1 implies even the following stronger result. 

\begin{theorem}[Generalized Thue equations]\label{1_4_2}
Let $f(X,Y)=a_0 \left(X-\alpha_1 Y\right)\cdots\allowbreak\left(X-\alpha_d Y\right)\in\Q\left[X,Y\right]$ with $a_0\ne 0, d\ge 3$ and $\alpha_1,\cdots,\alpha_d$ pairwise distinct. Let $g(X,Y)\in\Q\left[X,Y\right]$ of total degree $<\frac{d}{2}-1$. 
\\
Then there are only finitely many $(X,Y)\in\Z^2$ with
$$f(x,y)=g(x,y)$$
and
$g(x,y)\ne 0$.
\end{theorem}

\begin{ex}
$$x^5-2y^5=x-y$$
has only finitely many solutions $\left(x,y\right)\in\Z^2$. Indeed if $x-y=0$ then $x^5-2y^5=0$ thus $x=y=0$. 
Note Theorem can go wrong if $\alpha_1=\alpha_2$: 
$$(X^2-2Y^2 )^2=1.$$
\end{ex}

\begin{proof}[Proof (assuming Theorem 1.4.1)]
If $y=0$ then we have at most d possibilities for x. So we can assume $y\ne 0$. We claim that
$$\abs{x}\le c_1 \abs{y}$$
for some $c_1=c_1 (f,g)$. Clearly true when $\abs{x}\le\abs{y}$, so let's assume $\abs{x}>\abs{y}$. Then we write
$$f(x,y)=\sum_{i=0}^d {a_i x^{d-i} y^i}=\sum_{j+k\le d-1} {b_{jk} x^j y^k}=g(x,y)$$
\todo{sum $j+k \leq d-1$ or $d+1$ or $\frac{d}{2}-1$}
Dividing by $x^{d-1}$ yields
$$a_0 x=-\sum_{i=1}^d {a_i  \frac{y^i}{x^{i-1}}}+\sum_{j+k \leq d-1} {b_{jk} x^{j-d+1} y^k}$$
We have
$$\abs{\frac{y^i}{x^{i-1}}}\le\abs{y}$$
and
$$\abs{\frac{y^k}{x^{d-1-j}}}\le\abs{y}^{j+k-(d-1)}\le 1$$
Therefore $\abs{x}\le c_1 \abs{y}$, e.g. with $c_1=\frac{1}{\abs{a_0}} \left(\sum \abs{a_i} +\sum \abs{b_{jk}} \right)+1$.
From
$$f(x,y)=g(x,y),(\star)$$
we get
$$\abs{\alpha_0}\prod_{i=1}^d {\abs{\frac{x}{y}-\alpha_i}} \le c_2 \abs{y}^{e-d}$$
where $c_2=c_2\left(c_1,g\right)$ and $e<\frac{d}{2}-1$. So assume $(\star)$ has $\infty$-many solutions $\left(x,y\right)\in\Z^2$. Then $\exists i$, say $i=1$, such that $\abs{\frac{x}{y}-\alpha_1}\le\mu:=\frac{1}{2}\displaystyle{\min_{j\ne 1}{\left\{\abs{\alpha_j-\alpha_1}\right\}}}>0$ for $\infty$-many $(x,y)$ of these solutions of $(\star)$. 
Now
$$\abs{\frac{x}{y}-\alpha_j}\ge\abs{\abs{\alpha_j-\alpha_1}-\abs{\frac{x}{y}-\alpha_1}}\ge 2\mu-\mu=\mu>0$$
Hence, we conclude
$$\abs{\frac{x}{y}-\alpha_1}\le \frac{c_2}{\abs{a_0}}\mu^{1-d} \abs{y}^{e-d},(\star\star)$$
for these solutions $\left(x,y\right)$. Here we can assume $y>0$ (just replace $x$ by $-x$). Now let $d_1$ be the degree of $\alpha_1$. As $f(x,1)\in\Q[x]$, $f(x,1)\ne 0$ and $f(\alpha_1,1)=0$. Thus $d_1\le d$. Moreover, $d-e>\frac{d}{2}+1$ and this $\exists\lambda$ such that
$$d-e>\lambda>\frac{d_1}{2}+1.$$
If $d_1\ge 2$ then Theorem 1.4.1 implies that $(\star)$ has only finitely many solutions $\left(x,y\right)\in\Z^2$. Finally suppose $d_1=1$. Then $\alpha_1=\frac{p}{q}$, and $(\star\star)$ yields: 
$$\abs{x-\frac{p}{q} y} \le c_3 y^{e-d+1}\le c_3 y^{-\frac{d}{2}}.$$
Thus $x=\frac{p}{q} y=\alpha_1 y$ for $y$ large enough. But then $0=f(x,y)=g(x,y)$ a contradiction. 
\end{proof}

After Thue came Siegel (1921) who improved the exponent $\frac{d}{2}+1$ to $2\sqrt{d}$. This was slightly improved by Dyson and Gelfand (1947) to $\sqrt{2d}$. Finally in 1955 came Roth: 

\begin{theorem}[(Roth)]\label{1_4_3}
Let $\alpha$ be a real, algebraic irrational number, and $\lambda>2$. Then $\exists c=c(\alpha,\lambda)>0$ such that
$$\abs{\alpha-\frac{p}{q}}\ge \frac{c}{q^\lambda} ,\quad\forall(p,q)\in\Z\times\N.$$
\end{theorem}

By Corollary 1.1.2 $\lambda>2$ is best-possible. 
But if we allow more general functions $\phi(q)$, not only powers of $q$, then an improvement might be possible. However, since 1955 nobody was able to replace $q^{-\lambda}$ by a function $\phi(q)$ that decays more slowly, e.g. $\phi(q)=q^{-2} {\left(\log{q}\right)}^{-1}$. 
\\
However, back to the case where $\phi(q)$ is a power of $q$. 
\\
From Theorem 1.3.3. we know that for a generic real $\alpha$
$$\abs{\frac{p}{q}-\alpha}<q^{-\lambda}$$
has only finitely many solutions $p,q\in\Z\times\N$ provided $\lambda>2$. Any by Corollary 1.1.2 every irrational real number has $\infty$-many solutions when $\lambda=2$. And so from Roth's Theorem we see an algebraic irrational behaves \grqq essentially\grqq~like a generic number. 

\end{document}
